<<<<<<< HEAD
\subsection{Introduction}
\begin{itemize}
	\item General introduction
	\item Why is it interesting to study the fluid dynamics of the QGP
	\item What has been learnt so far at LHC 
	\item What questions remain
\end{itemize}
% text from Stefan
In a fluid dynamic description of heavy ion collisions,  one can understand the azimuthal harmonic flow coefficients $v_n$ as a response to deviations of the initial state from an isotropic form. Mathematically,  one can formulate this in terms of response functions that describe how the solution of the fluid dynamic evolution equations,  as well as resulting experimental observables such as azimuthal particle distributions,  get modified when the initial values of the fluid fields are changed. In cosmology,  closely related transfer functions are used to describe the response to initial density perturbations created by quantum fluctuations during inflation. Similar as in cosmology,  one can formulate linear as well as non-linear response functions for various fluid dynamic fields and observables. Moreover,  statistical symmetries of the initial state,  in particular azimuthal rotation invariance and the approximate Bjorken boost symmetry can be used to classify different perturbations. In the simplest implementation,  linear response functions describe the linear response of flow coefficients to eccentricities $v_n \sim\epsilon_n$,  while the quadratic part describes terms like $v_n \sim \epsilon_a\epsilon_b$ where symmetry reasons imply $|n|=|a\pm b|$. Response functions can not only be used to study deviations from azimuthal rotation symmetry but also for deviations from (approximate) Bjorken boost invariance,  vanishing baryon number density,  for electric fields and so on. Quite generally,  response functions cary interesting information about fluid properties such as thermodynamic and transport properties. Where the response functions are known,  one can infer properties of the initial state by reverse engineering.

Experimentally,  one can constrain properties of response functions indirectly via measurements of various particle correlation functions. Differential information as well as good statistics are valuable here. It is particularly interesting to compare situations with strong deviations from a symmetry (such as peripheral collisions for the case of azimuthal rotation invariance) to situations with small deviations (such as central collisions). This should allow to distinguish linear from quadratic and higher order response functions.

One may in principle also use classifications of events according to their net baryon number  in the central rapidity range or other event selection criteria. This idea is also known as event engineering and might become even more feasible with more statistics.

A general problem in the context of the fluid dynamic description of heavy ion collisions is that the initial state is not very well known. For the time being,  one must improve the knowledge of the initial state as well as the fluid properties of the quark-gluon plasma (which in turn determine the response functions) simultaneously by detailed comparison between experimental results and theoretical calculations. Dependence on external parameters like system size and collision energy as well as differential information such as on centrality,   or particle identification will be helpful to make further progress here. In addition to this,  theoretical techniques and models must be improved further.

