\section{Initial conditions} \label{sec:initialstate}
As already alluded to in Section \ref{sec:macro}, the intial state of heavy ion collisions is the major source of uncertainty when it comes to extracting transport properties of the produced medium. This has to do with the fact that calculations of the exact geometry and its fluctuations involve nonperturbative physics. The available descriptions for the initial state thus range from very simplistic models that assign deposited energy densities based on the wounded nucleons or binary collisions determined in a Monte-Carlo Glauber prescription, to classical effective theories of QCD that are valid in the high energy limit. The latter are certainly closer to first principles calculations, and should provide a realistic description of the initial state assuming that the high energy limit is a good approximation.

The major ingredient that needs to be provided by an initial state model is the geometry of the interaction region in the plane transverse to the beam. It is entirely dominated by the positions of incoming nucleons whose fluctuations also play an important role. Observables are also sensitive to the detailed way the energy is deposited given a distribution of wounded nucleons, and the constraints on this from data are surprisingly robust \cite{Moreland:2018gsh}. Using the Trento model it was found that the product of thickness functions of the two nuclei at every given transverse position gives the best description of the data (compared to a conventinal wounded nucleon model or other combinations of thickness functions). This type of energy deposition is very much what the EKRT \cite{Niemi:2015qia} and IP-Glasma \cite{Schenke:2012wb,Schenke:2012fw} models include, explaining their success in describing a wide range of observables \cite{Niemi:2015qia,Gale:2012rq}.

Fluctuations of wounded nucleon positions contribute the dominant effect to fluctuations of the initial geometry in heavy ion collisions. Smaller scale fluctuations (emerging e.g. from color charge fluctuations in the IP-Glasma model) have been shown to not significantly affect most observables measuring the momentum anoisotropy of produced particles \cite{Gardim:2017ruc}. The factorization breaking ratio shows some sensitivity and it should be pointed out that multiplicity fluctuations are also influenced by the existence of color charge fluctuations.

More recently, indications that the nucleon should possess a substructure of hotspots have emerged. The IP-Glasma model was unable to describe flow coefficients in p+Pb collisions assuming a round proton \cite{Schenke:2014zha}. Including subnucleonic fluctuations constrained by incoherent diffractive $J/\psi$ production data from HERA \cite{Mantysaari:2016ykx} (which also require a fluctuating proton geometry) improved the agreement with the p+Pb data significantly \cite{Mantysaari:2017cni}. Recent Bayesian analyses also found that nucleon substructure is necessary to simultaneously describe p+Pb and Pb+Pb bulk observables \cite{Moreland:2018gsh}.
It should be noted, however, that the size scales for nucleons and subnuclear hot spots extracted in that work are significantly larger than those obtained from comparison to HERA data in the IP-Glasma model.

Initial conditions for hydrodynamic simulations have to provide, in principle, all components of the energy momentum tensor as a function of spatial position (and initial conditions for other conserved charges, if considered). This includes, apart from the always included energy density distribution, the initial flow as well as initial viscous corrections. Initial flow is included in many recently developed models, that either assume free streaming \cite{Moreland:2018gsh}, including Yang-Mills evolution, which is close to free streaming \cite{Gale:2012rq} or an initial flow distribution motivated by strong coupling calculations \cite{Weller:2017tsr}. Initial viscous corrections are often set to zero. Only a few works \cite{Mantysaari:2017cni,Schenke:2018fci,Moreland:2018gsh} include the full viscous stress tensor provided by the initial state description.

Since the initial state models that provide the entire $T^{\mu\nu}$ all switch from essentially a freely streming system to strongly interacting hydrodynamics at a fixed time $\tau$, that transition is somewhat abrupt and not very physical. To improve over this situation an intermediate step using effective kinetic theory has been introduced \cite{Kurkela:2018wud,Kurkela:2018vqr}. This procedure allows for a somewhat smoother matching but has yet to be used in full fledged hydrodynamic simulations. The effect on observables in heavy ion collisions is likely small, while in small systems, such as p+A a larger effect can be expected.

As already discussed in Section \ref{sec:macro}, the choice of initial state has a significant effect on the extraction of transport coefficients. A more compact initial state and the presence of initial flow lead to a larger transverse flow, which requires a larger bulk viscosity to compensate for it and produce agreement with experimental data \cite{Schenke:2018fci}. Also, the models' eccentricities will affect the final momentum anisotropies, influencing the extracted shear viscosity to entropy density ratio. Two possible attempts to solve this problem have been pursued: Constrain an initial state description as well as possible using data from experiments other than heavy ion collisions (e.g. e+p scattering data, which will hopefully be extended to e+A in a future electron ion collider facility), or perform a combined Bayesian analysis of all parameters, including those of the initial state, to find the best fit for all transport coefficients along with the initial state description.

As mentioned above, at the moment the two approaches lead to some similar features of the initial state (product of thickness functions, presence of subnucleon structure), but also show discrepancies (size of the nucleon and sub-nucleon scales along with the size of the extracted bulk viscosity). Only when the two methods converge for all features of the initial state and medium properties, can one confidently declare that the initial state as well as the transport properties of the QGP are understood.