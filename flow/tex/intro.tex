\subsection{Introduction}

%High energy nuclear collisions at RHIC and the LHC produce a relativistic 
%  fluid composed of QCD matter, i. e. quarks and gluons. 
It is particularly interesting to study the macroscopic properties of 
  the QGP fluid because - at least conceptually - they are fully fixed 
  by the microscopic properties of a renormalizable, fundamental quantum 
  field theory, namely QCD. 
One can study here experimentally, as well as theoretically, how 
  macroscopic material properties result from microscopic field 
  theoretic processes. 
Many theoretical methods ranging from perturbative to non-perturbative 
  techniques are being developed to understand this in detail and one can 
  expect that the insights gained here will be valuable for many related 
  problems in fields ranging from condensed matter theory to cosmology in 
  the future.  
%Currently the field is still largely driven by experimental progress. 
%Theoretically, the connection between microscopic QCD physics and the 
%  macroscopic properties of the quark-gluon plasma are being understood 
%  step by step but a satisfactory understanding is still missing. 
Many different fronts of research are being explored at the moment. 
This ranges from conceptual questions on how to consistently formulate 
  relativistic fluid dynamics or how to solve quantum field theory in 
  non-equilibrium situations to very concrete practical questions about 
  the thermodynamic and transport properties (such as viscosities or 
  conductivities) of the QGP. 
The description of the initial state -- prior to QGP formation --
  and the mechanism by which the products of the collision rapidly thermalize
  are also under investigation.
Besides the role of strong interactions, also electromagnetic interactions 
  and in particular the role of magnetic fields and the production of 
  photons and dileptons are being explored. 
Other fronts of research concern the role of quantum anomalies, chirality 
  and vorticity or the dependence of collective behavior on system size 
  (nucleus-nucleus versus proton-nucleus and proton-proton collisions), 
  on centrality and collision energy, the initial state directly after 
  the collision, or various types of fluctuations. 
These challenges are discussed in more detail in the following subsections.


