\subsection{Introduction}

%High energy nuclear collisions at RHIC and the LHC produce a relativistic 
%  fluid composed of QCD matter, i. e. quarks and gluons. 
It is particularly interesting to study the macroscopic properties of 
  the QGP fluid because - at least conceptually - they are fully fixed 
  by the microscopic properties of a renormalizable, fundamental quantum 
  field theory, namely QCD. 
One can study here experimentally, as well as theoretically, how 
  macroscopic material properties result from microscopic field 
  theoretic processes. 
Many theoretical methods ranging from perturbative to non-perturbative 
  techniques are being developed to understand this in detail and one can 
  expect that the insights gained here will be valuable for many related 
  problems in fields ranging from condensed matter theory to cosmology in 
  the future.  
%Currently the field is still largely driven by experimental progress. 
%Theoretically, the connection between microscopic QCD physics and the 
%  macroscopic properties of the quark-gluon plasma are being understood 
%  step by step but a satisfactory understanding is still missing. 
Many different fronts of research are being explored at the moment. 
This ranges from conceptual questions on how to consistently formulate 
  relativistic fluid dynamics or how to solve quantum field theory in 
  non-equilibrium situations to very concrete practical questions about 
  the thermodynamic and transport properties (such as viscosities or 
  conductivities) of the quark-gluon plasma. Besides the role of strong 
  interactions, also electromagnetic interactions and in particular the 
  role of magnetic fields and the production of photons and dileptons 
  are being explored. 
Other fronts of research concern the role of quantum anomalies, chirality 
  and vorticity or the dependence of collective behavior on system size 
  (nucleus-nucleus versus proton-nucleus and proton-proton collisions), 
  on centrality and collision energy, the initial state directly after 
  the collision, or various types of fluctuations. 
These issues are discussed in more detail in the following subsections.





\subsection{QCD Equation of State}
The QCD equation of state, accessible in high-energy collisions (and in 
  the region around mid-rapidity) is that at vanishing baryon chemical 
  potential, and it has been established for some time that it features 
  a crossover transition to a chirally symmetric quark gluon plasma~\cite{Aoki:2006we}. 
Most recent lattice calculations~\cite{Steinbrecher:2018phh} have determined 
  the cross-over temperature to be $T_c\simeq 156.5 \pm 1.5\,{\rm MeV}$. 
Recent efforts are also exploring the equation of state at finite $\mu_B$, 
  which at LHC would have relevance mainly at very forward rapidities. 
Here, because of the sign problem, methods like Taylor expansion~\cite{Kaczmarek:2011zz,Endrodi:2011gv,Bazavov:2015zja,Bonati:2018nut} 
  or imaginary chemical potentials~\cite{Cea:2014xva,Bonati:2014kpa,Bonati:2015bha,Bellwied:2015rza,Cea:2015cya} 
  have to be employed. 
%%
To employ lattice QCD based equations of state in hydrodynamic caluclations, 
  they need to be matched to a hadron resonance gas model at low temperatures, 
  to cover the entire temperature range from zero to the maximally 
  reached teperature. 
Various equations of state~\cite{Huovinen:2009yb, Borsanyi:2013bia, Bazavov:2014pvz}, 
  using different lattice data and different matching conditions have been used 
  in simulations. 
A comparison of some of them can be found in~\cite{Moreland:2015dvc}, where 
  the sensitivity of observables to the choice of equation of state 
  was studied. 
While the mean transverse momentum varied by approximately 3\% when using 
  different equations of state, $v_2$ and $v_3$ changed by 8\% and 15\%, 
  respectively.
Because different lattice data was used, and matching to the hadron resonance 
  gas was performed at a higher temperature, the s95p-v1 lattice 
  parameterization has a smaller speed of sound in an extended temperature 
  regime, compared to the other equations of state. 
This leads to a reduced amount of flow. 
%%
Such differences will affect the precise extraction of transport coefficients, 
  such as $\eta/s$ and $\zeta/s$. 
Fortunately, the newer lattice QCD equations of state from the hotQCD 
  collaboration~\cite{Bazavov:2014pvz} and the Wuppertal-Budapest 
  collaboration~\cite{Borsanyi:2013bia} lead to differences only on the 
  percent level for the studied observables.
%%
Currently, the experimental data can not easily teach us directly about 
  the equation of state, because of the uncertainty in the transport 
  coefficients. 
One possibility would be to extend state of the art Bayesian techniques~\cite{Moreland:2018gsh} 
  to include free parameters describing the equation of state and fit them 
  along with other free parameters such as shear and bulk viscosities. 



\subsection{Shear viscosity of hot nuclear matter}
Ideal fluid dynamics has been very succesful in describing a variety of 
  bulk observables in heavy ion collisions~\cite{Kolb:2003dz,Huovinen:2003fa,Hirano:2002ds}, 
  indicating early on that the shear and bulk viscosities of the produced 
  matter cannot be large. 
Calculations in the strong coupling limit using gauge gravity duality have 
  found a value of $\eta/s=1/4\pi$ for an $N = 4$ super Yang-Mills 
  quantum system. 
This value was significantly smaller than the $\eta/s$ obtained in 
  perturbative QCD calculations, which were, however, plagued by 
  significant errors, mainly resulting from uncertainties in the 
  relevant scales~\cite{Arnold:2003zc}. 
Recently, such perturbative calculations have been extended to include 
  next-to-leading order corrections and aa significant reduction compared 
  to the leading order result was found~\cite{Ghiglieri:2018dgf}: 
At temperatures of the order of the QCD transition the NLO $\eta/s$ 
  is smaller by a factor of $5$ compared to the LO result, and reaches 
  values of approximately $2/4\pi$. 
Extractions of transport coefficients from lattice QCD calculations~\cite{Nakamura:2004sy,Meyer:2007ic,Pasztor:2018yae} 
  are extraordinarily hard, because the spectral function for use in 
  the Kubo formula follows from a difficult inversion of an integral 
  transform of a correlator of the energy momentum tensor. 
Most recent calcaultations for a pure gluon plasma find $\eta/s=0.17\pm 0.02$ at $T=1.5\,T_c$.
Apart from above direct calculations of the shear viscosity to entropy 
  density ratio, by means of hydrodynamic simulations its value can be 
  extracted by comparison to experimental data~\cite{Gale:2013da,Heinz:2013th}. 
This method suffers mainly from uncertainties in the initial state (see Section \ref{sec:initialstate}). 
The latest constraints come from simulations using the IP-Glasma initial 
  state~\cite{Schenke:2012wb,Schenke:2012fw}, the EKRT model~\cite{Niemi:2015qia} 
  and Bayesian analyses employing the Trento initial state model~\cite{Moreland:2018gsh}. 
IP-Glasma + hydrodynamic simulations including bulk and shear viscosity 
  find for the shear viscosity an effective constant value of $\eta/s=0.095$ 
  at the $\sqrt{s}=2.76\,{\rm TeV}$ LHC energy.% and $\eta/s=0.06$ at the top RHIC energy. 
EKRT simulations find good agreement with LHC data using a constant value 
  of $\eta/s=0.2$ and also certain temperature dependent $\eta/s$ values.
Finally, the latest Bayesian analyses of $\sqrt{s}=5\,{\rm TeV}$ Pb+Pb 
  collisions at the LHC using the Trento initial state determined an 
  approximately linearly rising $\eta/s$ with temperature, with 
  $(\eta/s)(T=150\,{\rm MeV})\approx 0.09$ and 
  $(\eta/s)(T=300\,{\rm MeV})\approx 0.16$.
The method of extracting $\eta/s$ using hydrodynamic simulations and comparison 
  to experimental data thus leaves us with an uncertainty of approximately 
  a factor of $3$ at this point. 
Comparison to more observables and additional collision systems, such as Ar+Ar or O+O, 
  that would allow to independently better 
  constrain features of the initial state and medium properties will 
  hopefully reduce this uncertainty in the future. 


\subsection{Bulk viscosity of hot nuclear matter}
There are several theoretical indications that bulk viscosity could play 
  an important role in the transition region of QCD (see~\cite{Ryu:2017qzn} 
  and references therein). 
Similar to the case of shear viscosity, bulk viscosity over entropy density 
  ratios have been calculated using holography in the strong coupling regime 
  using extensions to non-conformal theories~\cite{Buchel:2007mf,Finazzo:2014cna}. 
The temperature dependent $\zeta/s$ features a peak of value 0.05 around a 
  temperature of $\sim 160\,{\rm MeV}$. 
Perturbative calculations have shown that the simple estimate 
  $\zeta\approx 15 \eta(1/3-c_s^2)^2$~\cite{Horsley:1985dz} is parametrically 
  correct for QCD~\cite{Arnold:2006fz}, where $(1/3-c_s^2)$ is the deviation 
  from conformal symmetry. 
Lattice calculations using the Kubo formula extract large values of $\zeta/s$ 
  around $T_c$~\cite{Karsch:2007jc,Meyer:2007dy}, with large uncertainties for 
  the value at $T_c$, which is of order 1~\cite{Kharzeev:2007wb}, and exhibit 
  a fast drop with increasing $T$.
%%
Parametrizations of the bulk viscosity over entropy density's temperature 
  dependence were performed in~\cite{Denicol:2009am} with input from~\cite{Karsch:2007jc} 
  for the QGP phase and~\cite{NoronhaHostler:2008ju} for the hadronic phase. 
This parametrization features a peak of $\zeta/s$ around $T\approx 180\,{\rm MeV}$, 
  reaching approximately a value of 0.3. 
It has been used in various hydrodynamic simulations employing the IP-Glasma 
  initial state and led to good agreement of the calculated mean transverse 
  momentum with experimental data. 
Calculations using other initial states have reported the need for smaller bulk 
  viscosity over entropy density values. 
For example in a recent Bayesian analyss using the Trento initial state model, 
  $\zeta/s$ peaks at a value about 10 times smaller. 
In~\cite{Schenke:2018fci} it was discussed how the compact size and initial 
  flow present in the IP-Glasma initial state contribute to the need for a 
  larger bulk viscosity compared to other initial state models. 
%%
Viscous corrections to the distribution function at freeze-out affect the 
  low-$p_T$ part of the spectrum more for bulk viscosity than for the shear 
  part~\cite{Bozek:2009dw,Paquet:2015lta}. Consequently, the uncertainties 
  resulting from bulk viscous corrections are typically larger than for shear 
  when studying $p_T$ integrated observables. 



%\subsection{Second order transport properties}
%A peculiar feature of relativistic fluid dynamics is that it is not 
%  always causal. 
%More specifically, this problem arises when one goes beyond the ideal 
%  fluid approximation and includes dissipative transport properties 
%  such as shear viscosity, bulk viscosity and heat conductivity or 
%  baryon diffusion. 
%Keeping for the shear stress, bulk viscous pressure and baryon diffusion 
%  current only terms of first order in gradients of the fluid velocity,
%  temperature and chemical potential leads to a covariant version of the 
%  well known Navier-Stokes theory,  which however,  is not an hyperbolic 
%  differential equation and can therefore not be used for a time evolution 
%  that is causal in the relativistic sense.
%A way out has been proposed by Müller,  as well as Israel and Stewart. 
%In their framework,  the theoretical setup is modified in such a way that 
%  the shear stress,  bulk viscous pressure and baryon diffusion current are 
%  not related to the gradients of fluid velocity and thermodynamic variables 
%  by constraints but rather have their own evolution equation and relax towards 
%  the Navier-Stokes values on a proper time scale given by their respective 
%  relaxation times. 
%These relaxation times cannot be too small in order to have a causal set of 
%  fluid dynamic evolution equations.
%It would in fact be great to test the modifications proposed by Müller,  
%  Israel and Stewart (and subsequent authors) experimentally and to put an 
%  experimental bound on the value of the relaxation times (or their ratios 
%  to other thermodynamic and transport properties). 
%This would help for a better understanding of relativistic fluid dynamics 
%  that is also needed elsewhere,  for example in cosmology. 
%However,  this is not very easy and can only be done in an interplay of 
%  theory and experiment. 
%In fluid dynamic models of heavy ion collisions one can vary the value of 
%  second order transport coefficients and study the impact on various flow 
%  observables,  in particular particle spectra,  flow coefficients $v_n$,  
%  flow correlation functions and HBT parameters. 

%By a detailed comparison to experimental data one can put constraints on 
%  the second order transport coefficients. 
%It might become possible to vary second order transport coefficients in 
%  global Bayesian analysis calculations and provide experimental constraints 
%in this way. 
%A large variety of experimental flow observables,  detailed and differential data, 
%  as well as small statistical uncertainties are obviously helpful for this endeavor.




\subsection{Initial conditions} \label{sec:initialstate}
As already alluded to in Section~\ref{sec:macro}, the intial state of 
  heavy ion collisions is the major source of uncertainty when it comes 
	to extracting transport properties of the produced medium. 
This has to do with the fact that calculations of the exact geometry and 
  its fluctuations involve nonperturbative physics. 
The available descriptions for the initial state thus range from very 
  simplistic models that assign deposited energy densities based on the 
	wounded nucleons or binary collisions determined in a Monte-Carlo 
	Glauber prescription, to classical effective theories of QCD that are 
	valid in the high energy limit. 
The latter are certainly closer to first principles calculations, and 
  should provide a realistic description of the initial state assuming 
	that the high energy limit is a good approximation.
%%
The major ingredient that needs to be provided by an initial state model 
  is the geometry of the interaction region in the plane transverse 
	to the beam. 
It is entirely dominated by the positions of incoming nucleons whose 
  fluctuations also play an important role. 
Observables are also sensitive to the detailed way the energy is deposited 
  given a distribution of wounded nucleons, and the constraints on this 
	from data are surprisingly robust~\cite{Moreland:2018gsh}. 
Using the Trento model it was found that the product of thickness functions 
  of the two nuclei at every given transverse position gives the best description 
	of the data (compared to a conventinal wounded nucleon model or other 
	combinations of thickness functions). 
This type of energy deposition is very much what the EKRT~\cite{Niemi:2015qia} 
  and IP-Glasma~\cite{Schenke:2012wb,Schenke:2012fw} models include, explaining 
	their success in describing a wide range of observables~\cite{Niemi:2015qia,Gale:2012rq}.
%%
Fluctuations of wounded nucleon positions contribute the dominant effect to 
  fluctuations of the initial geometry in heavy ion collisions. 
Smaller scale fluctuations (emerging e.g. from color charge fluctuations 
  in the IP-Glasma model) have been shown to not significantly affect most 
	observables measuring the momentum anoisotropy of produced particles~\cite{Gardim:2017ruc}. 
The factorization breaking ratio shows some sensitivity and it should be pointed 
  out that multiplicity fluctuations are also influenced by the existence of 
	color charge fluctuations.
%%
More recently, indications that the nucleon should possess a substructure of hotspots 
  have emerged. 
The IP-Glasma model was unable to describe flow coefficients in p+Pb 
  collisions assuming a round proton~\cite{Schenke:2014zha}. 
Including subnucleonic fluctuations constrained by incoherent 
  diffractive $J/\psi$ production data from HERA~\cite{Mantysaari:2016ykx} 
	(which also require a fluctuating proton geometry) improved the agreement 
	with the p+Pb data significantly~\cite{Mantysaari:2017cni}. 
Recent Bayesian analyses also found that nucleon substructure is necessary 
  to simultaneously describe p+Pb and Pb+Pb bulk observables~\cite{Moreland:2018gsh}.
It should be noted, however, that the size scales for nucleons and subnuclear 
  hot spots extracted in that work are significantly larger than those obtained
	from comparison to HERA data in the IP-Glasma model.

Initial conditions for hydrodynamic simulations have to provide, in principle, 
  all components of the energy momentum tensor as a function of spatial position 
	(and initial conditions for other conserved charges, if considered). 
This includes, apart from the always included energy density distribution, 
  the initial flow as well as initial viscous corrections. 
Initial flow is included in many recently developed models, that either assume 
  free streaming~\cite{Moreland:2018gsh}, including Yang-Mills evolution, which 
	is close to free streaming~\cite{Gale:2012rq} or an initial flow distribution 
	motivated by strong coupling calculations~\cite{Weller:2017tsr}. 
Initial viscous corrections are often set to zero. 
Only a few works~\cite{Mantysaari:2017cni,Schenke:2018fci,Moreland:2018gsh} 
  include the full viscous stress tensor provided by the initial state description.

Since the initial state models that provide the entire $T^{\mu\nu}$ all switch 
  from essentially a freely streming system to strongly interacting hydrodynamics 
	at a fixed time $\tau$, that transition is somewhat abrupt and not very physical. 
To improve over this situation an intermediate step using effective kinetic theory 
  has been introduced~\cite{Kurkela:2018wud,Kurkela:2018vqr}. 
This procedure allows for a somewhat smoother matching but has yet to be used in 
  full fledged hydrodynamic simulations. 
The effect on observables in heavy ion collisions is likely small, while in small 
  systems, such as p+A a larger effect can be expected.

As already discussed in Section~\ref{sec:macro}, the choice of initial state has 
  a significant effect on the extraction of transport coefficients. 
A more compact initial state and the presence of initial flow lead to a larger 
  transverse flow, which requires a larger bulk viscosity to compensate for it 
	and produce agreement with experimental data~\cite{Schenke:2018fci}. 
Also, the models' eccentricities will affect the final momentum anisotropies, 
  influencing the extracted shear viscosity to entropy density ratio. 
Two possible attempts to solve this problem have been pursued: Constrain an 
  initial state description as well as possible using data from experiments 
	other than heavy ion collisions (e.g. e+p scattering data, which will hopefully 
	be extended to e+A in a future electron ion collider facility), or perform a 
	combined Bayesian analysis of all parameters, including those of the initial state, 
	to find the best fit for all transport coefficients along with the initial 
	state description.
As mentioned above, at the moment the two approaches lead to some similar 
  features of the initial state (product of thickness functions, presence 
	of subnucleon structure), but also show discrepancies (size of the nucleon 
	and sub-nucleon scales along with the size of the extracted bulk viscosity). 
Only when the two methods converge for all features of the initial state and 
  medium properties, can one confidently declare that the initial state as well 
	as the transport properties of the QGP are understood.









\subsection{Response functions}
In a fluid dynamic description of heavy ion collisions,  one can understand 
  the azimuthal harmonic flow coefficients $v_n$ as a response to deviations 
  of the initial state from an isotropic form. 
Mathematically,  one can formulate this in terms of response functions that 
  describe how the solution of the fluid dynamic evolution equations,  
  as well as resulting experimental observables such as azimuthal particle 
  distributions,  get modified when the initial values of the fluid fields 
  are changed. 
In cosmology,  closely related transfer functions are used to describe the 
  response to initial density perturbations created by quantum fluctuations 
  during inflation. Similar as in cosmology,  one can formulate linear as 
  well as non-linear response functions for various fluid dynamic fields 
  and observables. 
Moreover,  statistical symmetries of the initial state,  in particular 
  azimuthal rotation invariance and the approximate Bjorken boost symmetry 
  can be used to classify different perturbations. 
In the simplest implementation,  linear response functions describe the 
  linear response of flow coefficients to eccentricities $v_n \sim\epsilon_n$,  
  while the quadratic part describes terms like $v_n \sim \epsilon_a\epsilon_b$ 
  where symmetry reasons imply $|n|=|a\pm b|$. Response functions can not only be 
  used to study deviations from azimuthal rotation symmetry but also for deviations 
  from (approximate) Bjorken boost invariance,  vanishing baryon number density,  
  for electric fields and so on. 
Quite generally,  response functions cary interesting information about fluid 
  properties such as thermodynamic and transport properties. 
  Where the response functions are known,  one can infer properties of the 
  initial state by reverse engineering.
%%
Experimentally,  one can constrain properties of response functions indirectly 
  via measurements of various particle correlation functions. 
Differential information as well as good statistics are valuable here. 
  It is particularly interesting to compare situations with strong deviations 
  from a symmetry (such as peripheral collisions for the case of azimuthal 
  rotation invariance) to situations with small deviations (such as central collisions). 
This should allow to distinguish linear from quadratic and higher order response functions.
%%
One may in principle also use classifications of events according to their net 
  baryon number  in the central rapidity range or other event selection criteria. 
This idea is also known as event engineering and might become even more feasible 
  with more statistics.
%%
A general problem in the context of the fluid dynamic description of heavy 
  ion collisions is that the initial state is not very well known. 
For the time being,  one must improve the knowledge of the initial state 
  as well as the fluid properties of the quark-gluon plasma (which in turn 
	determine the response functions) simultaneously by detailed comparison 
	between experimental results and theoretical calculations. 
Dependence on external parameters like system size and collision energy as 
  well as differential information such as on centrality, or particle 
	identification will be helpful to make further progress here. 
In addition to this,  theoretical techniques and models must be improved further.



