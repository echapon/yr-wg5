\subsection{Summary}
The measurements of inclusive hadron \vn\ by traditional methods such as 
  two-particle correlations, event-plane/scalar-product methods, 
  multi-particle cumulants etc. have been performed with high precision 
  by the ALICE, ATLAS and CMS experiments at the LHC.
These inclusive hadron \vn\ measurements are not statistically limited 
  across most of the centrality-\pt\ phase space and further improvement 
  in the measurements is not a high priority for Run~3 and 4.
However, in the case of identified hadrons the increased statistics 
  will lead to further improvement in the \vn\ measurements.
This is true for both light hadrons such as pions, protons, $\phi$-mesons 
  as shown in Figure~\ref{fig:alice_vn}, as well as for heavy-flavor 
  particles such as $D^0$, $D^{\pm}$, J/$\psi$, $\Upsilon$ which are 
  discussed in Chapters~\ref{sec:HI_HF} and \ref{sec:quarkonia}, respectively.
%
Significant improvements are expected in measurements of 
  longitudinal flow fluctuations, which have only been briefly investigated
  in Run~1 and 2.
These are largely driven by the increases $\eta$ acceptance of the
  ATLAS and CMS tracking detectors in Run~4, the acceptance is planned 
  to reach $\pm$5 units.
The study of longitudinal flow fluctuations will allow comparisons to predictions 
  of 3+1D hydro models.
%
Flow measurements in light ions such as Ar--Ar and O--O, will lead 
  to stronger constraints on theoretical models describing different 
  stages of a heavy ion collision 
  -- initial conditions, equation of state, transport coefficients etc.
This is difficult presently, as flow observables are simultaneously dependent 
  on all of these, so it becomes difficult to constrain one any one of these 
  without full knowledge of the others.
Flow measurements across a variety of colliding species will provide independent 
  data points that can help in improvement of our understanding of these different 
  stages of heavy ion collisions.
Further physics motivations for colliding light are discussed in Chapter~\ref{sec:smallAsum}.

Other observables related to collective phenomena where current 
  measurements are statistics limited and are expected to improve 
  considerably are related to effects of vorticity and magnetic fields.
%
The current measurements of $\Lambda$ polarization from ALICE are statistics limited
  and consistent with both the null hypothesis as well as with the theoretically
  predicated value.
The ALICE projections for $\Lambda$ polarization in Run~3 and 4 show that 
  the measurements will have significantly smaller statistical uncertainties 
  and will easily be able to differentiate between the null and predicted values. 
%
ALICE and CMS have measured the fraction of the three-particle correlator
  $\gamma_{\alpha\beta}$ that arises from CME effects: $f_{\mathrm{CME}}$.
The measured $f_{\mathrm{CME}}$ by ALICE is consistent with zero but due to 
  large uncertainties its upper limit at 95\% CL can be as large as $\sim$0.5.
ALICE projections for Run~3 and 4 show that the $f_{\mathrm{CME}}$ can 
  be determined with a precision of less than 1\%.


















