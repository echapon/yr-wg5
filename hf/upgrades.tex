The upgrades of the four large LHC experiments during LS2 and LS3 will strongly enhance their performance for open heavy-flavour measurements. The detector improvements that will have the largest impact are the new silicon trackers, with higher granularity and precision, as well as extended pseudo-rapidity coverage. Brief descriptions of these improvements are reported in the following.

\begin{itemize}
\item ALICE. The new Inner Tracking System~\cite{Abelev:1625842}, which will be installed during LS2, is composed of seven layers of pixel detectors with a spatial precision of about $5\times 5~\mu$m$^2$ and a material thickness of 0.3\% of the radiation length in the innermost layers. The track pointing resolution will be improved by a factor 3 in the direction transverse to the beam line and by a factor 5 in the longitudinal direction, down to values of about $20~\mu$m for tracks with $\pT=1$~GeV$/c$. A Muon Forward Tracker~\cite{CERN-LHCC-2015-001}, composed of 5 disks of pixel detectors with the same spatial resolution as the Inner Tracking System, will instrument the region $2.5<\eta<3.6$, in front of the muon spectrometer, enabling the separation from the primary vertex of single and dimuons from $D$, $B$ and $J/\psi$ decays. The upgraded TPC with GEM-based readout chambers, together with readout upgrades of several other detectors and with a new Online-Offline computing system, will enable the full recording of Pb-Pb interactions with a minimum-bias trigger at a rate of 50~kHz, which is 50-fold larger than for the present apparatus.    
\item ATLAS.  The Inner Tracker (ITk)~\cite{ATL-PHYS-PUB-2016-025} will be an all-silicon tracker composed of pixels and strips installed during LS3 for ATLAS phase II.  The ITk will provide charged-particle tracking acceptance for $|\eta|<4$.  The performance of the ITk in Pb-Pb collisions is expected to be comparable to pp collisions.  The High Granularity Timing Detector~\cite{Collaboration:2623663} has been proposed to complement the spatial information of the ITk with timing information.  These detectors will improve jet reconstruction capabilities, and in particular tagging of heavy-flavour jets, as well as all studies using charged particles.
\item CMS. The following upgrades scheduled for LS3 will largely enhance the performance for heavy-flavour measurements, in particular in the low-momentum region. The upgraded inner tracker in LS3 will cover a large acceptance up to $|\eta|<4$~\cite{Contardo:2020886}. The improved L1 and DAQ rate (up to 60~GB/$s$) will allow more sophisticated triggers and to record a larger number of minimum-bias triggered events. In addition, the proposed MIP Timing Detector~\cite{Collaboration:2272264} with a radius of 1.16~m and a time resolution of $\approx 30$~ps. Together with other detectors, it could provide proton, pion and kaon separation in the interval $0.7<\pT < 2$~GeV$/c$ in 
$|\eta|<1.5$. 
%Those significant detector upgrades and trigger performance could be used to significantly improve the heavy flavor meson measurements, flow fluctuation studies in smaller systems, and jet spectra and substructure modifications in heavy ion collisions.
\item  LHCb. The experiment is preparing to run at five times larger instantaneous luminosities in pp collisions, processing the full event rate with a software trigger and preserving or exceeding the present performance.
%~\cite{Collaboration:1624070,Collaboration:1624074,Collaboration:1647400,CERN-LHCC-2014-016,TheLHCbCollaboration:2310827,LHCbCollaboration:2319756}. 
All tracking detectors will be upgraded~\cite{Collaboration:1624070,Collaboration:1647400}. 
Most notably for heavy-flavour observables, the active area of the upgraded Vertex Locator, the pixel detector replacing the present silicon strip detector, will move as close as 5.1~mm to the nominal beam spot.
The larger granularity for the majority of phase space will improve the performance in Pb-Pb collisions whereas proton-induced reactions will result in average in lower detector occupancies than the standard pp running.  
\end{itemize}

The physics performance studies presented in this chapter and labelled as {\it Simulation} were obtained with full simulations of the new detectors (detailed geometry, particle transport, event reconstruction), while those labelled as {\it Projection}
were obtained by scaling the uncertainties of existing measurements to the integrated luminosities expected for future data samples. The latter do not take into account the improved response of the new detectors, therefore they should be considered as conservative estimates of the future performance.