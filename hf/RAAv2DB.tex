%%\begin{itemize}
%%\item {Experiments:} state of the art (open questions?) and estimates for low and high pt.
% \begin{itemize}
%\item {Observables:} RAA, v2, v3 for D, B mesons (when possible for both), event-shape engineering v2 analyses
%\item {Complementarity:} focus on complementarity (pt, y), provide plot of pt vs y for all experiments for a couple of observables.
%\end{itemize}
%%\item {Connection to theory:}  constrain c and b diffusion coefficients. Additional insights from v2(D) vs v2(pi) on event-by-event
%%\end{itemize}

%%\textbf{FIGURES}:
%%\begin{enumerate}
%%\item D RAA (ALICE+CMS)
%%\item B, non-prompt D, non prompt J/psi RAA (ALICE+CMS) 
%%\item D, B, non-prompt D, non prompt J/psi  v2 (ALICE+CMS)
%%\item summary figure to show complementarity: pt vs y for all experiments for a couple of observables
%%\item EsE D mesons (ALICE) + theory?
%%\item Theory: Ds ranges (Bayesian approach)
%%\item Theory: Ds ranges (using D and using D+B, Catania)
%%\end{enumerate}

\subsubsection{Nuclear modification factor and collective flow of heavy-flavour hadrons}
\label{sec:HFRAAv2}
The main observable used to study the medium effect on heavy-flavour meson production is nuclear modification factor ($\RAA$), defined as the ratio of the \PbPb yield to the pp cross-section scaled by the nuclear overlap function. In the view of the pQCD-based models, heavy quarks lose energy via radiative and collisional interactions with the medium constituents. The so-called ``dead-cone effect"~\cite{DOKSHITZER2001199} is expected to reduce small-angle gluon radiation of heavy quarks when compared to both gluons and light quarks. At low $\pt$, the production rate of heavy-flavour mesons in heavy-ion collisions is sensitive to the elastic energy loss of the heavy quark in medium, the nuclear shadowing effect in the initial state, and the recombination of the heavy quark with light quark at the hadronization stage. At high $\pt$, the nuclear modification factor is sensitive to the medium-induced radiative energy loss of heavy quarks. Precise measurement of the $\RAA$ thus provides insights about the momentum dependence of heavy quark energy loss, and provides important tests of QCD predictions in particular test the expected flavour and mass dependence of the energy loss processes.

Another interesting observable is the azimuthal anisotropy of open heavy flavour production, which can be characterized by the Fourier coefficients $\vn$ in the azimuthal angle ($\varphi$) distribution of the heavy-flavour hadron yield with respect to the reaction plane in non-central \PbPb collisions. At low $\pt$, the $\vtwo$ (elliptic flow) heavy-flavour hadron $\vtwo$ can help quantify the interaction strength of heavy quarks with the medium. $\vtwo$ can also be used to study the recombination process of heavy quarks with light quarks. At high $\pt$, $\vtwo$ of heavy-flavour hadrons is sensitive to the path-length dependence of heavy quark energy loss. The simultaneous description of $\RAA$ and $\vtwo$ for heavy-flavour hadrons is still challenging for most of the theoretical calculations, because it entails accurate modelling of the initial heavy-quark production and its modification in nuclei, of the medium and its expansion, of the various quark-medium interaction mechanisms and of the possible modification of hadronisation processes.

%Simultaneous prediction to $\RAA$ and $\vtwo$ imposes a strict requirement to theoretical models.

\subsubsection{Experimental performance of the CMS and ALICE experiments}

Figure~\ref{fig:RAAv2.RAA} shows the projected performance for the $\RAA$ of several heavy-flavour hadrons or decay channels with $\Lint=10~\invnb$. The left panel presents the projection of charged particles, $\PDzero$, $\PBp$ and non-prompt \PJGy from b-hadron decay which can be measured by CMS. The right panel shows the ALICE simulation results for $\PDzero$, non-prompt \PJGy, non-prompt $\PDzero$, $\PBp$ ($\rightarrow \PDzero \PGpp$). The $\PBp$ ($\rightarrow \PDzero\PGpp$) hadronic decay provides an alternative channel for the study of the beauty meson $\RAA$ with a significance of larger than 5$\sigma$ at $\pt$ > 3 GeV/c. The central values of $\RAA$ are taken from~\cite{raatheory}. With the high luminosity and the Inner Tracker System Upgrade in ALICE, the $\RAA$ of light hadrons, charm hadrons and beauty hadrons can be clearly separated in a wide kinematic range. Also, the precise measurement could provide a strong constraint on theoretical models.

\begin{figure}[ht]
  \begin{center}
    \includegraphics[width=0.45\textwidth]{hf/figures/cRAA_lumiTG_10_lumiMB_0_v2_right.pdf}
   \includegraphics[width=0.54\textwidth]{hf/figures/ALICEUpgrade_charmbeautyRAA.pdf}
  %  \includegraphics[width=0.32\textwidth]{hf/figures/2016-Jun-10-BMeson_wPID_Raa.pdf}
    \caption{Nuclear modification factors of charged particles, $\PDzero$, $\PBp$ and non-prompt \PJGy in CMS~\cite{CMS-PAS-FTR-17-002} (left). $\RAA$ of $\PDzero$, non-prompt \PJGy and non-prompt $\PDzero$ in ALICE in central \PbPb collisions for $\Lint=10~\invnb$~\cite{Abelev:1625842} (right).}
    \label{fig:RAAv2.RAA}
  \end{center}
\end{figure}


Figure~\ref{fig:RAAv2.v2charm} shows the projected performance for $\vtwo$ of charm hadrons with $\Lint=10~\invnb$. The left panel shows the projection for $\PDzero$ in CMS, with the charged particle $\vtwo$ also shown for comparison. The right panel presents the projection for $\PDzero$, $\PDs$ and \PGLc in ALICE~\cite{Abelev:1625842}. The central values are taken from~\cite{v2charmtheory}. Precise measurements of charm hadron $\vtwo$ will allow the study of the thermalization of heavy quarks and the wide kinematic range allows to get insights on different process, as coalescence hadronization and energy loss. Figure~\ref{fig:RAAv2.v2beauty} shows the projected performance for $\vtwo$ of beauty hadrons in ALICE. The left panel presents the projection for non-prompt $\PDzero$ and non-prompt \PJGy, while the right panel shows the projection for $\PBp$. These will be the first precise measurements of B meson elliptic flow at the LHC.

\begin{figure}[ht]
  \begin{center}
    \includegraphics[width=0.49\textwidth]{hf/figures/cV2_lumiMB_0_wTheory_right.pdf}
   \includegraphics[width=0.49\textwidth]{hf/figures/ALICEUpgrade_charmv2.pdf}
    \caption{$\vtwo$ of charged particles and $\PDzero$ in CMS~\cite{CMS-PAS-FTR-17-002} (left), charm hadrons ($\PDzero$, $\PDs$, \PGLc) in ALICE (right) in \PbPb collisions with $\Lint=10~\invnb$~\cite{Abelev:1625842}.}
    \label{fig:RAAv2.v2charm}
  \end{center}
\end{figure}
\begin{figure}[ht]
  \begin{center}
    \includegraphics[width=0.49\textwidth]{hf/figures/ALICEUpgrade_beautyv2.pdf}
   % \includegraphics[width=0.49\textwidth]{hf/figures/2016-Jun-10-PlotAllResults_v2FinalEstimate_20-40.pdf}
    \caption{$\vtwo$ of non-prompt $\PDzero$, non-prompt \PJGy and $\PBp$ ($\rightarrow \PDzero$) in ALICE in \PbPb collisions for $\Lint=10~\invnb$~\cite{Abelev:1625842}.}
    \label{fig:RAAv2.v2beauty}
  \end{center}
\end{figure}

\subsubsection{Constraining the heavy-quark diffusion coefficient $\twopiTDsc$}
Over the past few years, many theoretical efforts have been taken to understand the properties of the QGP medium and the interaction between heavy flavours and the medium. Due to its short life time one has to reply on the comparison between theoretical calculation and the experimental measurements to probe the mechanisms and strength of the interactions. 
Although the interaction mechanism can widely vary among different theoretical models, the reduction to a few transport coefficients allow one to compare these models and evaluate different microscopic pictures. Theoretically the general consensus implies a dominant collisional energy loss mechanism in the lower $\pt$ region and dominant radiative energy loss in the higher $\pt$ region, how and where the transition happen become of significant interest.


So far on one hand, simultaneously describing heavy flavor  $\RAA$ and $\vn$ is still a challenge for theoretical calculations. On the other hand, for many of the models that can reasonably describe the single charm hadron $\RAA$ and $\vtwo$ data, the diffusion coefficient ($\Dsc$) varies significantly and is to be determined. The future data will be able to constrain the heavy quark diffusion coefficient in the QGP medium to a more precise level, potentially distinguish different mechanisms and greatly improve our understanding to nature of the interactions.


\begin{figure}[ht]
	\begin{center}
		\includegraphics[width=0.48\textwidth]{hf/figures/chi_Raa_Ds.pdf}
		\includegraphics[width=0.48\textwidth]{hf/figures/Plot_posterior_D2piT_calibrate_on_Duke-central_p0.pdf}
		\caption{Normalized $\chisquared$ as a function of spatial diffusion coefficient ($\twopiTDsc$) under different experimental precision from an Fokker-Plank transport model(left). Coefficient range (90\% credibility region) in the phase plane of $\twopiTDsc$ vs $\ToverTc$ under different experimental precision, estimated by a model-to-data Bayesian analysis using an improved Langevin framework (right).}
		\label{fig:RAAv2.Dstheory}
	\end{center}
\end{figure}

Figure~\ref{fig:RAAv2.Dstheory} shows the constraint power of experimental measurements of  $\RAA$ and $\vn$ on the heavy quark diffusion coefficient ($\twopiTDsc$) from two different transport models (Fokker-Plank on the left and an improved Langevin framework on the right~\cite{PhysRevC.97.014907}). The left figure presents a normalized $\chisquared$ as a function of spatial diffusion coefficient constant by comparing the model's calculation on $D$-meson $\RAA$ at PbPb 5.02 TeV in a single centrality (7 observation in total). With $\chisquared_{\RAA}/n.d.f$ around 1.5 (corresponding to 85\% confidence), one could see that by reducing the current experimental uncertainty to 50\%, the uncertainty for the estimation $\twopiTDsc$ would be reduced almost to 50$\%$. The right panel presents the diffusion coefficient as a function of temperature, which is estimated by a state-of-the-art Bayesian calibration on D-meson $\RAA$ and $\vtwo$ in \PbPb 5.02 TeV for different centralities. The $\twopiTDsc$ shows a positive temperature dependence with the minimum value around $\Tc$. Such behavior is consistent with the Bayesian estimation for shear viscosity $\eta/s$, which also implies a strongest interaction around critical temperature $\Tc$. While the left panels demonstrates the $\twopiTDsc$ estimation by reducing the current experimental uncertainties to 50\% (the blue triangle curve) and 20\% (the black squared curve), the right panel presents the potential improvement with a more realistic projection on future experimental uncertainties (the red shadowed area). With the future experimental measurements, one could further constrain the diffusion coefficient, especially its value around $\Tc$ to a potential significance of 50\% of current credibility region. Such estimation would help us to better distinguish between different interaction mechanisms and move towards the precision era. Figure~\ref{fig:RAAv2.Dstheory} shows the constrain power of experimental result of $\RAA$ and $\vn$ on the diffusion coefficient under different experimental precision. The left panel is normalized $\chisquared$ as a function of spatial diffusion coefficient ($\twopiTDsc$) from AdS/CFT. The right panel shows the coefficient range in the phase plane of $\twopiTDsc$ vs. $\Tc$ with Linearized Boltzmann Transport (LBT) model~\cite{PhysRevC.94.014909}.


% \subsubsection{Constraining the heavy-quark diffusion coefficient $2\pi TD_s$}

% Over the past few years, many theoretical efforts have been taken to understand the transport properties of heavy flavours in the medium. Although the mechanisms can widely vary among different theoretical models, transport coefficients can be used to compare these models. Despite successful estimation of some transport properties like the shear viscosity, the most basic energy loss mechanism are not yet understood and the diffusion coefficient ($\PDs$) is to be determined. So far simultaneously describing $\RAA$ and $v_{\mathrm{n}}$ is still a challenge for many models, the future precise data will be able to constrain the heavy quark diffusion coefficient in QGP, potentially distinguish different models and greatly improve our understanding to the interaction mechanism between heavy quarks and the medium.

% \begin{figure}[ht]
%   \begin{center}
%     \includegraphics[width=0.53\textwidth]{hf/figures/Greco.pdf}
%     \includegraphics[width=0.43\textwidth]{hf/figures/Plot_D2piT_posterior_p0.png}
%     \caption{Normalized $\chi^{2}$ as a function of spatial diffusion coefficient ($2\pi TD_{s}$) under different experimental precision from AdS/CFT (left). Coefficient range in the phase plane of $2\pi TD_{s}$ vs. $T_{c}$ under different experimental precision with LBT model (right).}
%     \label{fig:RAAv2.Dstheory}
%   \end{center}
% \end{figure}

 
\subsubsection{Event Shape Engineering}

Further insight into the dynamic of heavy quarks in the medium can be
obtained from measurements of the yield and elliptic flow of heavy-flavour
particles with the Event Shape Engineering (ESE) 
technique~\cite{Schukraft:2012ah}.
This technique consists of selecting events with the same centrality but 
different magnitude of the average bulk elliptic flow and therefore initial-state geometry eccentricity.
The analyses with ESE will allow us to investigate the correlation between 
the flow coefficients of heavy-flavour hadrons and soft hadrons, to study the interplay between elliptic and radial flow, and to further constrain the path-length dependence of the energy loss suffered by the heavy quarks in the QGP.
In the left-hand panel of Fig.~\ref{fig:ESE}, the prospects for the measurement of the ${\PDzero}$-meson $\vtwo$ with the ESE technique in the 30--50\% centrality class are shown.
In particular, the ratio between the $\vtwo$ of D mesons in the 10\% of the events with larger (smaller) elliptic flow of the bulk (quantified through the magnitude of the so-called reduced flow vector $q_2$) and the $\vtwo$ in all the collisions in the considered centrality class is reported as a function of $\pt$.
It is compared to the current measurement of the same ratio for charged 
particles, which is dominated by light-flavour hadrons.
The expected statistical uncertainties for the $\PDzero$-meson $\vtwo$ in the 0-10\% of events with larger (smaller) $q_2$ are of the order of about 1-2\% in the interval $1 < \pt < 8~\UGeVc$. This will allow us to resolve a possible difference of a few percent in the response of the $\vtwo$ to the ESE selection between the $\PDzero$ mesons and the light hadrons, providing new insight on the coupling of the charm quark with the medium constituents and on its degree of thermalisation. 
The performance for the measurements of $\PDzero$-meson 
$\pt$-differential yield in event-shape classes is displayed in the right-hand panel of Fig.~\ref{fig:ESE}.
The uncertainties expected with the Run3 data sample will guarantee a sensitivity of a few percent for the modification of 
the D-meson $\pt$ spectra in events with small (large) initial 
geometrical anisotropy.
This will open the way for precise studies on the interplay between the initial geometrical anisotropy (the collective flow of the bulk) and the heavy-flavour radial flow and energy loss.

\begin{figure}[ht]
  \begin{center}
    \includegraphics[width=0.7\textwidth]{hf/figures/ALICE_D0ESE_3050_upgradeprojection.pdf}
 
    \caption{Left: projection of the expected ratio of $\PDzero$-meson $\vtwo$ in the 10\% events with larger (smaller) $q_2$ with respect to the unbiased one as a function of $\pt$ for the 30--50\% centrality class. The modification of the $\PDzero$-meson $\vtwo$ was assumed to be equal to that measured for the charged particles in Pb--Pb collisions at $\sqrtsNN=2.76~{\UTeV}$ in the the 30--40\% centrality class (superimposed for comparison)~\cite{Adam:2015eta}. Right: projection of the expected ratio of $\PDzero$-meson $\pt$-differential yield in the 10\% events with larger $q_2$ with respect to the unbiased one, estimated considering the prediction provided by the POWLANG model~\cite{Beraudo:2018bxb}.}
    \label{fig:ESE}
  \end{center}
\end{figure}




