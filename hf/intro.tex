Being produced in initial hard scatterings, heavy quarks carry information of  the evolving hot and dense QCD medium, so called ``Quark-Gluon Plasma" (QGP), created in high-energy heavy-ion collisions, interacting with its constituents and losing energy via subsequent elastic scatterings and/or gluon radiation. 

Many experimental observations from RHIC and LHC showed evidence that charm and beauty quarks interact strongly with the QGP and that beauty quarks lose less energy compared to charm quarks~\cite{Adam:2015nna,Khachatryan:2016ypw}. 
While data are becoming more and more precise to start imposing constraints on models describing the microscopic interactions of heavy quarks with the medium, there are still unresolved questions:  What is the relative importance of radiative and collisional energy loss, and their parton pass dependence? 
Do heavy quarks thermalise in the medium, and how do they hadronise? 

\textbf{First version}

The Run 3 and 4 of the LHC will open a new precision era for heavy-flavour measurements in heavy-ion collisions. 
With the upgrades of the machine and of the tracking detectors, the higher accumulated statistics and higher precision will make it possible to quantify the properties of the Quark Gluon Plasma with heavy-flavour probes. Measurements of the nuclear modification factor and flow harmonics for particles with charm and beauty in the large kinematic range covered by combining the different LHC experiments will put the strongest constraints on the transport properties of the QGP, clarifying the microscopic mechanisms governing the interactions of charm and beauty quarks with the medium.
Furthermore, precise measurements of both strange and non-strange particles with charm and beauty, as well as mesons and baryons, will be crucial to quantify the degree of thermalisation of heavy quarks and the contribution of recombination with lighter quarks to their hadronization. This high-precision era will be accompanied by new and more differential measurements which will become accessible in Run 3 and 4. In particular, heavy-flavour correlations and jet observables, which are currently suffering from statistics limitations, will offer additional dimensions to understand parton mass effects in parton showers, redistribution of the lost energy, and the role of collisional and radiative energy loss.

\textbf{Second version}

The Run 3 and 4 of the LHC will open a new precision era for heavy-flavour measurements in heavy-ion collisions that will answer the above questions. 
With the upgrades of the machine and of the tracking detectors, the higher accumulated statistics and higher precision will make it possible to quantify the properties of the Quark Gluon Plasma with heavy-flavour probes. This high-precision era will also be accompanied by new and more differential measurements which will become accessible. 
The key measurements which are expected to have a strong impact on the characterization of the QGP with heavy-flavour observables are discussed in this chapter and summarized below:
\begin{itemize}
\item Nuclear modification factor and flow harmonics: these measurements for particles with charm and beauty in the large kinematic range covered by combining the different LHC experiments will put the strongest constraints on the transport coefficients (Ds, qhat) of the QGP, clarifying the microscopic mechanisms governing the interactions of heavy quarks with the medium.
\item Strange D and B mesons, charm and beauty baryons: currently limited by statistics, they will be essential to quantify the degree of thermalisation of heavy quarks and the contribution of recombination with lighter quarks to the hadronization process.
\item Heavy-flavour correlations and jet observables: also currently suffering from large statistics limitations, they will provide new insights in the parton mass effects in parton showers, in the redistribution of the lost energy, and in the role of collisional and radiative energy loss.

\end{itemize}


