Charm and beauty quarks are produced in hard scattering processes occurring in the early stage of heavy-ion collisions. They subsequently traverse the QGP medium and interact with its constituents through inelastic (gluon radiation) and elastic (or collisional) processes.
These interactions may lead to the thermalisation of low-momentum heavy quarks, which would thus take part in the expansion and hadronisation of the medium.
For these reasons, heavy-flavour hadrons provide information on all stages of the system evolution and they uniquely probe the quark-mass dependence of the QGP inner workings (see Refs.~\cite{Andronic:2015wma,Prino:2016cni,Rapp:2018qla} for recent reviews).

Many experimental observations from RHIC and LHC showed evidence that charm and beauty quarks interact strongly with the QGP and that beauty quarks lose less energy at low transverse momentum compared to charm quarks~\cite{Adam:2015nna,Khachatryan:2016ypw}. 
While data are becoming more and more precise to start imposing constraints on models describing the microscopic interactions of heavy quarks with the medium, there are still unresolved questions:  What is the relative importance of radiative and collisional energy loss, and their parton mass dependence? What is the degree of thermalisation of heavy quarks? Are they partly thermalised or fully? How do they hadronise? 

The Run 3 and 4 of the LHC will open a new precision era for heavy-flavour measurements in heavy-ion collisions that will address the above questions. 
With the upgrades of the machine and of the tracking detectors, the higher accumulated statistics and higher precision will make it possible to quantify the properties of the QGP with heavy-flavour probes. This high-precision era will also make new and more differential observables accessible for the first time.
The key measurements that are expected to have a strong impact on the characterisation of the QGP with heavy-flavour observables are discussed in this chapter and summarized below.
\begin{itemize}
\item Nuclear modification factor and flow harmonics: these measurements for particles with charm and beauty in the large kinematic range covered by combining the different LHC experiments will put the strongest constraints on the transport coefficients of the QGP, clarifying the microscopic mechanisms governing the interactions of heavy quarks with the medium, and quantifying their degree of thermalisation.
\item Strange D and B mesons, charm and beauty baryons: currently limited by statistics, these measurements will help to quantify not only the degree of thermalisation of heavy quarks, but also the contribution of recombination with lighter quarks to the hadronisation process. They are also sensitive to the mass scaling of the hyrodynamical flow in the heavy-flavour sector.
\item Heavy-flavour correlations and jet observables: they will provide new insights on the parton mass effects in parton showers, on the redistribution of the radiated energy, and on the role of collisional and radiative energy loss.

\end{itemize}


