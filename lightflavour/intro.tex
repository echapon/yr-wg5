\subsection{Introduction}

The analysis of the LHC Run 1 and 2 data has established a standard model for the production of light flavour hadrons (containing u,d and s quarks) in heavy-ion collisions: (1.) integrated particle yields (particle chemistry) are well described by thermal-statistical models and (2.) the spectral shapes (kinetic equilibrium) are well described by a common radial expansion governed by hydrodynamics.
While the physics of light flavours is often mistaken as not statistics hungry, the unprecedented large integrated luminosities of the LHC Run 3 and 4 offer a unique physics potential. In fact, light anti- and hyper-nuclei are composed of u, d, and s-quarks, but given their composite nature are only very rarely produced, thus enormously profiting from the significant increase in statistics expected in the years 2020 until 2028. The same holds true for the study of event-by-event fluctuations in particle production.
As a matter of fact, the production of light anti- and hyper-nuclei and the study of event-by-event fluctuations in particle production are closely linked. The chemical freeze-out temperature at which event-by-event fluctuations of conserved quantities are compared to Lattice QCD is most precisely determined by light anti- and hyper-nuclei data, which are not subject to feed-down corrections from strong decay.

This chapter is organised as follows. 
The first part is dedicated to the physics of light (anti-)(hyper-)nuclei and exotic multi-quark states and the observables that will become accessible to measurement in the Run 3 and 4. The hyperon-nucleon potential is not only constrained by femtoscopic measurements, but also by a precise measurement of the lifetimes of (anti-)hyper-nuclei.
Vice-versa, the structure of the wave-functions, which determines the coalescence probability of anti-hyper-nuclei depends on the hyperon-nucleon potential.
The measurements presented here are of high relevance also in astrophysical context: the hyperon-nucleon potential is a crucial input to the equation of state of neutron stars while the yields of anti-nuclei provide the baseline for the searches for anti-matter in space. 
The LHC, used as ``anti-matter factory'', opens for unprecedented possibilities to study properties of anti-matter. Many of the measurements proposed will remain a reference for decades with the LHC being the only active hadron collider for many years to come.
\textcolor{red}{to be completed}
