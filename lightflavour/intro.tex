\subsection{Introduction}
The analysis of the LHC Run 1 and 2 data has consolidated a standard model for the production of light flavour hadrons (containing u,d and s quarks) in heavy-ion collisions:
integrated particle yields (particle chemistry) are well described by the thermal-statistical model and  
the spectral shapes (kinetic equilibrium) are well described by a common radial expansion governed by hydrodynamics.
While the physics of light flavours is often perceived as not statistics hungry, the unprecedented large integrated luminosities of the LHC Run 3 and 4 offer a unique physics potential. Light anti- and hyper-nuclei are very rarely produced despite containing only u, d and s quarks, because of their composite nature and very large mass, thus enormously profiting from the significant increase in statistics expected in the years 2021 until 2029. 
The same holds true for the study of event-by-event fluctuations of the produced particles.
As a matter of fact, the production of light anti- and hyper-nuclei and the study of event-by-event fluctuations in particle production are closely linked. 
The chemical freeze-out temperature at which event-by-event fluctuations of conserved quantities are compared to Lattice QCD is most precisely determined by light anti- and hyper-nuclei data, which are not subject to feed-down corrections from strong decay.

The first part of the chapter is dedicated to the physics of light (anti-)(hyper-)nuclei and exotic multi-quark states and the related observables that will become accessible for measurements in the Run 3 and 4. 
The physics motivations are outlined as well as the implications of the findings at the LHC for astrophysics and searches for dark matter in space-based experiments.
In the second half, measurements of fluctuations of particle production and conserved charges are discussed as they give experimental access to fundamental properties of the system and allow for direct comparison with Lattice QCD calculations. 
The state of the art of the experimental and theoretical investigation is briefly presented, together with the experimental reach  with the luminosity collected in Run 3 and 4.

