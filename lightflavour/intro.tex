\subsection{Introduction}
The analysis of the data collected at the LHC during Run 1 and Run 2 has consolidated our understanding of a standard model for the production of light-flavour hadrons (containing u, d and s quarks) in heavy-ion collisions:
particle chemistry (described by integrated particle yields) is well described by the thermal-statistical model \cite{Andronic:2017, Acharya:2017bso} and kinetic equilibrium (reflected in the \pT-dependence of particle production) is well described by a common radial expansion governed by hydrodynamics \cite{Abelev:2013vea, Acharya:2018zuq}.
While the physics of light-flavour particles is often perceived as not statistics hungry, the unprecedented large integrated luminosities expected in Run 3 and Run 4 at the LHC offer a unique physics potential. 
Despite containing only u, d and s valence quarks, light \mbox{(anti-)(hyper-)nuclei} are very rarely produced because of their composite nature and very large mass. Their study will enormously profit from the significant increase in luminosity for heavy-ion collisions expected in the years 2021 until 2029.
The same holds true for the study of event-by-event fluctuations of the produced particles, which is closely linked to the production of light \mbox{(anti-)(hyper-)nuclei} in the scenario of a common chemical freeze-out determining light-flavour hadrons and (hyper-)nuclei abundances. 
If, as indicated by the recent experimental findings \cite{Acharya:2017bso}, the thermal-statistical approach is the correct model to describe (anti-)(hyper-)nuclei production, the chemical freeze-out temperature is most precisely determined by measurements of light \mbox{(anti-)(hyper-)nuclei} as they are not subject to feed-down corrections from strong decays \cite{Andronic:2017}. This is the same temperature at which event-by-event fluctuations of conserved quantities are compared to lattice QCD (lQCD).
The physics of light \mbox{(anti-)(hyper-)nuclei} and exotic multi-quark states together with the related observables that will become experimentally accessible in \PbPb collisions at the LHC Runs 3 and 4 are discussed in Sec.~\ref{sec:nuclei}. 
In Sec.~\ref{sec:fluctuations}, measurements of fluctuations of particle production and conserved charges are discussed as they give experimental access to fundamental properties of the QCD phase transition at $\mu_{\mathrm{B}}$ and allow for direct comparison with lQCD calculations. 
%The state of the art of the experimental and theoretical investigation is briefly presented, together with the experimental reach expected from the luminosity collected in heavy-ion collisions in Runs 3 and 4.
\\\noindent In small collision systems (\pp, \pPb), measurements of light-flavour hadrons provide fundamental input to the study of particle production mechanisms and collectivity across systems, as discussed in Ch.~\ref{chapter:smallsystems}. At the same time, the physics programme with \pp and \pPb collisions in Runs 3 and 4 will open the possibility for system-size dependent studies of (anti-)nuclei production and for precision measurements of the hyperon-nucleon potentials. 
The physics case for these measurements in small colliding systems is motivated in this chapter in Sec.~\ref{sec:astro}, as well as the implications of the findings at the LHC for astrophysics and searches for dark matter in space-based experiments.

%\textcolor{red}{The foreseen integrated luminosities for the heavy-ion and small systems campaigns are summarised in Ch.~\ref{sec:schedule}}.