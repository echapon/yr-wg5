% don't remove the folling lines, and edit the defintion of \main if needed
\documentclass[../report.tex]{subfiles}
\providecommand{\main}{..}
\IfEq{\jobname}{\currfilebase}{\AtEndDocument{\biblio}}{}
% until here

%file for shortcuts

\newcommand{\nch}{\ensuremath{N_{\mathrm {ch}}\xspace}}
\newcommand{\Ncoll}{\ensuremath{N_{\mathrm {coll}}}}
\newcommand{\Npart}{\ensuremath{N_{\mathrm {part}}}}
\newcommand{\dNdeta}{\mathrm{d}N_\mathrm{ch}/\mathrm{d}\eta}
\newcommand{\snn}         {\ensuremath{\sqrt{s_{\mathrm {NN}}}}}
\newcommand{\kT}          {\ensuremath{k_{\mathrm {T}}}}

\newcommand{\pp}          {pp}
\newcommand{\pPb}         {pPb}
\newcommand{\pA}          {pA}
\newcommand{\PbPb}        {PbPb}
\newcommand{\AuAu}        {AuAu}
\newcommand{\CuCu}        {CuCu}
\newcommand{\pAu}         {pAu}
\newcommand{\dAu}         {dAu}
\newcommand{\lsim}        {\,{\buildrel < \over {_\sim}}\,}
\newcommand{\gsim}        {\,{\buildrel > \over {_\sim}}\,}
\newcommand{\co}[1]       {\relax}
\newcommand{\nl}          {\newline}
\newcommand{\el}          {\\\hline\\[-0.4cm]}

\begin{document}

\section{Emergence of Hot and Dense QCD in Small Systems}

\subsection{Historic overview}
% Comments wrt initial LHC program
% Table (-> Constantin, Naghmeh)

\subsection{Open questions and their theoretical implications}

% Nature of collectivity / initial vs final state
% Are “collective effects” caused by final-state interactions in pp and pPb? → we should see jet quenching and medium-quark interactions
%   RAA, X+jet correlations
%   HF vn, Strangeness enhancement
% (Smooth?) transition from pp to pPb to PbPb
% Is there a turn on of at low Nch?
%   Subevents
%   Higher order cn (may be hopeless..)
% Make clear why pp and pPb are needed
% Integrate different modelling approaches (hydro, CGC, escape, …) 
% Shortcomings of current modelling in small systems (need to argue how run 3 and 4 can improve this)

\subsection{Key observables}

\subsubsection{Multiplicity distribution in pp}

\todo{NB. This section will most likely be too long, and has to be shortened for the yellow report}
For the performance estimates at high multiplicity in pp, a multiplicity-distribution extrapolation has been used which is based on existing ALICE (|eta| < 1.5) and ATLAS (|eta|< 2.5) data and the parameterisation with a negative binominal distribution which have been frequently used to characterize the multiplicity distribution\footnote{At LHC energies two NBDs are needed for a good fit to the full distribution, but one is sufficient for the tail of the distribution.}.

The data used is shown in Fig.~FIGURE (left panel: ALICE, right panel: ATLAS). The fit with a single negative binominal distribution of the tail of the distribution (XX\% of the cross-section) is also shown. The three parameters of this fit are shown as a function of center of mass energy in Fig.~FIGURE together with a power-law fit used to interpolate to 5.5 TeV and extrapolate to 14 TeV.

The resulting multiplicity distribution for 14 TeV is shown in Fig.~FIGURE for the ALICE and ATLAS case, and compared in Fig.~FIGURE scaled relative to the average multiplicity to allow a direct comparison. \todo{comment on comparison, and on general uncertainties of tail}. \todo{discuss comparison to pPb and PbPb}

\todo{add tables of multiplicity classes for pp, pPb and PbPb}.

% vn, cn
%   Higher orders
%   Subevent method
%   With PID + strangeness
%   HF + quarkonia
%   P(vn) should be within reach in pp even!
% Strangeness enhancement
% Energy loss
%   “RAA”
%   Hadrons (and jets)
%   Definition of RAA in pp to be discussed
%   X+jets (X=photon,W,Z,[hadron,jet])
%   Heavy Flavour
% Thermal radiation?

\subsection{Data-taking strategy}

% High multiplicity triggering, pile up, MB sample
% pp
%   ALICE: additional several month pp program
%   ATLAS/CMS: special runs (mu~1) or special conditions at end of fill
%   LHCb: either in nominal (mu~5) or special running
%   HM sample: 200 pb-1 pp (per experiment)
%   How much MB for low-multiplicity questions?
% p-Pb scheduled run
%   1000-2000 nb-1

\subsection{Summary}

\end{document}
