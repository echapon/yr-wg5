% don't remove the folling lines, and edit the defintion of \main if needed
\documentclass[../report.tex]{subfiles}
\providecommand{\main}{..}
\IfEq{\jobname}{\currfilebase}{\AtEndDocument{\biblio}}{}
% until here

\begin{document}

\section{Physics of Hot and Dense QCD in Small Systems}

\subsection{Historic overview}
% Comments wrt initial LHC program
% Table (-> Constantin, Naghmeh)

\subsection{Open questions and their theoretical implications}

% Nature of collectivity / initial vs final state
% Are “collective effects” caused by final-state interactions in pp and pPb? → we should see jet quenching and medium-quark interactions
%   RAA, X+jet correlations
%   HF vn, Strangeness enhancement
% (Smooth?) transition from pp to pPb to PbPb
% Is there a turn on of at low Nch?
%   Subevents
%   Higher order cn (may be hopeless..)
% Make clear why pp and pPb are needed
% Integrate different modelling approaches (hydro, CGC, escape, …) 
% Shortcomings of current modelling in small systems (need to argue how run 3 and 4 can improve this)

\subsection{Key observables}

% Multiplicity distribution used, with reference?

% vn, cn
%   Higher orders
%   Subevent method
%   With PID + strangeness
%   HF + quarkonia
%   P(vn) should be within reach in pp even!
% Strangeness enhancement
% Energy loss
%   “RAA”
%   Hadrons (and jets)
%   Definition of RAA in pp to be discussed
%   X+jets (X=photon,W,Z,[hadron,jet])
%   Heavy Flavour
% Thermal radiation?

\subsection{Data-taking strategy}

% High multiplicity triggering, pile up, MB sample
% pp
%   ALICE: additional several month pp program
%   ATLAS/CMS: special runs (mu~1) or special conditions at end of fill
%   LHCb: either in nominal (mu~5) or special running
%   HM sample: 200 pb-1 pp (per experiment)
%   How much MB for low-multiplicity questions?
% p-Pb scheduled run
%   1000-2000 nb-1

\subsection{Summary}

\end{document}
