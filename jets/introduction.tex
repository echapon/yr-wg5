\subsection{Introduction}

%{\it (pmj Nov 5: general intro text moved here from the Jet Deflection section...please integrate as appropriate but it seems to me that this may work as the first para...)} 
The most direct way to measure the structure of matter is the controlled scattering of a beam of probe particles. This approach was used to discover the atomic nucleus, and quarks and gluons, and it is employed today to explore the partonic structure of nucleons and nuclei. However, the partonic phase of the QGP lives for $\sim10^{-23}$ seconds before breaking up into its hadronic remnants, so that probing it by the scattering of an externally-generated beam is impossible in practical terms. As an alternative, energetic jets arising from high-$Q^2$ processes in the same nuclear collision that generates the QGP provide internally-generated probes that may be applied for this purpose~\cite{Bjorken:1982tu,Gyulassy:1990ye,Baier:1994bd,Zakharov:2018rst,Gyulassy:1999zd,Wiedemann:2009sh}. 

Jets, observed as collimated sprays of energetic particles, were predicted by Quantum Chromo Dynamics (QCD) to form in high energy collisions. They constitute a substantial part of the background in beyond the Standard Model physics searches and were instrumental in the Higgs boson discovery. While jet evolution in vacuum is well understood, the question of how jets interact with a dense deconfined medium remains an active field of study, that is largely driven in the recent years by the unprecedented experimental capabilities of the RHIC and LHC accelerators and detectors.  Understanding from first principles how a jet evolves as a multi-partonic system, spanning a large range of scales (from  $\sim$1 GeV to $\sim$1 TeV) is crucial to quantitatively probe the Quark Gluon Plasma (QGP).  The successful description of bulk observables by viscous hydrodynamic calculations with a small viscosity to entropy ratio have led to the standard picture of a strongly coupled plasma. However, due to the property of asymptotic freedom in QCD, the produced matter is expected to behave differently at smaller and smaller distances which can only be accessed with well calibrated probes, namely, QCD jets.

High-$Q^2$ processes between the partonic constituents of colliding nucleons occur early in the collision. Further interactions of the outgoing partons with the hot and dense QCD medium produced in heavy ion collisions are expected to modify the angular and momentum distributions of final-state jet fragments relative to those in proton-proton collisions. This process, known as jet quenching, can be used to probe the properties of the QGP~\cite{Bjorken:1982tu,Gyulassy:1990ye,Baier:1994bd,Zakharov:2018rst,Gyulassy:1999zd,Wiedemann:2009sh}. Jet quenching was first observed at RHIC, BNL~\cite{Adcox:2001jp,Adler:2003qi,Adler:2002xw,Adler:2002tq,Adams:2003kv,Adams:2003im,Adams:2006yt,Arsene:2003yk,Back:2003qr,Adamczyk:2016fqm,Adamczyk:2017yhe} and then at the CERN LHC~\cite{Aamodt:2010jd,Aamodt:2011vg,Aad:2015wga,CMS:2012aa,Aad:2010bu,Chatrchyan:2012nia,Aad:2012vca,Abelev:2013kqa,Adam:2015ewa,Khachatryan:2016odn,Adam:2015doa} by studying the redistribution of energy radiated from the parton because of interactions with the QGP. More recent detailed analyses have focused on modifications to the distribution of final-state particles emitted in the parton's shower~\cite{Chatrchyan:2013kwa,Chatrchyan:2014ava,Aaboud:2017bzv,Acharya:2017goa,Acharya:2018uvf,Sirunyan:2017bsd,Sirunyan:2018gct}.

One of the main goals of the RHIC and LHC heavy ion physics programs is utilization of jets and their decay products, including high $p_{\mathrm{T}}$ hadrons formed by light and heavy quarks, to investigate the QGP properties. A milestone in this program is the extraction of the transport coefficient $\hat q $ by the JET Collaboration~\cite{Burke:2013yra}, based on inclusive hadron suppression measurements at RHIC and the LHC. However, this result has significant systematic uncertainties, due both to theoretical issues and to the limited view provided by inclusive hadron suppression measurements into the fundamental processes underlying jet quenching. A more complete picture requires measurements of reconstructed jets and their in-medium modification. 

At the LHC, the collision energy is over an order of magnitude larger than at RHIC. Jet production cross-sections are correspondingly larger, enabling the study of hard processes over a wider kinematic range. Detectors at both facilities have extensive capabilities to study fully-reconstructed jets by grouping the detected particles within a given angular region into a jet, thereby capturing a significant fraction of the parton shower. Jets are a key diagnostic of the QGP, as their interactions with this new state of matter reveal its properties. The interaction with the medium can result in a broadening of the jet profile with respect to vacuum fragmentation. In this case, for a given jet size and a fixed initial parton energy, the energy of the jet reconstructed in heavy ion collisions will be smaller than in vacuum. In the case where the gluons are radiated inside the cone, the jet is expected to have a softer fragmentation and a modified density profile compared to jets in vacuum. Jets may also scatter coherently in the medium, and measurements of jet deflection may provide a direct probe of the micro-structure of the QGP. Fully reconstructed jets provide better theoretical control than high $p_{\mathrm{T}}$ hadrons because they are less sensitive to non-perturbative physics and therefore have the potential to provide a better characterization of the QGP. 
%{\it (pmj comment: this is an in-principle argument, taken directly from jet studies in vacuum, but it has afaik not yet be substantiated with an actual analysis. There is for instance no equivalent yet of the JET analysis of $\hat{q}$ using reconstructed jets. Many of us are working hard to substantiate the claim that jet measurements provide deeper and better controlled connection to theory than hadron measurements, but that is still work in progress to a large extent. This fact should be noted - in fact it provides important context for more precise and extensive jet measurements in Run 3.4.)}.
Furthermore, major theoretical and experimental advances were made recently in understanding the evolution of parton showers in a QCD medium with the development of novel jet substructure observables.

In the following sections the expected performance using a total integrated luminosity of 10 $\mathrm{nb}^{-1}$ of PbPb data, which is expected for HL-LHC, for a selection of key jet observables will be discussed. %\unit[10]{$nb^-1$}
