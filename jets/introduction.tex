\subsection{Introduction}
Jets, observed as collimated sprays of energetic particles, were predicted by Quantum Chromo Dynamics (QCD) to form in high energy collisions. They constitute a substantial part of the background in beyond the Standard Model physics searches and were instrumental in the Higgs boson discovery. While jet evolution in vacuum is well understood, the question of how jets interact with a dense deconfined medium remains an active field of study, that is largely driven in the recent years by the unprecedented experimental capabilities of the RHIC and LHC accelerator and detectors.  Understanding from first principles how a jet evolves as a multi-partonic system, spanning a large range of scales (from  $\sim$1 GeV to $\sim$1 TeV) is crucial to quantitatively probe the Quark Gluon Plasma (QGP).  The successful description of bulk observables by viscous hydrodynamic calculations with a small viscosity to entropy ratio have led to the standard picture of a strongly coupled plasma. However, due to the property of asymptotic freedom in QCD, the produced matter is expected to behave differently at smaller and smaller distances which can only be accessed with well calibrated probes, namely, QCD jets.

Scattering processes with large momentum transfer $Q$ between the partonic constituents of colliding nucleons occur early in the collision. Further interactions of the outgoing partons with the hot and dense QCD medium produced in heavy ion collisions is expected modify the angular and momentum distributions of final-state hadronic jet fragments relative to those in proton-proton collisions. This process, known as jet quenching, can be used to probe the properties of the QGP~\cite{Gyulassy:1990ye,Baier:1994bd}. Jet quenching was first observed at RHIC, BNL~\cite{Adcox:2001jp,Adler:2003qi,Adams:2003kv,Arsene:2003yk,Back:2003qr,Adamczyk:2016fqm,Adamczyk:2017yhe} and then at the CERN LHC~\cite{Aamodt:2010jd,Aamodt:2011vg,Aad:2015wga,CMS:2012aa,Aad:2010bu,Chatrchyan:2012nia,Aad:2012vca,Abelev:2013kqa,Adam:2015ewa,Khachatryan:2016odn,Adam:2015doa} by studying the redistribution of energy radiated from the parton because of interactions with the QGP. More recent detailed analyses have focused on modifications to the distribution of final-state particles emitted in the parton's shower~\cite{Chatrchyan:2013kwa,Chatrchyan:2014ava,Aaboud:2017bzv,Acharya:2017goa}.

The goal of detailed jet quenching studies in to extract the transport properties of hot QCD matter. Jets are extended multi-scale objects that interact strongly with the deconfined matter produced in heavy ion collisions. They provide a unique way to probe the non-equilibrium dynamics of non-Abelian plasmas, in particular how energy is transported from hard to soft modes in hot QCD matter; connecting to the physics of wave turbulence and plasma instabilities. The internal structure of a jets provides information about QCD dynamics in a dense medium.
%How color coherence is altered in the presence of a colored medium and how does it affect the mechanism of hadronization.
%How a jet interacts with a strongly coupled QGP.

%\item {\bf Extracting microscopic properties of the QGP using jets}

One of the main goals of the RHIC program was to use hard probes such as high $p_{\mathrm{T}}$ hadrons formed by light and heavy quarks to investigate the QGP properties. This program culminated in the extraction of the transport coefficient $\hat q $ by the JET Collaboration. However, many sources of uncertainties were reported which are mainly theoretical. 
At the Large Hadron Collider (LHC), the collision energy was increased over an order of magnitude compared to RHIC, which enhanced the parton cross-section, allowing the study of hard processes over a wider kinematic range. In addition, the detector technologies were optimized for the study of fully reconstructed jets capturing a significant amount of the parton shower by grouping the detected particles within a given angular region into a jet. Jets are a key diagnostic of the QGP, as their interactions with this new state of matter reveal its properties. The interaction with the medium can result in a broadening of the jet profile with respect to vacuum fragmentation. In this case, for a given jet size and a fixed initial parton energy, the energy of the jet reconstructed in heavy ion collisions will be smaller than in vacuum. In the case where the gluons are radiated inside the cone, the jet is expected to have a softer fragmentation and a modified density profile compared to jets in vacuum. 
Fully reconstructed jets allow a better theoretical control than high $p_{\mathrm{T}}$ hadrons because they are less sensitive to non-perturbative physics and therefore have the potential provide a better characterization of the QGP. Furthermore, major theoretical and experimental advances were made recently in understanding parton showers in a QCD medium with the development of novel jet substructure observables.

In the following sections the expected performance using a total integrated luminosity of 10 $\mathrm{nb}^{-1}$ of PbPb data, which is expected for HL-LHC, for a selection of key jet observables will be discussed. %\unit[10]{$nb^-1$}
