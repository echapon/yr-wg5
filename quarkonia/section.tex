% don't remove the folling lines, and edit the defintion of \main if needed
\documentclass[../report.tex]{subfiles}
\providecommand{\main}{..}
\IfEq{\jobname}{\currfilebase}{\AtEndDocument{\biblio}}{}
\IfEq{\jobname}{\currfilebase}{%file for shortcuts

\newcommand{\nch}{\ensuremath{N_{\mathrm {ch}}\xspace}}
\newcommand{\dNdeta}{\mathrm{d}N_\mathrm{ch}/\mathrm{d}\eta}
}{}

\begin{document}

\section{Quarkonia}

\label{sec:quarkonia}

{ \small
\noindent \textbf{Coordinators}: A. Andronic (M\"{u}nster University), E. Chapon (CERN)

\noindent \textbf{Contributors}: 
% contributors to the text
E. Ferreiro (Santiago de Compostela University), J.-P. Lansberg (Institut the Physique Nucl\'{e}aire d'Orsay), R. Rapp (Texas A\&M University),
% ATLAS and LHCb contacts
% Z. Citron (Ben-Gurion), S. Chen (Cagliari),
% ALICE figures
J. Castillo Castellanos (CEA Saclay),
C. Cheshkov (IPN Lyon),
% CMS figures: public document CMS-PAS-FTR-18-024 (in preparation)
% CMS contributors: J. Park (Korea University), (E. Chapon)
% theorist inputs
X. Du (Texas A\&M University),
M. Strickland (Kent State University), R. Venugopalan (BNL), I. Vitev (LANL)
% TBD
% N. Brambilla, P. Zhuang
}

%%%%%%%%%%%%%%%%%%%%%%%%%%%%%%%%%%%%%%%%%%%%%%%%%%%%%%%%%%%%%%%%
\subsection{Introduction} % R. Rapp
\label{sec_intro}
%%%%%%%%%%%%%%%%%%%%%%%%%%%%%%%%%%%%%%%%%%%%%%%%%%%%%%%%%%%%%%%%
A key objective in high-energy heavy-ion physics is to determine the in-medium forces that give rise to the remarkable many-body features of the quark-gluon plasma (QGP).
In the QCD vacuum, the unravelling of the fundamental force between two static color charges was made possible by the discovery of the charmonium and bottomonium states in the 1970's. 
Subsequent quantitative analyses of the bound-state spectra established a phenomenological potential of the Cornell type~\cite{Eichten:1979ms}, 
\begin{equation}
V(r) = -\frac{4}{3} \frac{\alpha_s}{r} + \sigma r \ ,
\label{V}
\end{equation} 
with a colour-Coulomb term due to gluon exchange dominant at short distances and a linear term with string tension $\sigma\simeq0.9$\,GeV/fm to account for confinement at large distance. 
This potential has also been quantitatively confirmed by lattice-QCD (lQCD) calculations~\cite{Bali:2000gf}, cf.~also Ref.~\cite{Brambilla:2004jw}. 
The corresponding effective field theory of QCD, potential non-relativistic QCD (pNRQCD), allows for the definition of the static potential in a  $1/m_Q$ expansion for large heavy-quark mass, $m_Q$~\cite{Brambilla:1999xf,Brambilla:2004wf}. 
The heavy-quark (HQ) potential thus provides a well calibrated starting point to probe the QCD medium, and the in-medium spectroscopy of quarkonia is the natural tool to carry this out in heavy-ion collisions, cf.~\cite{Rapp:2008tf,BraunMunzinger:2009ih,Kluberg:2009wc,Mocsy:2013syh,Liu:2015izf} 
for recent reviews.
The string term in the HQ potential, eq.~(\ref{V}), characterizes the long-range nonperturbative part of the force and is associated with the confining property of QCD. It is expected to play a critical role in the transition from hadronic to partonic degrees of freedom, and may well be responsible for the remarkable transport properties of the QGP, {\it i.e.}, its strongly coupled nature, up to temperatures of 2-3 times the (pseudo-)critical temperature, $T_c$~\cite{Liu:2016ysz}. 

Much like in vacuum, a systematic investigation of the in-medium force must involve the {\em spectroscopy} of different states, as they subsequently dissolve with increasing temperature. In this sense, quarkonia are not straightforwardly usable as a thermometer, which would imply that their dissociation pattern provides a known gauge. In the vacuum, only the 1S ground-state bottomonia (\PGUP{1S} and \PGhb) are small enough in size to be mostly bound by the colour-Coulomb force. All excited bottomonia and all charmonia are predominantly bound by the nonperturbative string term (and/or residual mesonic forces). 
Thus, charmonia and excited bottomonia are excellent probes of the in-medium confining force, as originally envisioned for the \PJgy~\cite{Matsui:1986dk}.
However, in the cooling of the expanding fireball, quarkonia can also be ``(re)generated" through recombination of individual heavy quarks and anti-quarks diffusing through the medium. It is important to emphasize that quarkonium formation occurs also from quarks and antiquarks from different initial pairs (hence the spelling ``(re)generation").
This mechanism \cite{BraunMunzinger:2000px,Thews:2000rj} has turned out to be critical in understanding the \PJgy production systematics at the LHC where (re)generation seems to constitute the major part of the yield observed in central \PbPb collisions.
The data is also compatible with production of \PJgy exclusively through statistical hadronisation at the crossover phase boundary \cite{Andronic:2017pug}.
Precise measurements of the $c\bar{c}$ production cross section and the charm-quark diffusion coefficient in Run 3 and 4 will be important for making a more definite statement; these are key objectives discussed in the chapter~\ref{sec:HI_HF} on open heavy-flavor production. Information from \pt spectra and elliptic flow will help complete the picture.

For bottomonia, the current understanding suggests that (re)generation is less important (although still significant) for \PGUP{1S}, but possibly figures as a major component in the strongly suppressed yield of excited states, especially the \PGUP{2S}. It is therefore of great importance to obtain additional information about the production "times" of the observed yields, in particular through $\pt$ spectra and elliptic flow which contain information about the fireball's collectivity imprinted on the quarkonia by the time of their decoupling. A schematic illustration of the current knowledge extracted from ``in-medium quarkonium spectroscopy", {\it i.e.}, their production systematics in heavy-ion collisions is shown in Fig.~\ref{FigQ:pot}.        

\begin{figure}[!h]
\begin{center}
\includegraphics[width=0.6\columnwidth]{\main/quarkonia/fig/in-med-pot.pdf}
\end{center}
\vspace{-0.5cm}
\caption{The vacuum heavy-quark potential as a function of $Q\bar Q$ separation. The horizontal lines indicate the approximate locations of the vacuum bound states while the vertical arrows indicate the minimal screening distances of the media produced at the SPS, RHIC and LHC, as extracted from quarkonium production systematics in \PbPb and Au-Au collisions, along with approximate initial temperatures reached in these collisions. Figure taken from Ref.~\cite{Rapp:2017chc}.}
\label{FigQ:pot}
\end{figure}

On the theoretical side, the basic objects are the quarkonium spectral functions which encode the information on the quarkonium binding energies, in-medium HQ masses and the (inelastic) reaction rates.  Ample constraints on the determination of the quarkonium spectral functions are available from thermal lQCD, {\it e.g.}, in terms of the heavy-quark free energy, euclidean and spatial quarkonium correlation functions, and HQ susceptibilities, and are being implemented into potential model calculations~\cite{Wong:2004zr,Mocsy:2005qw,Alberico:2006vw,Brambilla:2008cx,Riek:2010py,Burnier:2015tda,Liu:2017qah}.
In particular, the role of dissociation reactions has received increasing attention. Early calculations of gluo-dissociation~\cite{Bhanot:1979vb,Kharzeev:1994pz} or inelastic parton scattering~\cite{Grandchamp:2001pf} have been revisited and reformulated, \eg as a singlet-to-octet transition mechanism~\cite{Brambilla:2008cx} or in terms of an imaginary part of a two-body potential~\cite{Laine:2006ns}, respectively. In particular, the latter accounts for interference effects which reduce the rate relative to ``quasi-free" dissociation~\cite{Grandchamp:2001pf} in the limit of small binding; interference effects can also be calculated diagramatically~\cite{Park:2007zza}; they ensure that, in the limit of vanishing size, a color-neutral $Q\bar Q$ dipole becomes ``invisible" to the color charges in the QGP.

The information from the spectral functions can then be utilized in heavy-ion phenomenology via transport models. The latter provide the connection between first-principles information from lQCD and experiment that greatly benefits the extraction of robust information on the in-medium QCD force and its emergent transport properties, most notably the (chemical) equilibration rates of quarkonia. 
Thus far most transport models are based on rate equations and/or semiclassical Boltzmann equations. In recent years quantum transport approaches have been developed using, \eg a Schr\"odinger-Langevin~\cite{Blaizot:2015hya,Katz:2015qja,Kajimoto:2017rel} or density-matrix~\cite{Akamatsu:2014qsa,Brambilla:2016wgg} formulation. These will enable to test the classical approximation underlying the Boltzmann and rate equation treatments and ultimately quantify the corrections.  Quantum effects may be particular relevant at high $\pt$ in connection with the in-medium formation times of quarkonia, augmented by the Lorentz time dilation in the moving frame; schematic treatments of this effect in semiclassical approaches suggest that varying formation times can leave observable differences for high-momentum charmonia and bottomonia~\cite{Song:2015bja,Hoelck:2016tqf,Du:2017qkv,Aronson:2017ymv,Krouppa:2017jlg}. 
Finally, the implementation of phase-space distributions of explicitly diffusing heavy quarks into quarkonium transport is being investigated by various groups (see, \eg Ref.~\cite{Yao:2017fuc}), which, as mentioned above, will provide valuable constraints on the magnitude and \pt dependence of (re)generation processes.     

%ADD/MOVE HERE general remarks on benefits in Run3,4 ?

The higher statistics of experimental data in Runs 3,4, combined with improved detector performance and measurement techniques, will allow to significantly improve over the current measurements, with extended kinematic coverage (in \pT) and allowing to reach also currently-unmeasured quarkonium states, like 
%\PGyP{2S} and 
\PGUP{3S}.
The complementarity (and overlap) of all 4 LHC experiments is crucial in this endeavour and will call for a data combination strategy.
Quarkonia are measured in the dimuon channel in ATLAS, CMS, LHCb and ALICE at forward rapidity, and in the dielectron channel with ALICE at midrapidity.
We present below data projections and simulations for a selection of observables and compare to model predictions (which sometimes constitute the basis for the projections). The shown model uncertainties represent the current knowledge; we expect significant improvements both in what concerns the conceptual aspects discussed above as well for the input parameters, which will be constrained by data and theory (for instance in what concerns nuclear PDFs, see also Section~\ref{sec:nPDFs}).

%ADD ALSO kinematic coverage by experiments (with Figs.?) ...like:
%Charmonia are measured in PbPb collisions by ALICE (\pT$>$0, for  $|y|<0.9$ and $2.5<y<4$), CMS ($\pT>6.5\UGeVc$, for  $|y|<1.2$, $\pT>3.5\UGeVc$, for  $1.2<|y|<2.4$ ?), and ATLAS ($\pT>7\UGeVc$, for  $|y|<2$).
% need to also have the coverage with the Phase-II upgrades for ATLAS/CMS... and account for the fact that e.g. CMS can also measure upsilons down to 0 pt, and J/psi down to ~0 pt under certain conditions, see J/psi v2 in pPb for instance


%%%%%%%%%%%%%%%%%%%%%%%%%%%%%%%%%%%%%%%%%%%%%%%%%%%%%%%%%%%%%%%%
\subsection{Charmonia in \PbPb collisions} % (Main contributor: Anton Andronic)}
%%%%%%%%%%%%%%%%%%%%%%%%%%%%%%%%%%%%%%%%%%%%%%%%%%%%%%%%%%%%%%%%

A remarkable discovery at the LHC was that the suppression of \PJgy is significantly reduced in comparison to lower energies~\cite{Abelev:2012rv} and that the nuclear modification factor \raa is enhanced towards lower \pT~\cite{Abelev:2013ila,Adam:2016rdg}, confirming predictions of (re)generation at the (crossover) phase boundary of QCD~\cite{BraunMunzinger:2000px} or throughout the deconfined phase~\cite{Thews:2000rj,Zhao:2011cv}. 
Recently, the measurement of a significant elliptic flow coefficient $v_2$ of \PJgy~\cite{Khachatryan:2016ypw,Acharya:2017tgv,Aaboud:2018ttm} brought further crucial support that charm quarks thermalize in QGP.

%Projections and simulations were done in ALICE for the Upgrade proposal \cite{Abelevetal:2014cna,CERN-LHCC-2013-014}.

%Current results of charmonium production in the Statistical Hadronization Model \cite{Andronic:2017pug}) and in transport models \cite{Du:2015wha,Liu:2015izf}), that describe (well) the features of the data, are included below alongside experimental projections.

%but several open questions remain. 
%The description of the \pT~ dependence of J/$\psi$ production is not yet 
  
\begin{figure}[h]
\begin{center}
 \includegraphics[width=0.66\textwidth]{\main/quarkonia/fig/alice/jpsiv2proj10nb06_01_47.pdf}
\end{center}
 \caption{Projected measurement of elliptic flow coefficient $v_2$ as a function of \pT~ for \PJgy mesons (measured in ALICE, for $2.5<y<4$), for the centrality class 20--40\%, in comparison to model calculations \cite{Du:2015wha}.}
\label{FigQ:v2pTPbPb}
\end{figure}

The projected measurement of \PJgy $v_2$ as a function of \pT\ for the centrality class 20--40\%, for $2.5<y<4$, in comparison to model calculations \cite{Du:2015wha} is shown in Fig.~\ref{FigQ:v2pTPbPb}. Currently, transport model calculations \cite{Zhou:2014kka,Du:2015wha} underestimate the data for $\pT\gtrsim 6\UGeVc$~\cite{Acharya:2017tgv}. Since the current ALICE data are for inclusive \PJgy production and the component from B meson decays is expected to be significant for high \pT, only the data in Runs 3,4 will allow to conclude, aided by precision separation of the prompt and B-decay components. Measurements at midrapidity will become significant too and also a very good measurement of the $v_3$ coefficient is in reach; polarization will be measured too~\cite{Abelevetal:2014cna}, providing further insight in the different production mechanisms involved in \PbPb collisions as compared to \pp.

\begin{figure}[h]
\begin{center}
 \includegraphics[width=0.48\textwidth]{\main/quarkonia/fig/atlas/atlas_promptjpsi_models}
 \includegraphics[width=0.47\textwidth]{\main/quarkonia/fig/cms/Jpsi_pTvsLumi}
\end{center}
\caption{Left: \RAA vs \pT~ for prompt \PJgy in central (0--20\%) collisions (ATLAS, $|y|<2$, \cite{Aaboud:2018quy}). Right: CMS projection for the reach in \pT for minimum bias (0--100\%) collisions.
}
\label{FigQ:JpTPbPb}
\end{figure}

At high \pT, a raising trend is currently hinted by Run~2 measurements~\cite{Sirunyan:2017isk,Aaboud:2018quy}. The production mechanisms cannot be resolved with these data alone, given the statistical limitation in the data, see Fig.~\ref{FigQ:JpTPbPb} (left). Run 3--4 data, as illustrated in Fig.~\ref{FigQ:JpTPbPb} (right) with a CMS projection for the \pT reach, will allow to conclude on the important question of \PJgy
formation as determined by the Debye screening mechanism \cite{Kopeliovich:2014una,Aronson:2017ymv} or by %(re)generation from
energy loss of the charm quark or the $\text{c}\bar{\text{c}}$ pair~\cite{Spousta:2016agr,Arleo:2017ntr}.

%\begin{figure} \begin{center} \includegraphics[width=\textwidth]{\main/quarkonia/fig/alice/jpsi_polarisation.png} \end{center} \caption{prompt J/psi polarisation~\cite{Abelevetal:2014cna}} \end{figure}


\begin{figure}[h]
\begin{center}
 \includegraphics[width=0.49\textwidth]{\main/quarkonia/fig/alice/psi2s2psi_Npart_pbpb502_y0}
 \includegraphics[width=0.49\textwidth]{\main/quarkonia/fig/alice/psi2s2psi_Npart_pbpb502_y3}
\end{center}
 \caption{Production ratio $\PGyP{2S}/\PJgy$ vs. $N_{part}$ for $|y|<0.9$ (left) and $2.5<y<4$ (right) \cite{Abelevetal:2014cna,CERN-LHCC-2013-014}. Model predictions in the transport approach \cite{Du:2015wha}  and from statistical hadronization \cite{Andronic:2017pug} are included.}
\label{FigQ:psi2SPbPb}
\end{figure}

The measurement of \PGyP{2S} mesons is more difficult than that of \PJgy, because of a much small production cross section and even larger suppression in \PbPb, yielding a very low signal to background ratio.
The projections for the measurement of the \PGyP{2S} state in ALICE are shown in Fig.~\ref{FigQ:psi2SPbPb} as a function of centrality and compared to model predictions in the transport approach \cite{Du:2015wha}  and from the statistical hadronization model \cite{Andronic:2017pug}. This (\pT-integrated) measurement will significantly contribute to make a distinction between the two models that currently describe charmonium data. Projections are also available from the CMS experiment~\cite{CMS-PAS-FTR-17-002}.
%\begin{figure}\begin{center}\includegraphics[width=0.6\textwidth]{\main/quarkonia/fig/theory/psi_dr_vitev.png}\end{center} \caption{prompt psi(2S) (and J/psi, or ratio) vs. pT (at low and high pT) } \end{figure}
Other states, for instance \PGcc, may be measured too, albeit the measurement down to $\pT=0$ will remain challenging.

%\begin{figure} \caption{prompt psi(2S) v2 vs pT for 30--50\%} \end{figure}

%\begin{figure} \begin{center} \includegraphics[width=0.32\textwidth]{\main/quarkonia/fig/theory/chic.png} \end{center} \caption{RAA of chic(1P), X(3872) vs pT for $|y|<2.4$} \end{figure}

% \clearpage

%%%%%%%%%%%%%%%%%%%%%%%%%%%%%%%%%%%%%%%%%%%%%%%%%%%%%%%%%%%%%%%%
\subsection{Bottomonia in \PbPb collisions}% (Main contributor: Emilien Chapon)}
%%%%%%%%%%%%%%%%%%%%%%%%%%%%%%%%%%%%%%%%%%%%%%%%%%%%%%%%%%%%%%%%

% To cite: Strickland papers, Rapp papers, Nora, Debasish, exp. papers...

% Intro about bottomonia: specificities wrt charmonia, overview of theoretical ingredients.
% Mass, beauty vs charm, 3 states, Tdiss, no B feed-down, feed-down, ...
% split in th vs exp differences
The study of bottomonia with \PbPb data from the Runs 3 and 4 of the LHC can bring further information on the various physics aspects described above, and more.
Although their production is a priori subject to the same effects as charmonia, in practice the two quarkonium families feature some fundamental differences.
Binding energies differ, which is reflected in the different dissociation temperatures: about twice the critical temperature $T_c$ for \PGUP{1S}, much higher than the about
$1.2 T_c$ for \PJgy or \PGUP{2S}, for instance~\cite{Mocsy:2007jz}. The feed-down pattern is also more complex: while the contribution of B meson decays is specific to charmonia, more states can
contribute to the different $S$-wave bottomonia, owing to decays of \PGUP{2S} and \PGUP{3S}, as well as the many states of the \PGcb family~\cite{Andronic:2015wma}. In practice, in \pp collisions, up to 30--40\% of the measured 
\PGUP{1S} and \PGUP{2S} yields actually result from the feed-down from other states. At the same time, this important feature of bottomonium production means that
a large portion of measured \PGUP{1S} suppression can be due to the stronger suppression of the feed-down states -- \PGUP{2S} and \PGUP{3S} mesons also receive a significant contribution from feed-down.
In addition, the feed-down fractions, for the contribution of the 
different states to the measured bottomonium states, are constrained experimentally as a function of $\pt$ but with limited precision (in fact there is no measurement of the \PGcb{(3P)} to \PGUP{3S} feed-down for $\pt<20\UGeVc$), which is one source of uncertainty in the models.
The impact of (re)generation from uncorrelated $\text{b}\bar{\text{b}}$ is also expected to be much smaller than for charmonia, because of the much smaller number of $\text{b}\bar{\text{b}}$
pairs per \PbPb event compared to $\text{c}\bar{\text{c}}$. The importance of regeneration for bottomonia is however still very model dependent, and no unambiguous experimental signal
for it has been found yet. Possible ways of constraining this contribution will be discussed in this section.

Experimentally, the higher mass of bottomonia compared to charmonia implies higher $\pt$ decay leptons, allowing the ATLAS and CMS experiments to measure the production down to 0 $\pt$,
as is possible for ALICE for both charmonia and bottomonia~\cite{Abelev:2014nua,Acharya:2018mni}. The proximity in mass between the different mass states, especially between the \PGUP{2S} and \PGUP{3S} states, also 
means that good muon (or electron) momentum resolution is essential to their measurement, especially for excited states. 
% , while the addition of the MFT will also beneficial in improving the muon momentum resolution of the ALICE detector.

% Summary of Run1+Run2 results 
It is useful to remind quickly the status in 2018, based on results from Run1 and early Run2 LHC data as well as RHIC data. \PGU production is found to be suppressed in \PbPb compared to \pp
collisions, in all rapidity, $\pt$ and centrality ranges measured. Suppression is stronger in central events, as expected from the hotter and longer-lived medium in such events.
Another striking feature is that excited states are more suppressed than the ground state, the \PGUP{3S} being still unmeasured in AA collisions ($\raa(\PGUP{3S})<0.094$ at 95\% confidence
level, for $\sqrtsnn = \unit[5.02]{TeV}$~\cite{Sirunyan:2017lzi,Sirunyan:2018nsz}). In addition, the large suppression found for \PGUP{1S} in central collisions does not seem compatible with unmodified direct \PGUP{1S} production,
though uncertainties on non-QGP effects (initial state modifications and final state effects) do not allow for an undebatable statement regarding \PGUP{1S} melting in the medium.
No significant dependence of the suppression of \PGU states is found on collision energy or rapidity.

% % FIXME what follows should go to the introduction
% Experimentally, the main provision of Runs 3 and 4 data, compared to Runs 1 and 2, will be the much higher quantity of data, expected to be \unit[10]{nb}$^{-1}$, an increase of a factor about
% 100 for the ALICE experiment %FIXME check
% and about 5 for the ATLAS and CMS experiments. The much higher precision will be most beneficial in regions of the phase space where the cross section is smaller: for instance at
% forward rapidity and in peripheral collisions. At the same time, there is access to higher $\pt$ bottomonia, up to about 50\,GeV with ATLAS and CMS, where one could look for hints of an
% increasing \raa, as found for prompt \PJgy in Run~2 data~\cite{Sirunyan:2017isk,Aaboud:2018quy}. In addition, more data will be even more appreciable for \PGUP{2S} and \PGUP{3S}: 
% the modification of the former in \PbPb collisions is known with limited precision today, and only upper limits on the production of the latter are available. More data will also have
% an impact on the main systematic uncertainties impacting results today. Efficiency uncertainties will be reduced, thanks to the larger calibration datasets recorded (minimum bias data
% and \PJgy data for tag-and-probe corrections). At the same time, uncertainties on the modelling of signal and background mass shapes will also be reduced, some unrealistic 
% parametrisations being rejected by the more precise data.
% 
% % comments on Phase-I and Phase-II upgrades
% The addition of the Muon Forward Tracker (MFT) during LS2 will also impact \PGU measurements from ALICE. The main improvement will be a reduction in the background, yielding to better signal over background
% ratios. This will further improve the precision for \PGUP{1S} measurements, as well as enable differential $\PGUP{2S}+\PGUP{3S}$ measurements.
% In addition, the ATLAS and CMS detectors will also undergo major upgrades between Runs 3 and 4, with an inner tracker
% extending to $|\eta|\lesssim 4.0$ for ATLAS (CMS), and with extended muon coverage to $|\eta|\lesssim 2.7$ (3.0). While the detector improvements will have a smaller impact than the
% increase in statistics, this increase in pseudorapidity coverage is appreciable in also giving an overlap with the range of ALICE and LHCb.

% Discuss regeneration

\begin{figure}
\begin{center}
 \includegraphics[width=0.4\textwidth]{\main/quarkonia/fig/cms/CMS-PAS-FTR-17-002_Figure_008-b.pdf}
 \includegraphics[width=0.4\textwidth]{\main/quarkonia/fig/alice/craa_cent}
%  \includegraphics[width=0.32\textwidth]{\main/quarkonia/fig/theory/rapp_raa_npart_CMS.png}
\end{center}

 \caption{Centrality dependence of \PGUP{1S}, \PGUP{2S} and \PGUP{3S} \raa, as projected by the CMS\cite{CMS-PAS-FTR-17-002,Krouppa:2016jcl} (left) and ALICE (centre) experiments, and from a transport model\cite{Du:2017qkv}}
 \label{fig:upsi_raa_npart}
\end{figure}

% \begin{figure}
%  \begin{center}
%   \includegraphics[width=0.32\textwidth]{\main/quarkonia/fig/theory/rapp_dr_npart.png}
%  \end{center}
% 
%  \caption{Y(2S,3S)/Y(1S) vs Npart~\cite{Du:2017qkv}}
% \end{figure}

% discuss centrality dependence: importance of different ingredients... potential, initial T, hydro vs transport, etc

The main striking features of \PGU suppression can be observed in Fig.~\ref{fig:upsi_raa_npart} and in current data: first a strong dependence of the suppression with the collision
centrality, with a stronger suppression in the most central collisions, and also a stronger suppression of the excited states compared to the ground state. Good qualitative agreement
is already found between models and data regarding this suppression, within current uncertainties. Figure~\ref{fig:upsi_raa_npart} shows the the projected uncertainty on the \raa of \PGUP{1S} will be much smaller than the current model uncertainties.

Differences exist however in the treatment of the suppression of the bottomonia in the medium.
For instance, the heavy quark potential, qualitatively a Debye screened potential above the deconfinement temperature, as proposed originally~\cite{Matsui:1986dk}, is one important
ingredient. Some models assume a real potential, using usually the free energy or the internal energy (as in Ref.~\cite{Du:2017qkv}). More recently, developments from lQCD
have allowed to also study the imaginary part of the potential, which physics implications are under study but may be related to Landau damping and singlet-octet transitions. Through the 
use of such lQCD-vetted potential, it has been shown~\cite{Krouppa:2017jlg} that predictions are quite sensitive to this choice of potential, as compared to a perturbative one. Note that the coherence between the observed relative suppression in \PbPb and \pPb collisions was investigated in \cite{Ferreiro:2018wbd}.

Models also differ in the treatment of the evolution of the quarkonia with the medium. Frameworks include a transport model with a kinetic-rate equation~\cite{Du:2017qkv},
anisotropic hydrodynamics~\cite{Krouppa:2017jlg}, comovers~\cite{Ferreiro:2018wbd}, effective field theory in the framework of open quantum systems with 
a Lindblad equation~\cite{Brambilla:2017zei}. Precise predictions require many other ingredients. Some can be constrained using measurements in \pp collisions, such as
the feed-down fractions from other states, or in \pPb collisions for cold nuclear matter and initial state effects (including nPDF). In other cases, bottomonia may bring information
complementary to other probes, using the sensitivity of the suppression to the medium shear viscosity or to its initial temperature.

\begin{figure}
\begin{center}
%  \includegraphics[width=0.32\textwidth]{\main/quarkonia/fig/theory/strickland_pt.png}
%  \includegraphics[width=0.32\textwidth]{\main/quarkonia/fig/theory/rapp_raa_pt_CMS.png}
 \includegraphics[width=0.38\textwidth]{\main/quarkonia/fig/alice/craa_pt}
 \includegraphics[width=0.3\textwidth]{\main/quarkonia/fig/cms/Projection_RAA_vs_pt_wo3S.pdf}
 \includegraphics[width=0.3\textwidth]{\main/quarkonia/fig/cms/Upsilon1S_pTvsLumi}
\end{center}

 \caption{Projected \raa for \PGUP{1S} and \PGUP{2S} expected from the ALICE (left) and CMS (center) experiments, as a function of \pt, with \unit[10]{nb}$^{-1}$ of \PbPb data. The expected \pt reach for \PGUP{1S}, from the CMS experiment is
 also shown, as the position of the last \pt bin of the measurement, with constant number of observed \PGUP{1S}, as a function of integrated luminosity.
 }
 \label{fig:upsi_raa_pt}
\end{figure}

\begin{figure}
\begin{center}
%  \includegraphics[width=0.32\textwidth]{\main/quarkonia/fig/theory/strickland_pt.png}
%  \includegraphics[width=0.32\textwidth]{\main/quarkonia/fig/theory/rapp_raa_pt_CMS.png}
%  \includegraphics[width=0.32\textwidth]{\main/quarkonia/fig/alice/craa_pt}
 \includegraphics[width=0.38\textwidth]{\main/quarkonia/fig/alice/craa_y}
 \includegraphics[width=0.3\textwidth]{\main/quarkonia/fig/cms/Projection_RAA_vs_rap_wo3S.pdf}
\end{center}

 \caption{Projected \raa for \PGUP{1S} and \PGUP{2S} expected from the ALICE (left) and CMS (center) experiments, as a function of rapidity, with \unit[10]{nb}$^{-1}$ of \PbPb data.
 }
 \label{fig:upsi_raa_y}
\end{figure}


% Discuss pt / y dependence with boost... what we learn from it (also regeneration)
A precise measurement of the $\pt$ dependence of the \PGUP{1S} \raa will be possible using LHC data from Runs 3 and 4. At low and medium $\pt$, up to about 15\UGeV, the measurement is
sensitive to the possible regeneration component in \PGU meson production. Projections for the expected precision of \PGU measurements from the ALICE and CMS detectors
using \unit[10]{nb}$^{-1}$ of data after the Runs 3-4 are shown as a function of $\pt$ in Fig.~\ref{fig:upsi_raa_pt} and $y$ in in Fig.~\ref{fig:upsi_raa_y}, and compared to the expectations from two 
models~\cite{Krouppa:2017jlg,Du:2017qkv}. In the first model~\cite{Krouppa:2017jlg} (not shown), all the measured \PGU are from the primordial production and there is no regeneration, leading
to a rather flat \raa at low and medium $\pt$. Only at higher $\pt$ is a small rise predicted, which will be discussed below. In the second model~\cite{Du:2017qkv} however, a regeneration
component is considered, and several assumptions are explored, especially on the degree of thermalisation of the bottom quarks. 
% Indeed, because of the about three times higher mass of
% the bottom quark than that of the charm quark, it cannot be assumed that regenerated bottomonia would have a thermal blast-wave expression, as is a good approximation for charmonia.
% In Fig.~\ref{fig:upsi_raa_pt}, this second model~\cite{Du:2017qkv} uses a coalescence model instead, providing more realistic nonequilibrium $\pt$ spectra for the input b quarks. 
It predicts a maximum in the \raa as a function of $\pt$, at around 10\UGeVc for \PGUP{1S}. The current data is not precise enough to confirm or disfavour this feature, but Run~3+4 
data will allow to look for it.

Almost no rapidity dependence is expected at the LHC for the nuclear modification factor of \PGU mesons within the acceptance of ATLAS and CMS ($|\eta|\lesssim 2.5-3$), which can be better
tested using Run~3+4 data. A modest increase is predicted in the acceptance of ALICE, as can be seen in Fig.~\ref{fig:upsi_raa_y}, because of a cooler QGP. Again, this cannot be tested
within the current experimental uncertainties, but can be looked for in future data.

% Discuss high pt
Though not as sensitive as \PJgy to energy loss processes, because of their higher mass implying a lower boost at a given $\pt$, much can be learnt as well from the measurement of
\PGUP{1S} at high $\pt$. As can be seen in Fig.~\ref{fig:upsi_raa_pt}, it is expected that a measurement up to a $\pt$ of about 50\UGeVc\ can be performed with the ATLAS and CMS detectors with
\unit[10]{nb}$^{-1}$ of data, where either a small increase of the \raa or a bump around 10\UGeVc\ is predicted by current models.

\begin{figure}
\begin{center}
 \includegraphics[width=0.3\textwidth]{\main/quarkonia/fig/cms/proj_Y1S_cent5-60.pdf}
 \includegraphics[width=0.3\textwidth]{\main/quarkonia/fig/cms/proj_Y2S_cent5-60.pdf}
 \includegraphics[width=0.38\textwidth]{\main/quarkonia/fig/alice/canvasFlow.pdf}
\end{center}

 \caption{$v_2$ projections for the CMS (left and centre) and ALICE (right) experiments for the \PGUP{1S} and \PGUP{2} mesons, assuming the predictions from a transport model~\cite{Du:2017qkv}.}
 \label{fig:upsi_v2}
\end{figure}

% Discuss v2 
If we come back to the matter of regeneration, much can be learnt about it by a measurement of the elliptic flow of \PGUP{1S} mesons~\cite{Das:2018xel}, 
unmeasured to date in any collision system. 
A parallel can be
drawn with that of \PJgy, which is still not properly described by models. This observable requires a more detailed implementation of the dynamics of the interactions between the 
quarkonium and the medium: thermalisation of the heavy quarks, time dependence of regeneration, path length dependence of energy loss, as well as initial geometry fluctuations and 
elastic rescattering of the quarkonia in the medium. Thus, collective flow brings complementary information to the \raa, and its measurement can help disentangle some
effects. In the case of \PGUP{1S} mesons, a small $v_2$ (order of 1--2\%) is expected~\cite{Du:2017qkv,Yao:2018zrg,Bhaduri:2018iwr}, as can be seen in Fig.~\ref{fig:upsi_v2},
essentially because the ground state is formed early in the fireball evolution, at a time when 
developed momentum-space anisotropies are still small. For the same reason, the elliptic flow of \PGUP{2S} could be a factor 2 or more higher~\cite{Du:2017qkv,Bhaduri:2018iwr}, both from the regenerated and primordial
components. For both states, projections show that experimental precision may not be enough for a significant $v_2$ measurement, assuming $v_2$ values as in Ref.~\cite{Du:2017qkv}. For this reason,
combining results between the different LHC experiments would be beneficial to reach a better sensitivity.

% few sentences about spectra (instead of raa): measure them, more direct comparison to models, already measured now, etc
While we have focused on the \raa and $v_2$ in this section, bottomonium production can be studied using other observables. For instance, fully corrected yields or cross sections
in \PbPb can be studied, without making the ratio to a \pp measurement in a \raa. Such a measurement, already reported in some of the available experimental results~\cite{Sirunyan:2018nsz},
can directly be compared to a production model. 

% Also a figure on the prospects for the $B_c$ meson?

%\clearpage

%%%%%%%%%%%%%%%%%%%%%%%%%%%%%%%%%%%%%%%%%%%%%%%%%%%%%%%%%%%%%%%%
\subsection{Quarkonia in \pPb and \pp collisions}% (Main contributors: J.P. Lansberg (pPb), E. Ferreiro (pp) ...cannibalized by AA)}
%%%%%%%%%%%%%%%%%%%%%%%%%%%%%%%%%%%%%%%%%%%%%%%%%%%%%%%%%%%%%%%%
%[{\it NB: avoid duplicate with the "small systems" section}]

\subsubsection{\pPb collisions} % J.-P. Lansberg + AA
Quarkonium-production studies in high-energy \pPb collisions are usually carried out to measure how much specific nuclear effects, those which do {\it not} result from the creation of a deconfined state of matter, can alter the quarkonium yields. They should indeed be accounted for in the interpretation of \PbPb  results. They are also interesting on their own as they provide means to probe the modification of the gluon densities in the nuclei, the interaction between such pure heavy-quark bound states and light hadrons, or phenomena such as the coherent medium-induced energy loss of these quark-antiquark pairs.
The measurements as a function of event activity (violence of the collision, quantified via the charged-particle multiplicity or any centrality-related selection) brought several surprises, hotly discussed presently.
 
 Usually, a separation into initial-state and final-state effects is done (the energy loss effect can be seen as an interplay between the two types of effects). Yet, it is probably more instructive to separate out the effects which are believed to impact {\it all} the states of the charmonium or the bottomonium family with the {\it same} magnitude
 from those which are expected to affect much more the excited states. In principle, initial-state effects are of the first kind as the nature of the to-be produced quarkonium state is not yet fixed when the effects take place.  On the contrary, final-state effects can be sensitive to the properties of the produced quarkonium state and can thus be of the second kind.
 
However, in \pPb collisions at LHC energies, final-state interactions between the heavy-quark pair and the nuclear matter likely occur {\it before} the pair hadronises. This is due to the large boost between the nucleus and the pair -- and thus the quarkonium. At rest, a $c \bar c$ or $b\bar b$ pair takes 0.3--0.4\,fm/$c$ to hadronise; seen from the nucleus, at, for instance $y^{\rm lab}_{\rm pair}-y_{\rm beam} \sim 7$ , it takes $\gamma=\cosh(7) \simeq 500$ times longer. As such, final-state interactions with the nucleus likely do not discriminate ground and excited quarkonium states. Such an argument based on the existence of a large boost is nevertheless not applicable if one considers effects arising from the interactions between the pair and other particles {\it produced} by the \pPb collisions, not those contained in the Pb nucleus. The former are indeed not moving at the Pb projectile rapidity. In fact, some of these particles can have similar rapidities as the quarkonium and can thus be considered as comoving with it.
 
The simultaneous study of open-heavy flavoured hadrons along with both ground and excited quarkonium states can shed light on all these phenomena. Along the lines exposed above, one expects forward-quarkonium production in \pPb collisions (namely when the quarkonia flies in the direction of the proton) to be sensitive to low-$x$ phenomena like the gluon shadowing or saturation in the lead ion or to the coherent energy loss. On the contrary, the backward production should be sensitive to the gluon antishadowing. Moreover, the scatterings of quarkonia with comoving particles occur more often backward than forward, due to the rapidity-asymmetric particle multiplicities, and more often as well with the larger and less tightly bound excited states. 
   
With a wide rapidity coverage spanning from about $-5$ to 5, the LHC data from the 4 experiments are unique as they allow one to probe much smaller $x$ values than at RHIC and a with larger reach in $\pT$. The higher c.m.s. energy, the competitive luminosities and the resolution of the detectors also allow for more extensive studies of the bottomonium family. In fact, arguably the most striking observation, made already with Run 1 data, was that of a relative suppression in \pPb collisions of the excited \PGUP{2S},\PGUP{3S} states compared to that of the \PGUP{1S} observed by CMS~\cite{Chatrchyan:2013nza} as a function of the event activity (recently confirmed by ATLAS~\cite{Aaboud:2017cif}). Not only was it unexpected, but it constitutes a challenge to the conventional interpretation of suppression observed in \PbPb collisions~\cite{Chatrchyan:2012lxa,Sirunyan:2017lzi,Sirunyan:2018nsz}, which is of a significantly larger magnitude, but of a similar pattern. Such a relative suppression was also observed in the charmonium sector~\cite{Abelev:2014zpa}, where it is as well very remarkable.

As far as the suppression of the \PGUP{1S} and \PJgy is concerned, they seem to follow the expectations based on the RHIC results with a strong forward suppression compatible with a strong shadowing -- of a compatible magnitude of that observed with HF data~\cite{Kusina:2017gkz}. We should however note that the same observation can be explained by the coherent energy loss mechanism~\cite{Arleo:2010rb}. More data, including that on \PGU and Drell-Yan production, are clearly needed to disentangle both effects. More precision for \PGU and non-prompt \PJgy is in general critically needed as the typical experimental uncertainties are still on the order of the expected effects. As a case in point, backward $y$ data are not yet precise enough to quantify the magnitude of the gluon antishadowing, see section 11 for the relevance of quarkonium \pPb LHC data on nuclear PDF fits.

\begin{figure}[h]
 \begin{center}
%  \includegraphics[width=0.32\textwidth]{\main/quarkonia/fig/alice/alice_jpsi_v2_projected.pdf}
  \includegraphics[width=0.45\textwidth]{\main/quarkonia/fig/alice/alice_jpsi_v2_projected2.pdf}
 \end{center}
 \caption{The \pT dependence of the $v_2$ coefficient of \PJgy mesons in p--Pb collisions, for \unit[500]{nb}$^{-1}$ (ALICE). The projections are based on current ALICE data for 0--20\% centrality \cite{Acharya:2017tfn}  and are shown separately for negative and positive $y_{CM}$), assuming the same magnitude and are compared with transport model (TAMU) calculations \cite{Du:2018wsj} for midrapidity.}
\label{FigQ:v2pTpPb}
\end{figure}

Recently, the measurement of $v_2$ of \PJgy in \pPb collisions became available~\cite{Acharya:2017tfn,Sirunyan:2018kiz}, indicating a large -- and completely unexpected -- flow magnitude, $v_2 \lesssim 0.1$ up to $\pt \lesssim 8\UGeVc$.
%which may question the overall interpretation of the quarkonium $R^{\rm \pPb}$ measurements. More precise data are therefore crucial including additional correlations measurements. 
Recent transport model calculations \cite{Du:2018wsj}, which are successful in describing the features of the data, including the transverse momentum and centrality dependence of  \PJgy  and \PGyP{2S} production in \pPb, cannot reach the high value of the $v_2$ coefficient seen in data \cite{Acharya:2017tfn,Sirunyan:2018kiz} (see Fig.~\ref{FigQ:v2pTpPb}),
%, where the current 1-$\sigma$ reduction of the data implies a $v_2$ smaller by about 20\% for $\pT=5\UGeVc$), % FIXME what is meant here??
suggesting that the observed $v_2$ is in \pPb collisions mostly an initial state effect.
A precision measurement in Run 3, 4 for a broad rapidity range %(exploiting the 2 beam directions and the excellent capabilities of all detectors at the LHC)
will clarify this.
%We illustrate the expected quality of the measurement with the projections for the \pT dependence of the $v_2$ coefficient of \PJgy mesons in \pPb collisions in ALICE for \unit[500]{nb}$^{-1}$ (ALICE), shown in Fig.~\ref{FigQ:v2pTpPb}.

%More precise Run 3 \& 4 data can thus expediently advance such studies in 4 directions:  
% \PGU and non-\PJgy  $R_{\pPb}$ differential in $y$;
%\PJgy $R^{\pPb}$ at higher \pT\, double differential distribution in \pT\ \& $y$ as well as $v_2$ measurements; 
% $\psi(2S)$ and \PGUP{2},\PGUP{3} $R_{\rm \pPb}$ with possibly some $v_2$ analyses, then $R_{\rm \pPb}$ for new states including  the  $\chi_c(nP)$ and $\chi_b(nP)$ and possibly the $\eta_c$;
%new observables such as associated-production channels already studied in \pp collisions {\bf [I can add Refs if needed]}  .

In addition to conventional LHC collider data, one should not overlook the discriminating power of data which can be collected in the fixed-target mode~\cite{Brodsky:2012vg,Lansberg:2012kf}. Not only they correspond to completely different energy and (c.m.s.) rapidity ranges, but extremely competitive luminosities, up to a few fb$^{-1}$, are easily reachable. This is far beyond anything which can be reached in the collider mode during Run 3 \& 4. The LHCb collaboration has paved the way for a full fixed-target program at the LHC with their SMOG luminosity monitor~\cite{FerroLuzzi:2005em} used as an internal (He, Ne, Ar) gas target \cite{Aaij:2018ogq} (see also Section~\ref{sec:fixedtarget}). It is now clear that corresponding studies to those suggested above are possible~\cite{Hadjidakis:2018ifr} with the LHCb and ALICE detectors with light technical adjustments. They would drastically expand the scope of current proton-nucleus quarkonium studies.

\subsubsection{High-multiplicity \pp collisions} %E. Ferreiro + AA + JP Lansberg

Systematic studies of the quarkonium production in high-multiplicity \pp events can play an important role in understanding hadronization.
In particular, the correlation of the quarkonium yields with the charged-particle multiplicity can provide new insights into the interplay between hard and soft processes in particle production.
Hidden and open heavy-flavour production measurements as a function of the event activity were carried out  at the LHC during Run 1~\cite{Abelev:2012rz,Chatrchyan:2013nza}.
The striking feature of the data is that the production yields of quarkonia in high multiplicity events are significantly enhanced relative to minimum bias events, like for D mesons~\cite{Adam:2015ota}.
Specifically, the measurements of the self-normalized yields (the yield divided by the mean yield in minimum bias collisions) as a function of the self-normalized charged-particle multiplicity show an increase which is stronger than linear at the highest multiplicities.
The similarity between the D-meson and \PJgy results~\cite{Abelev:2012rz,Adam:2015ota} suggests that this behaviour is most likely related to the production processes, and that hadronization may only play a secondary role.
When comparing \PJgy preliminary results at $\sqrt{s}=13$\,TeV~\cite{Weber:2017hhm} to the ones previously obtained at $\sqrt{s} = 7$\,TeV~\cite{Abelev:2012rz}, no significant energy dependence is observed, {\it i.e.} the relative \PJgy~yields for events with identical relative multiplicities give similar results.

The data are described both by initial-state models as well as by a model assuming hydrodynamic evolution \cite{Werner:2013tya}, considering that the energy density reached in \pp collisions at LHC is high enough to apply such evolution.
Initial-state (saturation) effects are considered within
i) the Color-Glass-Condensate (CGC) framework~\cite{Ma:2018bax}; ii) the percolation approach~\cite{Ferreiro:2012fb,Ferreiro:2015gea}; iii) a model with higher Fock states~\cite{Kopeliovich:2013yfa}, based on parameters derived from \pPb collisions.
The energy dependence of the cross sections is controlled by the saturation momentum $Q_s(x)$ in the case of the CGC or density of colour ropes $\rho_s(y,p_T)$ in the percolation model, which also governs the charged-hadron multiplicity; events at different energies with the same $Q_s$ or $\rho_s$ are therefore identical.
For a given event multiplicity, they predict the relative yields to be almost energy independent.
It seems that, in any case, multiple interactions at the partonic level need to be taken into account in order to reproduce the data \cite{Sjostrand:2014zea,Skands:2014pea,Sjostrand:2017cdm}.

Runs 3, 4 data, reaching unprecedented high multiplicities because of larger data samples, and allowing for differential studies in \pT, will certainly help discriminate models.
For instance, in the percolation model, where colour interactions produce a reduction of the charged-particle multiplicities, the deviation from the linear behaviour is expected to be steeper for high-\pT quarkonia (and D mesons).
Moreover, measurements of \PJgy  yields relative to those of D mesons with the same transverse mass could help to elucidate the relative contribution of hadronisation and initial-state effects.

Studies of double differential ratios of excited-to-ground quarkonium states versus relative multiplicity could help clarify the presence of final-state effects, either QGP-like or the ones proposed by the comover model \cite{Ferreiro:2014bia,Ferreiro:2018wbd}.
Also, within the CGC+NRQCD framework \cite{Ma:2018bax}, the relative contributions of the 4 leading \PJgy Fock states have been calculated as a function of the event activity.
The contribution of the $^3S_1^{[8]}$ state to \PJgy production has been found to significantly increase with the event multiplicity. The growth of the contributions from the $^1S_0^{[8]}$  and $^3P_J^{[8]}$ channels is indeed relatively much smaller. The implication of such results in the understanding of the production mechanisms in minimum bias \pp collisions as well in  ${\rm e}^+{\rm e}^-$ and e+p collisions is yet to be investigated. 


\subsection*{Acknowledgement}
RR has been supported by the US National Science Foundation under
grant number PHY-1614484, and in part by the ExtreMe Matter Institute EMMI at 
the GSI Helmholtzzentrum f\"{u}r Schwerionenforschung (Darmstadt,Germany).

\end{document}
