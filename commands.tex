% Additional commands
% Please put yours in the appropriate section

%%% TODO - WILL BE REMOVED - PLEASE REPLACE %%%
\newcommand{\pt}{\pT}
\providecommand{\sqrtsnn}{\sqrtsNN}
\newcommand{\Raa}{\RAA}
\providecommand{\raa}{\ensuremath{R_\text{AA}}\Xspace}

%%%% GENERAL %%%%
\DeclareRobustCommand{\Pepem}{\HepParticle{\Pe}{}{+}\HepParticle{\Pe}{}{-}\Xspace} % positron-electron pair
\DeclareRobustCommand{\PGmpGmm}{\HepParticle{\PGm}{}{+}\HepParticle{\PGm}{}{-}\Xspace} % anti-muon-muon pair
\renewcommand{\pT}{\ensuremath{p\sb{\scriptstyle\mathrm{T}}}\Xspace}

\newcommand{\sqrts}{\ensuremath{\sqrt{s}}\Xspace}
\newcommand{\sqrtsNN}{\ensuremath{\sqrt{\sNN}}\Xspace}
\newcommand{\Npart}{\ensuremath{N_{\rm part}}\Xspace}
\newcommand{\Ncoll}{\ensuremath{N_{\rm coll}}\Xspace}
\newcommand{\RpPb}{\ensuremath{R_{\rm pPb}}\Xspace}
\newcommand{\RAA}{\ensuremath{R_{\rm AA}}\Xspace}
\newcommand{\RpA}{\ensuremath{R_{\rm pA}}\Xspace}
\newcommand{\TAA}{\ensuremath{T_{\rm AA}}\Xspace}
\newcommand{\RCP}{\ensuremath{R_{\rm CP}}\Xspace}
\newcommand{\vtwo}{\ensuremath{v_{\rm 2}}\Xspace}
\newcommand{\vone}{\ensuremath{v_{\rm 1}}\Xspace}
\newcommand{\vthree}{\ensuremath{v_{\rm 3}}\Xspace}
\newcommand{\vfour}{\ensuremath{v_{\rm 4}}\Xspace}
\newcommand{\vfive}{\ensuremath{v_{\rm 5}}\Xspace}
\newcommand{\vsix}{\ensuremath{v_{\rm 6}}\Xspace}
\newcommand{\vseven}{\ensuremath{v_{\rm 7}}\Xspace}
\newcommand{\vn}{\ensuremath{v_{\rm n}}\Xspace}

%%%Small systems
\newcommand{\nch}         {\ensuremath{N_{\mathrm {ch}}\xspace}}
\newcommand{\meannch}     {\ensuremath{\langle N_{\mathrm {ch}}\xspace \rangle}}
\newcommand{\meanpT}      {\ensuremath{\langle\pT\rangle}}
\newcommand{\dNdeta}      {\mathrm{d}N_\mathrm{ch}/\mathrm{d}\eta}
\newcommand{\dNdy}        {\mathrm{d}N_\mathrm{ch}/\mathrm{d}y}
\newcommand{\kT}          {\ensuremath{k\sb{\scriptstyle\mathrm{T}}}}
\newcommand{\ptt}         {\ensuremath{p_{\mathrm{T, trig}}}}
\newcommand{\pta}         {\ensuremath{p_{\mathrm{T, assoc}}}}

\newcommand{\pp}          {pp\Xspace}
\newcommand{\pPb}         {p--Pb\Xspace}
\newcommand{\Pbp}         {Pb--p\Xspace}
\newcommand{\pO}          {p--O\Xspace}
\newcommand{\pA}          {p--A\Xspace}
\newcommand{\AOnA}        {A--A\Xspace}
\newcommand{\PbPb}        {Pb--Pb\Xspace}
\newcommand{\XeXe}        {Xe--Xe\Xspace}
\newcommand{\AuAu}        {Au--Au\Xspace}
\newcommand{\CuCu}        {Cu--Cu\Xspace}
\newcommand{\pAu}         {p--Au\Xspace}
\newcommand{\dAu}         {d--Au\Xspace}

\newcommand{\sigmaBFPP}   {\ensuremath{\sigma_\mathrm{BFPP}}\Xspace}

\newcommand{\lsim}        {\,{\buildrel < \over {_\sim}}\,}
\newcommand{\gsim}        {\,{\buildrel > \over {_\sim}}\,}
\newcommand{\co}[1]       {\relax}
\newcommand{\nl}          {\newline}
\newcommand{\el}          {\\\hline\\[-0.4cm]}

%%% LIGHT FLAVOURS %%%%
%nuclei
\newcommand{\gmom}{\ensuremath{\mathrm{GeV}\kern-0.05em/\kern-0.02em c}}
\newcommand{\antid}{$\overline{\mathrm{d}}$}
\newcommand{\tritium}{\ensuremath{{}^{3}H}}
\newcommand{\antitritium}{\ensuremath{{}^{3}\overline{\mathrm{\mathrm{He}}}}}
\newcommand{\hethree}{\ensuremath{{}^{3}\mathrm{He}}}
\newcommand{\hefour}{\ensuremath{{}^{4}\mathrm{He}}}
\newcommand{\antihethree}{\ensuremath{{}^{3}\overline{\mathrm{He}}}}
\newcommand{\antihefour}{\ensuremath{${}^{4}$\overline{\mathrm{He}}}}
%hypernuclei
\newcommand{\hyp}        {$^{3}_{\Lambda}\mathrm H$}
\newcommand{\antihyp}    {$^{3}_{\bar{\Lambda}} \overline{\mathrm H}$}
\newcommand{\hypfour}    {$^{4}_{\Lambda}\mathrm H$}
\newcommand{\antihypfour}{$^{4}_{\bar{\Lambda}} \overline{\mathrm H}$}
\newcommand{\hyphefour}    {$^{4}_{\Lambda}\mathrm{He}$}
\newcommand{\antihehypfour}{$^{4}_{\bar{\Lambda}} \overline{\mathrm{He}}$}
\newcommand{\sla}{\slash \hspace{-0.2cm}}
\newcommand{\slam}{\slash \hspace{-0.25cm}}
\newcommand{\no}{\nonumber}
\def\lsim{\mathrel{\rlap{\lower4pt\hbox{\hskip1pt$\sim$}}
    \raise1pt\hbox{$<$}}}         %less than or approx. symbol
\def\gsim{\mathrel{\rlap{\lower4pt\hbox{\hskip1pt$\sim$}}
    \raise1pt\hbox{$>$}}}         %greater than or approx. symbol
%fluctuations
\def\Bs{{\overline{B}}_s}
\def\R{\mathcal{R}}
\newcommand{\gev}{\mathrm{GeV}}
\newcommand{\tev}{\mathrm{TeV}}
\newcommand{\mev}{\mathrm{MeV}}
\newcommand{\e}{\epsilon}
\newcommand{\tce}{\frac{t_{\rm cool}(\e)}{t_{\rm esc}(\e)}}
\newcommand{\tcer}{\frac{t_{\rm c}(\R)}{t_{\rm esc}(\R)}}
\def\Xe{X_{\rm esc}}
\def\X{X_{\rm esc}}
\def\te{t_{\rm esc}}
\def\tc{t_{\rm cool}}
\def\nb{n_{\rm B}}
\def\nc{n_{\rm C}}
\def\ni{n_{i}}
\def\rism{\rho_{\rm ISM}}
\def\nism{n_{\rm ISM}}
\def\x{(\R,\vec r,t)}
\def\xo{(\R,\vec r_\odot,t_\odot)}
\def\ap{\overline{\rm p}}
\def\ad{\overline{\rm d}}
\def\ep{e^+}
\def\Qep{Q_{e^+}}
\def\epm{$e^\pm$\ }
\def\ah{\overline{\rm ^3He}}
\def\at{\overline{\rm t}}
\def\s{$(*)$}
\newcommand{\dd}{\text{d}}
\newcommand{\gaga}{\gamma\gamma}
\newcommand{\Rp}{\mathcal{R}^\prime}
\newcommand{\Lp}{L^{\prime}}
%%% end of LIGHT FLAVOURS %%%%

%%% QUARKONIA
%%% end of QUARKONIA


%% ACCELERATOR
%%\newcommand{\text}[1]{\mathrm{#1}}

% quantity as number and unitbbb

\newcommand{\qty}[2]{\ensuremath{#1\,\mathrm{#2}}}  %% quantity as number and unit
\newcommand{\enum}[2]{\ensuremath{#1\times10^{#2}}} % a number in scientific notation
\newcommand{\NQTY}[2]{\mbox{$[#1/{\rm #2}]$}}     % numerical formula e.g. \NQTY{E}{GeV} gives [E/GeV]
\newcommand{\UQTY}[2]{\ensuremath{#1/\mathrm{#2}}}  % numerical formula e.g. \UQTY{E}{GeV} gives E/GeV
\newcommand{\eqty}[3]{\qty{\enum{#1}{#2}}{#3}}  % for scientific notation plus units
\newcommand{\invnb}[1]{\qty{#1}{nb^{-1}}}           %inverse nanobarn

\newcommand{\elumi}[2]{\qty{\enum{#1}{#2}}{cm^{-2}s^{-1}}}
\newcommand{\murad}[1]{\qty{#1}{\mu rad}}
\newcommand{\intlumimub}[1]{\qty{#1}{\mu b^{-1}}}

\newcommand{\sNN}{\ensuremath{s_{\mbox{\tiny NN}}}}
\newcommand{\yNN}{\ensuremath{y_{\mbox{\tiny NN}}}}
\newcommand{\bstar}{\ensuremath{\beta^{*}}}
\newcommand{\emittn}{\ensuremath{\varepsilon_n}}
\newcommand{\LAA}{\ensuremath{L_\text{AA}}}
\newcommand{\LpA}{\ensuremath{L_{pA}}}
\newcommand{\Lpp}{\ensuremath{L_{pp}}}
\newcommand{\Lpeak}{\ensuremath{\hat{L}}}
\newcommand{\LNN}{\ensuremath{ L_{\text{NN}}}}

\newcommand{\isotope}[3]{\ensuremath{^{#1}\mathrm{#2}^{#3}}}
%%\newcommand{\Pb}{\isotope{208}{Pb}{82+}}

% This is the header for the tables of ion parameters
\newcommand{\speciesheader}{ &
\isotope{16}{O}{8+}&
\isotope{40}{Ar}{18+}&
\isotope{40}{Ca}{20+}&
\isotope{78}{Kr}{36+}&
\isotope{129}{Xe}{54+}&
\isotope{208}{Pb}{82+}
}


\newcommand{\bfunc}{$\beta$-function}
\newcommand{\bstarval}[1]{$\bstar = #1\,\mbox{m}$}
\newcommand{\betarel}{\ensuremath{\beta_\text{rel}}}
\newcommand{\emittnx}{\ensuremath{\epsilon_{n,x}}}
\newcommand{\emittny}{\ensuremath{\epsilon_{n,y}}}
\newcommand{\emittnxy}{\ensuremath{\epsilon_{n,xy}}}
%\newcommand{\emittl}{\ensuremath{\epsilon_l}}
\newcommand{\emitts}{\ensuremath{\epsilon_s}}
\newcommand{\sigs}{\ensuremath{\sigma_s}}
\newcommand{\sigp}{\ensuremath{\sigma_p}}
\newcommand{\kb}{\ensuremath{k_b}}
\newcommand{\frev}{\ensuremath{f_0}}
\newcommand{\Nb}{\ensuremath{N_b}}
\newcommand{\Eb}{\ensuremath{E_b}}
\newcommand{\emittval}[1]{\ensuremath{\emittn=\qty{#1}{\mu m\,rad}}}
\newcommand{\Nbval}[2]{\ensuremath{\Nb=\enum{#1}{#2}}}
%\newcommand{\sigz}{\ensuremath{\sigma_z}}
\newcommand{\taul}{\ensuremath{\tau_l}}
\newcommand{\taulval}[1]{\ensuremath{\taul=\qty{#1}{ns}}}
\newcommand{\sigzval}[1]{\ensuremath{\sigz=\qty{#1}{cm}}}
\newcommand{\etev}[1]{\ensuremath{\Eb=\qty{#1}{TeV}}}
\newcommand{\VRF}{\ensuremath{V_{\mathrm{RF}}}}
\newcommand{\lumival}[2]{\ensuremath{\mathcal{L}=\qty{#1\times 10^{#2}}{cm^{-2} s^{-1}}}}

\newcommand{\aibsx}{\ensuremath{\alpha_{\mathrm{IBS},x}}}
\newcommand{\aibsy}{\ensuremath{\alpha_{\mathrm{IBS},y}}}
\newcommand{\aibsxy}{\ensuremath{\alpha_{\mathrm{IBS},x,y}}}
\newcommand{\aradd}{\ensuremath{\alpha_{\mathrm{rad}}}}
\newcommand{\aradds}{\ensuremath{\alpha_{\mathrm{rad},s}}}
\newcommand{\araddx}{\ensuremath{\alpha_{\mathrm{rad},x}}}
\newcommand{\araddy}{\ensuremath{\alpha_{\mathrm{rad},y}}}
\newcommand{\araddxy}{\ensuremath{\alpha_{\mathrm{rad},x,y}}}
\newcommand{\Z}{\ensuremath{Z_\text{ion}}}
\newcommand{\A}{\ensuremath{A_\text{ion}}}
\newcommand{\Circ}{\ensuremath{C_\text{ring}}}
\newcommand{\lumi}{\ensuremath{\mathcal{L}}}
\newcommand{\Lb}{\ensuremath{\mathcal{L}_b}}
\newcommand{\Lint}{\ensuremath{\mathcal{L}_{\text{int}}}}
\newcommand{\Lbint}{\ensuremath{\mathcal{L}_{b,\text{int}}}}
\newcommand{\Lbpeak}{\ensuremath{\mathcal{L}_{b,\text{peak}}}}

%% end of ACCELERATOR
