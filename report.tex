\documentclass[11pt,twoside,a4paper]{cernrep} 
\usepackage{rep_common}
\usepackage{units}
\usepackage{pdflscape}
\usepackage{afterpage}
\usepackage{lineno}
\usepackage{diagbox}
\usepackage{hyperref}
\usepackage{placeins}
\usepackage[utf8]{inputenc}

\pagestyle{plain}
% --
% -- The bibliography file paths for the main volume and all parts
% --
\def\bibfiles{\main/bib/chapter,\main/introduction/bib/section,\main/schedule/bib/section,\main/hf/bib/section,\main/quarkonia/bib/section,\main/thermalradiation/bib/section,\main/flow/bib/section,\main/smallsystems/bib/section,\main/lightflavour/bib/section,\main/jets/bib/section,\main/smallx/bib/section,\main/beyond/bib/section,\main/helhc/bib/section,\main/accelerator/bib/section,\main/smallAexec/bib/section}
% --
\providecommand{\biblio}{\nocite{article-minimal}\bibliographystyle{report}\clearpage\bibliography{\bibfiles}}  % *Modification: added `\main/` to specify relative file location.
 
\linenumbers

\begin{document}
\newcommand{\main}{.}
%\def\biblio{}

%% commands
%file for shortcuts

\newcommand{\nch}{\ensuremath{N_{\mathrm {ch}}\xspace}}
\newcommand{\Ncoll}{\ensuremath{N_{\mathrm {coll}}}}
\newcommand{\Npart}{\ensuremath{N_{\mathrm {part}}}}
\newcommand{\dNdeta}{\mathrm{d}N_\mathrm{ch}/\mathrm{d}\eta}
\newcommand{\snn}         {\ensuremath{\sqrt{s_{\mathrm {NN}}}}}
\newcommand{\kT}          {\ensuremath{k_{\mathrm {T}}}}

\newcommand{\pp}          {pp}
\newcommand{\pPb}         {pPb}
\newcommand{\pA}          {pA}
\newcommand{\PbPb}        {PbPb}
\newcommand{\AuAu}        {AuAu}
\newcommand{\CuCu}        {CuCu}
\newcommand{\pAu}         {pAu}
\newcommand{\dAu}         {dAu}
\newcommand{\lsim}        {\,{\buildrel < \over {_\sim}}\,}
\newcommand{\gsim}        {\,{\buildrel > \over {_\sim}}\,}
\newcommand{\co}[1]       {\relax}
\newcommand{\nl}          {\newline}
\newcommand{\el}          {\\\hline\\[-0.4cm]}

\title{Future physics opportunities for high-density QCD with ion and proton beams at the LHC \\
HL/HE-LHC Physics Workshop Report Working Group 5}
\author{
Z.~Citron$^{5}$,
A.~Dainese$^{28}$,
J.F.~Grosse-Oetringhaus$^{8}$,
J.~Jowett$^{8}$,
Y.L.~Lee$^{51}$,
U.A.~Wiedemann$^{8}$,
M.~Winn$^{41}$ (editors), \\ 
A.~Andronic$^{50}$,
F.~Bellini$^{8}$,
E.~Bruna$^{29}$,
E.~Chapon$^{8}$,
H.~Dembinski$^{49}$,
D.~d'Enterria$^{8}$,
I.~Grabowska-Bold$^{1}$,
G.M.~Innocenti$^{8,51}$,
C.~Loizides$^{58}$,
S.~Mohapatra$^{13}$,
C.A.~Salgado$^{92}$,
M.~Verweij$^{65,96}$,
M.~Weber$^{4,71}$ (chapter coordinators), \\
J.~Aichelin$^{69}$,
A.~Angerami$^{43}$,
L.~Apolinario$^{33,42}$,
F.~Arleo$^{44}$,
N.~Armesto$^{78}$,
M.~Arslandok$^{24}$,
R.~Bailhache$^{23}$,
S.A.~Bass$^{17}$,
C.~Bedda$^{94}$,
R.~Bellwied$^{26}$,
A.~Beraudo$^{29}$,
R.~Bi$^{51}$,
C.~Bierlich$^{88}$,
K.~Blum$^{8,98}$,
A.~Borissov$^{50}$,
P.~Braun-Munzinger$^{18}$,
R.~Bruce$^{8}$,
G.E.~Bruno$^{63}$,
J.~Castillo Castellanos$^{7}$,
R.~Chatterjee$^{95}$,
Y.~Chen$^{8}$,
Z.~Chen$^{66}$,
C.~Cheshkov$^{31}$,
L.~Cunqueiro~Mendez$^{57}$,
T.~Dahms$^{19}$,
A.F.~Dobrin$^{8}$,
B.~Doenigus$^{21}$,
L.~Van Doremalen$^{94}$,
X.~Du$^{75}$,
A.~Dubla$^{22}$,
M.~Dumancic$^{98}$,
M.~Dyndal$^{16}$,
L.~Fabbietti$^{73}$,
E.~Ferreiro$^{70}$,
F.~Fionda$^{86}$,
S.~Floerchinger$^{24}$,
J.G.~Contreras~Nuno$^{15}$,
G.~Giacalone$^{32}$,
P.B.~Gossiaux$^{69}$,
V.~Greco$^{83}$,
F.~Grosa$^{84}$,
M.~Guilbaud$^{8}$,
T.~Gunji$^{10}$,
V.~Guzey$^{25,40,90}$,
S.~Hassani$^{81}$,
M.~He$^{54,74}$,
I.~Helenius$^{80}$,
P.~Huo$^{72}$,
P.M.~Jacobs$^{47}$,
P.~Janus$^{2,74}$,
M.A.~Jebramcik$^{8,36}$,
J.~Jia$^{6,72}$,
N.~K. Behera$^{34}$,
A.P.~Kalweit$^{8}$,
H.~Kim$^{12}$,
M.~Klasen$^{50}$,
S.~Klein$^{47}$,
M.~Klusek-Gawenda$^{27}$,
J.~Kremer$^{2,74}$,
G.~Krintiras$^{82}$,
F.~Krizek$^{3}$,
E.~Kryshen$^{40}$,
A.~Kurkela$^{8}$,
A.~Kusina$^{62}$,
J.P.~Lansberg$^{35}$,
R.~Lea$^{85}$,
M.~van~Leeuwen$^{56,94}$,
W.~Li$^{66}$,
J.~Margutti$^{94}$,
A.~Marin$^{22}$,
C.~Marquet$^{9}$,
J.~Martin Blanco$^{46}$,
A.~Mastroserio$^{20}$,
C.~Mayer$^{62}$,
C.~Mcginn$^{51}$,
G.~Milhano$^{8,33,42}$,
A.~Milov$^{98}$,
A.~Mischke$^{94}$,
N.~Mohammadi$^{8}$,
M.~Mulders$^{8}$,
M.~Murray$^{91}$,
J.~Noronha-Hostler$^{67}$,
A.~Ohlson$^{24}$,
V.~Okorokov$^{53}$,
F.~Olness$^{68}$,
P.~Paakkinen$^{90}$,
J.~Park$^{39}$,
H.~Paukkunen$^{25,90}$,
D.~Perepelitsa$^{87}$,
D.~Peresunko$^{55}$,
M.~Peters$^{51}$,
C.C.~Peng$^{64}$,
S.~Piano$^{30}$,
T.~Pierog$^{37}$,
M.~Ploskon$^{47}$,
S.~Plumari$^{83}$,
F.~Prino$^{29}$,
M.~Puccio$^{84}$,
R.~Rapp$^{75}$,
K.~Redlich$^{18,99}$,
K.~Reygers$^{24}$,
A.~Rossi$^{60}$,
A.~Rustamov$^{22,24,52}$,
M.~Rybar$^{13}$,
M.~Schaumann$^{8}$,
B.~Schenke$^{6}$,
I.~Schienbein$^{45}$,
L.~Schoeffel$^{81}$, 
A.M.~Sickles$^{77}$,
M.~Sievert$^{67}$,
P.~Silva$^{8}$,
T.~Song$^{89}$,
M.~Spousta$^{11}$,
J.~Stachel$^{24}$,
P.~Steinberg$^{6}$,
D.~Stocco$^{69}$,
M.~Strickland$^{38}$,
M.~Strikman$^{61}$,
J.~Sun$^{76}$,
D.~Tapia~Takaki$^{91}$,
K.~Tatar$^{51}$,
C.~Terrevoli$^{60}$,
S.~Trogolo$^{84}$,
B.~Trzeciak$^{94}$,
A.~Trzupek$^{14}$,
R.~Ulrich$^{37}$,
A.~Uras$^{93}$,
R.~Venugopalan$^{6}$,
I.~Vitev$^{48}$,
G.~Vujanovic$^{59,97}$,
J.~Wang$^{51}$,
T.W.~Wang$^{51}$,
R.~Xiao$^{64}$,
Y.~Xu$^{17}$,
H.~Zanoli$^{79}$,
M.~Zhou$^{72}$
}
\institute{
\footnotesize
$^{1}$ AGH UST, Krak\'ow,
$^{2}$ AGH University of Science,
$^{3}$ Academy of Sciences, Prague,
$^{4}$ Austrian Academy of Sciences,
$^{5}$ Ben-Gurion University of the Negev,
$^{6}$ Brookhaven National Lab,
$^{7}$ CEA Saclay,
$^{8}$ CERN,
$^{9}$ CPT/{\'E}cole Polytechnique,
$^{10}$ Center for Nuclear Study, Graduate School of Science, the University of Tokyo,
$^{11}$ Charles University,
$^{12}$ Chonnam National University,
$^{13}$ Columbia University,
$^{14}$ Cracow, INP,
$^{15}$ Czech Technical University in Prague,
$^{16}$ DESY,
$^{17}$ Duke University,
$^{18}$ EMMI$/$GSI,
$^{19}$ Excellence Cluster Universe, Technical University Munich,
$^{20}$ Foggia University and INFN Bari,
$^{21}$ Frankfurt University,
$^{22}$ GSI,
$^{23}$ Goethe-University Frankfurt,
$^{24}$ Heidelberg University,
$^{25}$ Helsinki University,
$^{26}$ Houston University,
$^{27}$ IFJ PAN, PL-31342 Krak\'ow, Poland,
$^{28}$ INFN Padova,
$^{29}$ INFN Torino,
$^{30}$ INFN Trieste,
$^{31}$ IPN Lyon,
$^{32}$ IPT CNRS/CEA,
$^{33}$ IST Lisbon,
$^{34}$ Inha University,
$^{35}$ Institut the Physique Nucl\'{e}aire d'Orsay,
$^{36}$ Johann-Wolfgang-Goethe Universit\"{a}t, Frankfurt,
$^{37}$ Karlsruhe Institute of Technology,
$^{38}$ Kent State University,
$^{39}$ Korea University,
$^{40}$ Kurchatov institute, Gatchina,
$^{41}$ LAL, now DPhN, CEA/IRFU,
$^{42}$ LIP,
$^{43}$ LLNL Livermore,
$^{44}$ LLR/{\'E}cole Polytechnique,
$^{45}$ LPSC/Universit{\'e} Grenoble,
$^{46}$ Laboratoire Leprince Ringuet,
$^{47}$ Lawrence Berkeley National Laboratory,
$^{48}$ Los Alamos National Laboratory,
$^{49}$ MPI for Nuclear Physics, Heidelberg,
$^{50}$ M\"{u}nster University,
$^{51}$ Massachusetts Institute of Technology,
$^{52}$ NNRC Baku,
$^{53}$ NRNU MEPhI, Moscow,
$^{54}$ Nanjing University of Science,
$^{55}$ National Research Centre Kurchatov Institute, Moscow,
$^{56}$ Nikhef,
$^{57}$ ORNL,
$^{58}$ Oak Ridge National Laboratory,
$^{59}$ Ohio State University,
$^{60}$ Padova University and INFN Padova,
$^{61}$ Pennsylvania State University,
$^{62}$ Polish Academy of Sciences, Cracow,
$^{63}$ Politecnico di Bari and INFN Bari,
$^{64}$ Purdue University,
$^{65}$ RIKEN BNL Research Center,
$^{66}$ Rice University,
$^{67}$ Rutgers University,
$^{68}$ SMU Dallas,
$^{69}$ SUBATECH, Universit\'e de Nantes,
$^{70}$ Santiago de Compostela University,
$^{71}$ Stefan Meyer Institute Vienna,
$^{72}$ Stony Brook University,
$^{73}$ TU Munich,
$^{74}$ Technology,
$^{75}$ Texas A\&M University,
$^{76}$ Tsinghua University, Beijing,
$^{77}$ U. Illinois, Urbana-Champaign,
$^{78}$ Universidade de Santiago de Compostela,
$^{79}$ Universidade de Sao Paulo,
$^{80}$ Universit\"{a}t T\"{u}bingen,
$^{81}$ Universit\'e Paris-Saclay,
$^{82}$ Universit\'{e} Louvain,
$^{83}$ Universit\`a di Catania and INFN Catania,
$^{84}$ University and INFN Torino,
$^{85}$ University and INFN Trieste,
$^{86}$ University of Bergen,
$^{87}$ University of Colorado Boulder,
$^{88}$ University of Copenhagen,
$^{89}$ University of Gie{\ss}en,
$^{90}$ University of Jyvaskyla,
$^{91}$ University of Kansas,
$^{92}$ University of Santiago de Compostela,
$^{93}$ Universit{\'e} de Lyon, CNRS/IN2P3, IPN-Lyon,
$^{94}$ Utrecht University,
$^{95}$ VECC Calcutta,
$^{96}$ Vanderbilt University,
$^{97}$ Wayne State University,
$^{98}$ Weizmann Institute of Science,
$^{99}$ Wroclaw University
}


\maketitle

\begin{abstract}
 The future opportunities for high-density QCD with ions and proton beams at the LHC are presented. Four major scientific goals are identified: the characterisation of the macroscopic long wavelength Quark-Gluon Plasma~(QGP) properties with unprecedented precision, the investigation of the microscopic parton dynamics underlying QGP properties, the development of a unified picture of particle production and QCD dynamics from small (\pp) to large (nucleus--nucleus) systems, the exploration of parton densities in nuclei in a broad ($x$, $Q^2$) kinematic range and the search for the possible onset of parton saturation. In order to address these scientific goals, high-luminosity \PbPb and \pPb programmes are considered as priorities for Runs~3 and~4, complemented by high-multiplicity studies in pp collisions and a  short run with Oxygen ions. High-luminosity runs with intermediate-mass nuclei, for example Ar or Kr, are considered as an appealing case for extending the heavy-ion programme at the LHC beyond Run~4. The potential of the High-Energy LHC to probe QCD matter with newly available observables and phenomena is investigated.
\end{abstract}

% -- Set the level of the TOC: 2 means including subsection, 1 means
% including section 
\setcounter{tocdepth}{2}
{ 
\baselineskip=12pt
\tableofcontents
}
\clearpage

% -- List of sections
\subfile{\main/introduction/section}
\clearpage
\subfile{\main/accelerator/section}
\clearpage
\subfile{\main/lightflavour/section}
\clearpage
\subfile{\main/flow/section}
\clearpage
\subfile{\main/hf/section}
\clearpage
\subfile{\main/jets/section}
\clearpage
\subfile{\main/quarkonia/section}
\clearpage
\subfile{\main/thermalradiation/section}
\clearpage
\subfile{\main/smallsystems/section}
\clearpage
\subfile{\main/smallx/section}
\clearpage
\subfile{\main/beyond/section}
\clearpage
\subfile{\main/schedule/section}
\clearpage
\subfile{\main/helhc/section}
\clearpage
\section*{Acknowledgements}

% HOWTO: Please insert your acknowledgements of funding agencies etc. here. Please add a comment from which chapter they come. We will merge them to a nice text later.

% Small systems
Pinco Pallino acknowledges the support of Bike to CERN.

%Light flavour
N. K. Behera acknowledges the National Research Foundation of Korea (NRF), Basic Science Research Program, funded by the Ministry of Education, Science and Technology (Grant No. NRF-2014R1A1A1008246).
S. Floerchinger acknowledges the support by the DFG Collaborative Research Centre SFB 1225 (ISOQUANT). 
K. Redlich acknowledges the Polish National Science Center NCN under Maestro grant DEC-$\mathrm{2013/10/A/ST2/00106}$. 

%Thermal radiation
Michael Weber: Austrian Academy of Sciences and Nationalstiftung f\"ur Forschung, Technologie und Entwicklung, Austria
Torsten Dahms: DFG cluster of excellence ``Origin and Structure of the Universe'' 
Raphaelle Bailhache: The here conducted work has been supported by the German Federal Ministry of Education and Research (BMBF)
Spencer Klein: This work was funded by the U.S. DOE under contract number DE-AC02- 05-CH11231
Dmitri Peresunko: "the Russian Science Foundation grant 17-72-20234"
Ralf Rapp: (The research of) RR has been supported by the U.S. National Science Foundation under
grant no. PHY-1614484.


% -- Add bibliography to table of contents
%\addcontentsline{toc}{chapter}{References}

% dummy reference to avoid that bibtex fails
% -- Add volume bibliography and part specific bibliographies
%\bibliographystyle{report}
%\bibliography{\bibfiles}
\footnotesize
\biblio

\end{document}
