\documentclass[11pt,twoside,a4paper]{cernrep} 
\usepackage{rep_common}
\usepackage{units}
\usepackage{pdflscape}
\usepackage{afterpage}
\usepackage{lineno}
\usepackage{diagbox}
\usepackage{hyperref}
\usepackage{placeins}
\usepackage[utf8]{inputenc}

\pagestyle{plain}
% --
% -- The bibliography file paths for the main volume and all parts
% --
\def\bibfiles{\main/bib/chapter,\main/introduction/bib/section,\main/schedule/bib/section,\main/hf/bib/section,\main/quarkonia/bib/section,\main/thermalradiation/bib/section,\main/flow/bib/section,\main/smallsystems/bib/section,\main/lightflavour/bib/section,\main/jets/bib/section,\main/smallx/bib/section,\main/beyond/bib/section,\main/helhc/bib/section,\main/accelerator/bib/section,\main/smallAexec/bib/section}
% --
\providecommand{\biblio}{\nocite{article-minimal}\bibliographystyle{report}\clearpage\bibliography{\bibfiles}}  % *Modification: added `\main/` to specify relative file location.

\linenumbers

\begin{document}
\newcommand{\main}{.}
%\def\biblio{}

%% commands
%file for shortcuts

\newcommand{\nch}{\ensuremath{N_{\mathrm {ch}}\xspace}}
\newcommand{\dNdeta}{\mathrm{d}N_\mathrm{ch}/\mathrm{d}\eta}


\title{Future physics opportunities for high-density QCD with ions and proton beams at LHC \\
HL/HE-LHC Physics Workshop Report Working Group 5}
\author{Conveners: E. Bruna$^{1}$, G.M. Innocenti$^{2,3}$ \\
Contributors: J. Aichelin$^{17}$, S. A. Bass$^{15}$, C. Bedda$^{6}$, A. Beraudo$^{1}$, G.E. Bruno$^{7}$, Z. Citron$^{13}$, A. Dainese$^{8}$, A. Dubla$^{9}$, F. Fionda$^{10}$, P.-B. Gossiaux$^{17}$, V. Greco$^{14}$, F. Grosa$^{11}$, Y.-J. Lee$^{3}$, J. Margutti$^{6}$, A. Mischke$^{6}$, M. Peters$^{3}$, S. Plumari$^{14}$, F. Prino$^{1}$, A. Rossi$^{8}$, J. Sun$^{4}$,C. Terrevoli$^{8}$, B. Trzeciak$^{6}$, A. Uras$^{12}$, L. Van Doremalen$^{6}$, I. Vitev$^{16}$, J. Wang$^{3}$, T.-W. Wang$^{3}$, M. Winn$^{5}$,Y. Xu$^{15}$}
\institute{
$^{1}$ Universit\`a degli Studi di Torino and INFN, Italy \\
$^{2}$ European Organization for Nuclear Research (CERN), Geneva, Switzerland \\
$^{3}$ Massachusetts Institute of Technology (MIT), Cambridge, USA \\
$^{4}$ Tsinghua University, Beijing, China \\
$^{5}$ Universit\'e Paris-Saclay, Paris, France \\
$^{6}$ Utrecht University, Utrecht, the Netherlands \\
$^{7}$ Politecnico di Bari and INFN, Bary, Italy \\
$^{8}$ Universit\`a degli Studi di Padova and INFN, Padova, Italy \\
$^{9}$ GSI Helmholzzentrum f\"{u}r Schwerionenforschung, Darmstadt, Germany \\
$^{10}$ University of Bergen, Bergen, Norway \\
$^{11}$ Politecnico di Torino and INFN, Torino, Italy \\
$^{12}$ IPN-Lyon, Villeurbanne, France \\
$^{13}$ Ben-Gurion University of the Negev, Beersheba, Israel \\
$^{14}$ Universit\`a di Catania and INFN, Catania, Italy \\
$^{15}$ Duke University, Durham, USA \\
$^{16}$ Los Alamos National Laboratory, Los Alamos, USA \\
$^{17}$ SUBATECH, Universit\'e de Nantes, Nantes, France \\
}




\maketitle

\begin{abstract}
 The future opportunities for high-density QCD with ions and proton beams at the LHC are presented. Four major scientific goals are identified: the characterisation of the macroscopic long wavelength Quark-Gluon Plasma~(QGP) properties with unprecedented precision, the investigation of the microscopic parton dynamics underlying QGP properties, the development of a unified picture of particle production and QCD dynamics from small (\pp) to large (nucleus--nucleus) systems, the exploration of parton densities in nuclei in a broad ($x$, $Q^2$) kinematic range and the search for the possible onset of parton saturation. In order to address these scientific goals, high-luminosity \PbPb and \pPb programmes are considered as priorities for Runs~3 and~4, complemented by high-multiplicity studies in pp collisions and a  short run with Oxygen ions. High-luminosity runs with intermediate-mass nuclei, for example Ar or Kr, are considered as an appealing case for extending the heavy-ion programme at the LHC beyond Run~4. The potential of the High-Energy LHC to probe QCD matter with newly available observables and phenomena is investigated.
\end{abstract}

% -- Set the level of the TOC: 2 means including subsection, 1 means
% including section 
\setcounter{tocdepth}{2}
{ 
\baselineskip=12pt
\tableofcontents
}
\clearpage

% -- List of sections
\subfile{\main/introduction/section}
\clearpage
\subfile{\main/accelerator/section}
\clearpage
\subfile{\main/lightflavour/section}
\clearpage
\subfile{\main/flow/section}
\clearpage
\subfile{\main/hf/section}
\clearpage
\subfile{\main/jets/section}
\clearpage
\subfile{\main/quarkonia/section}
\clearpage
\subfile{\main/thermalradiation/section}
\clearpage
\subfile{\main/smallsystems/section}
\clearpage
\subfile{\main/smallx/section}
\clearpage
\subfile{\main/beyond/section}
\clearpage
\subfile{\main/schedule/section}
\clearpage
\subfile{\main/helhc/section}
\clearpage
\section*{Acknowledgements}

% HOWTO: Please insert your acknowledgements of funding agencies etc. here. Please add a comment from which chapter they come. We will merge them to a nice text later.

% Small systems
ABC would like to acknowledge the support of Bike to CERN.


% -- Add bibliography to table of contents
%\addcontentsline{toc}{chapter}{References}

% dummy reference to avoid that bibtex fails
% -- Add volume bibliography and part specific bibliographies
%\bibliographystyle{report}
%\bibliography{\bibfiles}
\footnotesize
\biblio

\end{document}
