% don't remove the following lines, and edit the definition of \main if needed
\documentclass[../report.tex]{subfiles}
\providecommand{\main}{..}
\IfEq{\jobname}{\currfilebase}{\AtEndDocument{\biblio}}{}
\IfEq{\jobname}{\currfilebase}{%file for shortcuts

\newcommand{\nch}{\ensuremath{N_{\mathrm {ch}}\xspace}}
\newcommand{\Ncoll}{\ensuremath{N_{\mathrm {coll}}}}
\newcommand{\Npart}{\ensuremath{N_{\mathrm {part}}}}
\newcommand{\dNdeta}{\mathrm{d}N_\mathrm{ch}/\mathrm{d}\eta}
\newcommand{\snn}         {\ensuremath{\sqrt{s_{\mathrm {NN}}}}}
\newcommand{\kT}          {\ensuremath{k_{\mathrm {T}}}}

\newcommand{\pp}          {pp}
\newcommand{\pPb}         {pPb}
\newcommand{\pA}          {pA}
\newcommand{\PbPb}        {PbPb}
\newcommand{\AuAu}        {AuAu}
\newcommand{\CuCu}        {CuCu}
\newcommand{\pAu}         {pAu}
\newcommand{\dAu}         {dAu}
\newcommand{\lsim}        {\,{\buildrel < \over {_\sim}}\,}
\newcommand{\gsim}        {\,{\buildrel > \over {_\sim}}\,}
\newcommand{\co}[1]       {\relax}
\newcommand{\nl}          {\newline}
\newcommand{\el}          {\\\hline\\[-0.4cm]}}{}
% until here

\begin{document}

\section{Introduction}

Experiments with heavy-ion collisions at the LHC create and diagnose strongly-interacting matter under the most extreme conditions of density and temperature accessible in the laboratory. Aside of its intrinsic interest, this line of research is central to our understanding of the Early Universe and the evolution of ultra-dense stars. In practice, the main focus of experimentation with nuclear beams at the LHC is on learning how collective phenomena and macroscopic properties, involving many degrees of freedom, emerge under extreme conditions from the microscopic laws of strong interaction physics. In doing so, proton-nucleus (pA) and nucleus-nucleus (AA) collision experiments at the LHC aim at applying and extending the Standard Model of particle physics to matter properties that govern dynamically evolving systems of finite size.

Recent experiments with nuclear beams at collider energies have identified opportunities to further strengthen the connection between the rich phenomenology of ultra-dense matter, and its understanding in terms of the fundamental laws of strong interaction physics. Two broad classes of phenomena may be highlighted in this context.

First, the observation of flow-like phenomena in essentially all measured soft particle spectra and particle correlations lends strong support to understanding bulk properties of heavy-ion collisions in terms of viscous fluid dynamics. As fluid-dynamic evolution is solely based on combining conservation laws with thermodynamic transport theories that are calculable from first principles in quantum field theory. This provides an experimentally accessible inroad to constraining QCD matter properties via soft flow, correlation and fluctuation measurements. As further explained in this document, this motivates future improved measurements of flow and transport phenomena, including in particular measurements of soft heavy flavour and electromagnetic radiation.  It also motivates improved experimental control over the system size dependence of flow phenomena to better constrain under which conditions and in which kinematic regime ultra-relativistic pA and AA collisions show fluid dynamic behaviour and where this picture fails.

Second, the observation of quantitatively-large quenching phenomena in essentially all measured hard hadronic observables in AA collisions has established the feasibility of testing the produced QCD matter with a broad set of probes whose production rates are controlled with good precision with pp reference measurements and perturbative QCD calculations. Hard quarks and gluons are known to interact with the medium and they thus {\it probe} medium properties. As detailed in this document, important physics opportunities are related to analyzing hard probes in pA and AA collisions with the greater precision and kinematic reach accessible in future LHC runs. For instance, the identification of (Rutherford-type) large-angle jet-medium scattering could constrain the quasi-particle nature of the fluid-like medium. This is of central importance since it critically tests the working hypothesis that the matter produced in AA collisions is a fluid with a ratio of shear viscosity over entropy density $\eta/s$ close to the theoretical minimum value. Such a fluid would be void of quasi-particles, while QCD is definitive in predicting that a microscope with sufficiently high resolution will reveal partonic (quasi-) particle structure. Identifying the scale at which inner structure (such as quasi-particles and related non-vanishing mean free paths) arises would provide a microscopic understanding of how fluid-like behaviour arises in nucleus-nucleus collisions. Therefore, probing the inner workings of the QGP by resolving its properties at shorter and shorter length scales is one of the main motivations for future experimentation with hard probes. Also at intermediate transverse momentum, this document identifies important questions that will be accessible experimentally. For instance, more differential studies of quarkonium bound state dissociation as a function of transverse momentum, rapidity and system size are expected to yield further insights into the mechanisms of colour deconfinement and recombination. Moreover, there is the general question of how fluid-like phenomena at low momentum scales transition to quenching phenomena at intermediate and high momentum scales.

Capitalizing on previous discoveries at RHIC, the LHC experimental programme with Pb-Pb and p-Pb collisions has significantly advanced the state of the art in both the soft and the hard physics sector. In the soft physics sector, differential measurements at the LHC have in particular allowed for precise $\pT$, particle species and rapidity-dependent measurements of all higher flow harmonics $v_n$, their mode-mode couplings and the resulting reaction plane correlations. This flow systematics extends to charmed flavored hadrons and possibly even to beauty ones. It is at the basis of constraining QCD transport properties today. In the hard physics sector, the wider kinematic reach of LHC has given qualitatively novel access to quarkonium suppression and jet quenching phenomenology, including precision measurements of bottomonium, novel observations of charmonium enhancement, and a rich phenomenology of calorimetrically defined jets and jet substructures in nuclear collisions.

LHC experiments have also lead to surprises that pose significant novel challenges for the understanding of pA and AA collisions. Most notable in this context is the discovery that flow-like phenomena are not limited to nucleus-nucleus collisions but they persist with significant magnitude in pA collisions and in high-multiplicity pp collisions at the LHC and some of their signatures have been observed even in minimum-bias pp collisions. However, pp collisions are typically expected to show vanishingly small re-interaction rates between produced final state particles. In contrast, the perfect fluid paradigm that underlies the successful phenomenology of flow-like phenomena in nucleus-nucleus collisions implicitly assumes that re-interaction rates are so large that even the notions of quasi-particle and mean-free path become meaningless. Does the persistence of flow-like phenomena in pA and pp collisions indicate, in contrast to previous belief, that the perfect fluid paradigm applies to these smaller collision systems? Or, if the prefect fluid paradigm is not applicable to pp and pA collisions, is it conceivable that significant corrections to a fluid-dynamic picture of vanishing mean free path persist also in the larger AA collision systems? The LHC discoveries in pp and pA that give rise to these questions provide arguably the strongest motivation for a future programme of detailed experimentation that aims at constraining microscopic structures and length-scales in the produced QGP matter and that is expected to clarify in this way the microscopic mechanisms underlying the apparent fluid-dynamic behaviour of pp, pA and AA collisions.   

Historically, experimental heavy-ion programmes have always addressed a very diverse set of phenomenological opportunities. Some of the proposed experimental measurements have always reached out to other areas of science and could be clearly related to fundamental open question such as the origin of mass in the Universe, QCD deconfinement, or the determination of thermodynamic transport properties (that led in the past to unforeseen connections between string theory and the thermodynamics of quantum field theories). Other parts of the experimental programme were originally not related to a working hypothesis based on an open fundamental question, but they sometimes revealed themselves a posteriori as elements of crucial insight. This can be said for instance about the LHC pA programme, that was not part of the original LHC design, that was first conceived mainly as a set of benchmark measurements for establishing the cold nuclear matter baseline for interpretation of heavy ion data, and that has resulted in one of the most surprising discoveries made in the LHC nuclear beams programme. We therefore emphasize that heavy ion physics at the LHC, in the future as well as in the past, is likely to have multiple ways of reaching out and contributing to physics at large. At the time of writing this report, questions about the origin of collectivity in small pp and pA collision systems, and their implications for the interpretation of collective phenomena observed in AA, are arguably identified as the most pressing conceptual issue in the scientific debate. As outlined so far, they are clearly related to further experimentation with soft processes, and to research on the internal structure of QGP matter utilizing hard processes.  However, future experimentation at the LHC is not limited to this set of questions. From improved constraints on nuclear parton distribution functions that may inform us about the physics reach of future electron-ion facilities, to improved measurements of anti-nuclei, to ultra-peripheral collisions of electromagnetic Weizs\"acker-Williams photons at unprecedented field strength, to the search for qualitatively novel signatures of ultra-strong QED magnetic fields, the LHC nuclear beams programme can provide new insight in a much broader range of subject areas. 
% The following itemized list highlights in more completeness research directions for which the future LHC programme offers unprecedented opportunities:

As detailed in this report, the HL/HE-LHC physics working group 5 has identified future physics opportunities for high-density QCD with ions and proton beams that can be grouped broadly into the following four goals that are coming now within experimental reach:

\begin{enumerate}
\item Characterizing the macroscopic long-wavelength QGP properties with unprecedented precision. 
\item Accessing the microscopic parton dynamics underlying QGP properties.
\item Developing a unified picture of particle production from small (pp) to larger (pA and AA) systems.
\item Probing parton densities in nuclei in a broad $(x,\,Q^2)$ kinematic range and searching for the possible onset of parton saturation.
\end{enumerate}
 
In the following, we provide short summaries about how the four general goals are addressed by the measurements discussed in this report.

\subsection{Macroscopic QGP properties}
At sufficiently long wavelength, essentially all forms of matter can be described by fluid dynamics. The observation of flow-like behaviour in nucleus-nucleus collisions and in smaller collision systems demonstrates that this universal long-wavelength limit of hot and dense QCD matter can be accessed experimentally at the LHC. This provides an experimental inroad to fundamental questions about QCD thermodynamics and hydrodynamics since i) the QGP properties entering a fluid dynamic description are calculable from first principles in quantum field theory, and ii) hydrodynamic long wavelength properties depend on the effective physical degrees of freedom in the plasma and they are thus sensitive to the microscopic dynamics that governs their interactions. The following properties of the QCD matter produced in TeV-scale collisions
are accessible via future measurements at the LHC

\begin{enumerate}
\item {\it Temperature}\\
Within the programme of determining the QCD equation of state, QCD lattice simulations at finite temperature have established since long a precise relation between the QCD energy density and pressure that determine the fluid dynamic expansion, and the temperature of QCD matter.  While energy density and pressure can be constrained experimentally by many measurements, an independent determination of temperature is of great value for testing the idea of local equilibration in heavy ion collisions or for establishing deviations thereof. Future LHC experiments will constrain  temperature and its time evolution with unprecedented precision, in particular via thermal radiation of real and virtual (dileptons) photons (Chapter~\ref{chapter:electromagnetic_radiation}) and via the measurement of the dissociation of bottomonium states (Chapter~\ref{sec:quarkonia}), for which the in-medium regeneration probability is expected to be small.  
\item {\it QCD phase transition at $\mu_B \simeq 0$}\\ Collisions at the LHC realize systems of close-to-zero baryo-chemical potential. QCD calculations on the lattice predict in this regime a smooth cross-over transition from a hot partonic plasma to a cold hadron gas. Fluctuation measures of conserved charges are sensitive to the characteristics of the phase transition. In future LHC experiments, they are accessible with unprecedented precision and completeness (Chapter~\ref{sec:lf}). 
\item {\it Viscosity and other QCD transport coefficients}\\ Existing flow measurements provide tight upper bounds on the value of $\eta/s$ and they have been a cornerstone in supporting the perfect fluid paradigm. In the future, measuring higher order cross correlations of flow coefficients will significantly extend this line of research.  In addition, we note that the value of $\eta/s$ can be related to the existence and size of the mean free path (isotropization length scale), which in turn results from the existence of (quasi)particle-like excitations in the produced matter. On the one hand, this motivates increasing the precision on eta/s with the aim of establishing the tightest lower bound on this quantity (Chapter~\ref{sec:flow}). On the other hand, this motivates detailed studies of flow in smaller systems with the idea of identifying the scale at which the system size becomes comparable to the mean free path (Chapter~\ref{sec:smallsystems}). 
\item {\it Heavy-quark transport coefficients}\\ Heavy quarks provide unique tools for testing collective phenomena in nuclear collisions. As they are produced in initial hard scattering processes and flavour is conserved throughout the collective dynamical evolution, they are the best experimental proxy to the idea of putting coloured test charges of well-defined mass into the medium and testing how they participate in the evolution. Of particular interest are precision measurements of transverse momentum anisotropies $v_n$ and nuclear modification factors of open heavy flavoured mesons that are known to constrain e.g.\,the heavy-quark diffusion coefficient $2\pi T D_s$ and its dependence on the temperature $T$, that can be compared to first-principle calculations of QCD on the lattice (Chapter~\ref{sec:hf}).
\item {\it Searching for transport phenomena related to the presence of strong electrodynamic fields.}\\ Heavy ion collisions produce the largest electromagnetic field of any system accessible to laboratory experiments. The field is largest in the early phase of the collision, thus the early-produced heavy quarks are expected to be the most sensitive to its strength (Chapter~\ref{sec:hf}).
As the maximal field strengths are estimated to be of order of the pion mass ($e\, B^2 \sim m_\pi^2$), effects are also likely to be present for light-flavour charged hadrons (Chapter~\ref{sec:flow}). Other measurements of interest include transport coefficients such as the electric conductivity with which the plasma responds to an electromagnetic field, and that are calculable within QCD. In addition, as a consequence of the chiral anomaly, QCD coupled to QED gives rise to {\it anomalous hydrodynamics} that displays various qualitatively novel phenomena, including for instance a component of the electromagnetic currents that flows parallel to the magnetic field. The existence of these anomalous phenomena follows from first principles in field theory and thermodynamics, but the size of potential experimental signatures is model dependent.  Beyond determining conventional QED transport phenomena, LHC allows to search with increased precision for these intriguing signatures of anomalous fluid dynamics (Chapter~\ref{sec:flow}). 
\end{enumerate}


\subsection{Accessing the inner workings of hot QCD matter}
Previous experiments at the LHC and at lower centre of mass energy have established that the QCD matter produced in nucleus-nucleus collisions is subject to strong collective evolution. However, the nature of the effective constituents of that matter, and its characteristic inner length scales (such as screening lengths or mean scattering times, if any) are not yet understood. The scale dependence of QCD implies that one must be able to resolve partonic constituents of hot QCD matter at sufficiently high resolution scale. This motivates the use of high-momentum transfer processes (hard probes) to study the inner workings of hot QCD matter. In addition, the current status of phenomenological modelling does not exclude the existence of a sizeable mean free path of hot QCD matter (which is assumed in transport model simulations of heavy ion collisions, but which is not assumed in almost perfect fluid dynamic models). This motivates to learn about the inner workings at hot QCD matter also from particle production at intermediate transverse momentum (such as heavy quark transport at intermediate and low $\pT$). Here, we highlight the following opportunities for further experimentation.
\begin{enumerate}
\item {\it Constraining with jet quenching the colour field strength of the medium}\\ In general, the fragments of jets produced in nucleus-nucleus collisions are medium-modified due to  interactions with the hot QCD matter. These jet quenching effects depend on the inner structure of that matter. In particular, the average medium-induced colour field strength experienced by the escaping jet can be quantified e.g.\,with the quenching parameter $\hat q$, which measures the average exchanged transverse momentum squared per unit path length. Experiments at the LHC will provide improved constraints on this field strength measurement (Chapter~\ref{sec:jets}).  Qualitatively novel opportunities for testing the time evolution of the medium opacity to hard partons could arise if boosted tops could be studied in nuclear matter. A run at the LHC with lighter (than Pb) nuclei, like e.g.\,$^{40}$Ar, would provide sufficient luminosity to this end, as well as largely enhanced kinematic and statistics reach for $\gamma$-jet and Z-jet recoil measurements (Chapter~\ref{sec:lighter}). The opportunities for boosted top measurements in Pb-Pb or lighter nuclei collisions at the HE-LHC are also discussed (Chapter~\ref{sec:helhc}).
\item {\it Constraining with jet substructure measurements the quasi-particle structure of QCD matter}\\ While $\hat q$  characterizes the effects of jet-medium interactions in the coherent regime in which individual constituents in the medium are not resolved, modern jet substructure measurements in conjunction with the increased rates of future LHC experiments are expected to access a regime of Rutherford-type large angle jet-medium scattering, in which the detection of recoil or of large angle deflections provides direct evidence for the microscopic structure of the produced matter (Chapter~\ref{sec:jets}).  
\item {\it Quarkonia: Bottomonium production tests colour screening and serves as thermometer}\\To dissociate heavy quark color singlet bound states, the bound state - medium interactions need to exchange sufficiently high-momentum gluons that resolve the bound state. As a consequence, the quarkonium dissociation depends sensitively on the temperature of the medium, and 
increasingly-tight bound states are expected to melt with increasing temperature (sequential quarkonia dissociation). Beyond testing this dynamical consequence of hot QCD matter with increased precision, the higher rates of future experiments will allow one to cross-correlate measurements of bottomonium suppression with other manifestations of collectivity, such as elliptic flow (Chapter~\ref{sec:quarkonia}).
\item {\it Quarkonia: Charmonium production tests colour screening and regeneration dynamics}\\ In close similarity to bottomonium, the medium-modification of charmonium bound states is sensitive to colour screening and it is subject to the same QCD dissociation dynamics. However, since charm quarks and anti-quarks are produced abundantly in nucleus-nucleus collisions at the TeV scale, $c$ and $\bar c$ produced in different hard processes can form bound states. Indications of this qualitatively novel bound state formation process are accessible at low and intermediate transverse momentum, and they motivate high-precision measurements of nuclear modification factor $R_{\rm AA}$ and elliptic flow $v_2$. In addition, future open heavy-flavour measurements reaching down to zero $\pT$ with high precision will help to determine the total charm cross section which is a central input for understanding regeneration effects (Chapter~\ref{sec:quarkonia} and \ref{sec:hf}). 
\item {\it Formation of hadrons and light nuclei from a dense partonic system}\\ The question of how a collective partonic system of many degrees of freedom evolves into the hadronic phase and produces colour singlet hadrons and light nuclei is essential for a complete dynamical understanding of nucleus-nucleus collisions. Recent LHC measurements in proton-proton and proton-nucleus collisions suggest that also in small-system hadronic collisions the hadronization process may be modified with respect to elementary $e^+e^-$ collisions. 
Future measurements at the LHC will enable comparative and multi-differential studies of these modifications with unprecedented precision, for both the heavy-flavour sector (strange charm and beauty mesons, charm and beauty baryons, see Chapter~\ref{sec:hf}) and for light nuclei and hyper-nuclei (Chapter~\ref{sec:lf}). For open heavy flavour, these measurements are also crucial to disentangle the role of mass-dependent radial flow and of recombination, as well as to constrain the parameters of hadronization in the models that are used to estimate QGP properties like the heavy-quark diffusion coefficients. For light nuclei, precise measurements of nuclei and hyper-nuclei with mass numbers 3 and 4 as well as possible observation of exotic baryonic states will address the question whether their production is dominated by coalescence of protons, neutrons and $\Lambda$ baryons or by statistical hadronization of a partonic system. These measurements, in addition to that of high-momentum deuteron production, also have important astrophysical implications (dense compact start, dark matter searches in the Cosmos).
\end{enumerate}


\subsection{Developing a unified picture of QCD collectivity across system size}
As discussed at the beginning of this chapter, recent LHC discoveries of signatures of collectivity in proton-nucleus and in proton-proton collisions question common believes. For the smallest collision systems, these measurements indicate that more physics effects are at work in multi-particle production than traditionally assumed in the modelling of proton-proton collisions. For the larger collision systems (proton-nucleus and nucleus-nucleus collisions), they question whether the origin of signatures of collectivity is solely (perfect) fluid dynamical, given that these signatures persist in proton-proton collisions. This raises important qualitative questions like: What is the smallest length scale on which QCD displays fluid dynamic behaviour? Is there a non-vanishing characteristic mean free path for the production of soft and intermediate $\pT$ hadrons, and if so, is it smaller or larger than the proton diameter? What are the novel physics concepts with which underlying event simulations in proton-proton collisions need to be supplemented (e.g. in multi-purpose event generators) to account for the totality of observed phenomena?  While some of these questions sound technical, it needs to be emphasized that the size of dissipative properties of QCD matter, such as its shear viscosity, are quantitatively related to the presence or absence of intrinsic length scales such as a mean free path. Any systematic experimental variation of the system size therefore relates directly to a search for intrinsic length scales that determine the dissipative properties of hot QCD matter. Within the present report, we identify in particular the following future opportunities for an improved understanding of the system size dependence of collective phenomena (Chapter~\ref{chapter:smallsystems}):

\begin{enumerate}
\item {\it Flow measurements in pp and pA systems: Onset and high order correlations}\\ While flow signals have been established in smaller collision systems in recent years, their detailed characterization lacks behind the state of the art achieved in nucleus-nucleus collisions. Future measurements will allow for characterizing higher-order cumulants in largely non-flow suppressed multi-particle correlations, and test whether there is a system size dependence in the characteristic correlations between different flow harmonics \vn, or the characteristic reaction plane correlations. 
\item {\it Flow of heavy flavour and quarkonium in smaller systems}\\  Is there a minimal system size needed to transport heavy quarks within a common flow field? Given that the local hard production of heavy flavour is expected to be independent of any collective direction, precision measurements of heavy-flavour flow in \pp and \pA collisions will provide decisive tests of heavy quark transport, thus experimentally addressing the question of how QCD flow field build up efficiently and on short length and time scales.
\item  {\it Strangeness production as a function of system size}\\ One of the recent surprising LHC discoveries that is not accounted for in traditional models of the \pp underlying event but that may be accounted for in a thermal picture is the smooth increase of strangeness with event multiplicity across system size. We discuss in detail how future measurement at the LHC, such as the study of strange D-mesons or baryons, can extend the systematics underlying this observation {\bf chapters 5 and 7.4}.
\item {\it Searching for the onset/existence of energy loss effects in small systems}\\ All dynamical models of collectivity involve final-state interactions. This implies the existence of jet-medium final-state interactions and, a fortiori, the existence of parton energy-loss effects. The latter have not been identified experimentally yet. In this report, we discuss novel opportunities to test for their existence, including tests in future \pPb collisions, as well as opportunities specific to \OO collisions.
\item {\it Searching for the onset/existence of thermal radiation in small systems}\\ If the collectivity observed in smaller collision systems is due to final-state isotropization and equilibration phenomena, it must be accompanied by thermal radiation. The search for the corresponding conceptually clean electromagnetic signatures, such as thermal dilepton and photon production in \pPb collisions, is an important part in developing a unified  picture of QCD collectivity. This report discusses the experimental opportunities in light of experimental upgrades.   
\end{enumerate}

\subsection{Nuclear parton densities and search for non-linear QCD evolution}
Future experiments at the LHC offer a variety of opportunities for precision measurements with nuclear beams. Here, we highlight three opportunities that are clearly related to the main physics challenges of the heavy-ion programme (Chapter~\ref{sec:smallx}):

\begin{enumerate}
\item {\it Precise determination of nuclear PDFs at high $Q^2$}\\ As high momentum transfer processes are short distance, their production is not affected by the long-wavelength particle excitations of the QCD matter in nuclear collisions. This implies that the primary production rates of hard process in nucleus-nucleus collisions are determined perturbatively, whereas their medium-modifications arise from traversing a dense QCD matter of considerable spatial extent and considerable colour field strength. For  a dynamical understanding of jet quenching, control over primary production rates is indispensable and this necessitates the knowledge of nuclear parton distribution functions. Global npdf-fits that reflect the current state of the art of nuclear parton distributions could be improved at the LHC in the near future at high $Q^2$ and $x\sim 10^{-3}$--$10^{-2}$ in particular with high-precision W, Z and dijets measurements in p-Pb and Pb-Pb collisions. High-luminosity Ar-Ar collisions would enable for the first time using top quarks to constrain nuclear PDFs at very high $Q^2$ and large $x\sim 10^{-2}$--$10^{-1}$ and would contribute to constraining experimentally the nuclear mass number $A$ dependence of nuclear PDFs. 
\item {\it Constraining nuclear PDFs at low $Q^2$}\\ Drell-Yan and photon measurements {\bf chapter 12.3} in p-Pb collisions could provide significantly improved constraints on the nuclear parton distribution functions at low $Q^2$ and low $x\sim 10^{-5}$--$10^{-3}$, where nuclear effects are larger. In addition, although so far global PDFs do not use measurements from ultra-peripheral collisions (UPC), it is thought that quarkonia and di-jet production in UPC can constrain nuclear PDFs in the future. This report identifies opportunities for the corresponding measurements. 
\item {\it Access to non-linear QCD evolution at small-x}\\ The scale-dependence of parton distribution functions is known to obey linear QCD evolution equations within a logarithmically wide range in $Q^2$ and $\ln x$. However, where partonic density in the incoming hadronic wave function are not perturbatively small, qualitatively novel non-linear density effects are expected to affect the QCD evolution. For $Q^2$ smaller than a characteristic saturation scale $Q^2_s(x)$, these effects are dominant, and as the saturation scale $Q^2_s(x)$ increases with decreasing $x$, one expects on general grounds at sufficiently small $\ln x$ a qualitatively novel saturation regime in which non-linear QCD evolution occurs at perturbatively large $Q^2$. Future measurements at LHC will provide novel test for these saturation effects with previously unexplored measurements. The larger centre-of-mass energy of the HE-LHC would extend the small-$x$ coverage by an additional factor 0.5. Measurements of relevance include in p-Pb collisions dilepton and photon production at small-$x$ and forward measurements of di-hadrons and di-jet correlations. The perspectives for such measurements are discussed. 
\end{enumerate}
\end{document}
