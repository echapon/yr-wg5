% don't remove the folling lines, and edit the defintion of \main if needed
\documentclass[../report.tex]{subfiles}
\providecommand{\main}{..}
\IfEq{\jobname}{\currfilebase}{\AtEndDocument{\biblio}}{}
% until here

\begin{document}

\section{Introduction}

Reference test with the ALICE LoI \cite{Abelev:1475243}.

Experiments with hadron-nucleus and nucleus-nucleus collisions at the LHC create and diagnose matter under the most extreme conditions of density and temperature accessible in the laboratory. Aside of its intrinsic interest, this line of research is central to our understanding of the Early Universe and the evolution of ultra-dense stars. In practice, the main focus of experimentation with nuclear beams at the LHC is on learning how collective phenomena and macroscopic properties, involving many degrees of freedom, emerge under extreme conditions from the microscopic laws of strong interaction physics. In doing so, hadron-nucleus and nucleus-nucleus collision experiments at the LHC aim at applying and extending the Standard Model of particle physics to matter properties that govern dynamically evolving systems of finite size.

Recent experiments with nuclear beams at collider energies have identified opportunities to further strengthen the connection between the rich phenomenology of ultra-dense matter accessible in nucleus-nucleus collisions, and its understanding in terms of the fundamental laws of strong interaction physics. Two broad classes of phenomena may be highlighted in this context.

First, the observation of flow-like phenomena in essentially all measured soft particle spectra and particle correlations lends strong support to understanding bulk properties of heavy ion collisions in terms of viscous fluid dynamics. As fluid dynamic evolution is solely based on combining conservation laws with thermodynamic transport theories that are calculable from first principles in quantum field theory, this provides an experimentally accessible inroad to constraining QGP matter properties via soft flow, correlation and fluctuation measurements. While phenomenologically successful descriptions of these observables require modeling to bridge between data and fundamental theory, recent progress in constraining transport properties like the ratio of shear viscosity over entropy density has established the feasibility of extracting QGP matter properties via a fluid dynamic analysis of fluctuations and correlations. As further explained in this document, this motivates now improved measurements of flow and transport phenomena, including in particular measurements of soft heavy flavor and electromagnetic radiation.  It also motivates improved experimental control over the system size dependence of flow phenomena to better constrain under which conditions and in which kinematic regime ultra-relativistic pA and AA collisions show fluid dynamic behavior and where this picture fails.

Second, the observation of quantitatively large quenching phenomena in essentially all measured hard hadronic spectra and particle correlations in AA collisions has established the feasibility of testing the produced QCD matter with a broad set of probes whose production rates are controlled externally with good precision. Hard processes are known to scatter on the medium and they thus {\it probe} medium properties. As detailed in this document, significant physics opportunities are related to analyzing hard probes in pA and AA collisions with the greater precision and kinematic reach accessible now. For instance, the identification of (Rutherford-type) large-angle jet-medium scattering could constrain the quasi-particle nature of the fluid-like matter produced in heavy ion collisions. This is of central importance since it critically tests the working hypothesis that the matter produced in AA collisions is a fluid of close to minimal shear viscosity. A fluid of minimal shear viscosity would be void of quasi-particles, while QCD is definitive in predicting that a microscope with sufficiently high resolution will reveal partonic (quasi-) particle structure. Identifying the scale at which inner structure (such as quasi-particles and related non-vanishing mean free paths) arises would provide a microscopic understanding of how fluid-like behavior arises in nucleus-nucleus collisions. Therefore, probing the inner workings of the QGP by resolving its properties at shorter and shorter length scales is one of the main motivations for future experimentation with hard probes. Also at intermediate transverse momentum, this document identifies important questions that will be accessible experimentally. For instance, more differential studies of quarkonium bound state dissociation as a function of transverse momentum, rapidity and system size are expected to yield further insights into the mechanisms of color deconfinement and recombination. Moreover, there is the general question of how fluid-like phenomena at low momentum scales transition to quenching phenomena at intermediate and high momentum scales.

Capitalizing on previous discoveries at RHIC, the LHC experimental programme with Pb-Pb and p-Pb collisions has significantly advanced the state of the art in both the soft and the hard physics sector. In the soft physics sector, differential measurements at the LHC have in particular allowed for precise pT, PID and rapidity-dependent measurements of all higher flow harmonics vn, their mode-mode couplings and the resulting reaction plane correlations. This is at the basis of constraining QCD transport properties today. In the hard physics sector, the wider kinematic reach of LHC has given qualitatively novel access to quarkonium suppression and jet quenching phenomenology, including precision measurements of bottomonium, novel observations of charmonium enhancement, and a rich phenomenology of calorimetrically defined jets and jet substructures in nuclear collisions.

LHC experiments have also lead to surprises that pose significant novel challenges for the understanding of pA and AA collisions. Most notable in this context is the discovery that flow-like phenomena are not limited to nucleus-nucleus collisions but they persist with significant magnitude in pA collisions and in high-multiplicity pp collisions at the LHC and some of their signatures have been observed even in minimum bias pp collisions. However, pp collisions are typically expected to show vanishingly small re-interaction rates between produced final state particles. In contrast, in the perfect fluid paradigm that underlies the current successful phenomenology of flow-like phenomena in nucleus-nucleus collisions implicitly assumes that re-interaction rates are so large that even the notion of quasi-particle and mean-free path becomes meaningless. Does the persistence of flow-like phenomena in pA and pp collisions indicate in contrast to previous belief that the perfect fluid paradigm applies to these smaller collision systems? Or, if the prefect fluid paradigm is not applicable to pp and pA collisions, is it conceivable that significant corrections to a fluid-dynamic picture of vanishing mean free path persist also in the larger AA collision systems? The LHC discoveries in pp and pA that give rise to these questions provide arguably the strongest motivation for a future programme of detailed experimentation that aims at constraining microscopic structures and length-scales in the produced QGP matter and that is expected to clarify in this way the microscopic mechanisms underlying the apparent fluid-dynamic behavior of pp, pA and AA collisions.   

Historically, experimental heavy-ion programmes have always addressed a very diverse set of phenomenological opportunities. Some of the proposed experimental measurements have always reached out to other areas of science and could be clearly related to fundamental open question such as the origin of mass in the universe, QCD deconfinement, or the determination of thermodynamic transport properties (that had led in the past to unforeseen connections between string theory and the thermodynamics of quantum field theories). Other parts of the experimental program were originally not related to a working hypothesis based on an open fundamental question, but they sometimes revealed themselves a posteriori as elements of crucial insight. This can be said for instance about the LHC pA programme, that was not part of the original LHC design, that was first conceived mainly as a set of benchmark measurements for establishing the cold nuclear matter baseline for interpretation of heavy ion data, and that has resulted in one of the most surprising discoveries made in the LHC nuclear beams programme to day. We therefore emphasize that heavy ion physics at the LHC, in the future as well as in the past, is likely to have multiple ways of reaching out and contributing to physics at large. At the time of writing this report, questions about the origin of collectivity in small pp and pA collision systems, and their implications for the interpretation of collective phenomena observed in AA, are arguably identified as the most pressing conceptual issue in the scientific debate. As outlined in this introductory narrative so far, they are clearly related to further experimentation with soft processes, and to research on the internal structure of QGP matter utilizing hard processes.  However, future experimentation at the LHC is not limited to this set of questions. From improved constraints on nuclear parton distribution functions that may inform us about the physics reach of future electron-ion facilities, to improved measurements of anti-nuclei, to ultra-peripheral collisions of electromagnetic Weizsäcker-Williams photons at unprecedented field strength, to the search for qualitatively novel signatures of ultra-strong QED magnetic fields, the LHC nuclear beams program can exploit unprecedented opportunities in a much broader range of subject areas. The following itemized list highlights in more completeness research directions for which the future LHC programme offers unprecedented opportunities:

\end{document}
