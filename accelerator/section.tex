% don't remove the folling lines, and edit the defintion of \main if needed
\documentclass[../report.tex]{subfiles}
\providecommand{\main}{..}
\IfEq{\jobname}{\currfilebase}{\AtEndDocument{\biblio}}{}
\IfEq{\jobname}{\currfilebase}{%file for shortcuts

\newcommand{\nch}{\ensuremath{N_{\mathrm {ch}}\xspace}}
\newcommand{\Ncoll}{\ensuremath{N_{\mathrm {coll}}}}
\newcommand{\Npart}{\ensuremath{N_{\mathrm {part}}}}
\newcommand{\dNdeta}{\mathrm{d}N_\mathrm{ch}/\mathrm{d}\eta}
\newcommand{\snn}         {\ensuremath{\sqrt{s_{\mathrm {NN}}}}}
\newcommand{\kT}          {\ensuremath{k_{\mathrm {T}}}}

\newcommand{\pp}          {pp}
\newcommand{\pPb}         {pPb}
\newcommand{\pA}          {pA}
\newcommand{\PbPb}        {PbPb}
\newcommand{\AuAu}        {AuAu}
\newcommand{\CuCu}        {CuCu}
\newcommand{\pAu}         {pAu}
\newcommand{\dAu}         {dAu}
\newcommand{\lsim}        {\,{\buildrel < \over {_\sim}}\,}
\newcommand{\gsim}        {\,{\buildrel > \over {_\sim}}\,}
\newcommand{\co}[1]       {\relax}
\newcommand{\nl}          {\newline}
\newcommand{\el}          {\\\hline\\[-0.4cm]}}{}
% until here


%<Notes>
% Dear contacts,

% Several chapters have started defining custom commands, and this lead to some compilation issues.

% I have now prepared a file commands.tex which is in the main folder. 
% This file is included when you compile the full report (report.tex), but also when you compile the separate sections.

% Please add your custom commands to this file. When you do that, please
% - check that a command does not already exist and can be used,
% - add a comment above your commands, e.g. "% heavy flavour"

% NB. pT is \pT (already predefined in a CERN report style file). Please replace in case you are using \pt

% Best regards,
% Jan Fiete
%</Notes>

% 05/09/18 09:31:59
% Hello John,

% Very good. In addition, we should write explictly in "your" chapter that we decided for the projections to use p = 1.5 ... 1.9 which converts into an increase of LNN of 8 ... 25.

% Best regards,
% Jan Fiete

\begin{document}
\section{Collider performance for heavy ions} 

\subsection{LHC performance with heavy ions up to Run 2}


This year's 4th one-month Pb-Pb run of the LHC will bring its Run~2 to an end and launch the hardware upgrades to the collider, and to the ALICE experiment, that should allow the full  ``HL-LHC'' heavy-ion performance to be delivered from 2021 onward.   
Indeed, if we compare the specifications of the 2004 LHC Design Report  \cite{lhc1}, for Pb-Pb collisions only with a peak luminosity of \lumival{1}{27},  much  of the upgraded performance is already in hand.   
Not only has that peak Pb-Pb luminosity goal been exceeded by a factor of 3.6, but the p-Pb collision mode---an upgrade beyond the initial design whose very feasibility was widely doubted---has yielded similarly high luminosity 
in multiple operating conditions  (see \cite{Jowett:2018yqk} and references therein). 
Table~\ref{tab:LHCparams} summarises the main parameters of the runs to date. 
Most recently, in 2017, the LHC has collided beams of Xe nuclei \cite{Schaumann:2018qat}, 
providing many of the new results presented in this conference.  
The goal for 2018 will be to complete the accumulation of goal 
of an integrated \PbPb\ 
luminosity of \qty{1}{nb^{-1}} to 
each of the ALICE, ATLAS and CMS experiments.  

	
\begin{table*}[t]
\caption{Representative simplified beam parameters 
at the start of the highest luminosity physics   fills, 
in conditions that lasted for $>\qty{5}{days}$,   
in each annual Pb-Pb and p--Pb run~\cite{ipac2011:Jowett:2011zz,Jowett:1492972,ipac2013:Jowett:1572994,ipac2016:PbPb2015,ipac2017:Jowett:2289686}.  
The original design values for Pb--Pb~\cite{lhc1} 
and p-Pb~\cite{Salgado:2011wc}  
and future upgrade Pb--Pb goals are also shown 
(in these columns the integrated luminosity goal is to be attained over the 4~P--Pb 
runs in the
{10-year periods} before and after 2020).
Peak and integrated luminosities are averages for ATLAS and CMS 
(ALICE being levelled).
The smaller luminosities delivered to LHCb from 2013--2016 and 
in the minimum-bias part of the run in 2016 are not shown.
Emittance and bunch length are RMS values. 
Single bunch parameters for p-Pb or Pb-p runs are generally for Pb.
The series of runs with $\sqrt{\sNN}=\qty{5.02}{TeV}$ also included p--p reference runs, not shown here.
Design and record achieved nucleon-pair luminosities are \protect\fbox{boxed} for easy comparison. 
The upgrade value is reduced by a factor $\simeq3$ from its potential value by levelling.
\label{tab:LHCparams}}	
\centering
\small
\vspace{1em}
\begin{tabular}{@{}l|cc|ccccc|c@{}}
 
\hline                   
  Quantity                        & \multicolumn{2}{c|}{``design''} & \multicolumn{5}{c|}{achieved} & upgrade \\
  \hline
Year                                & (2004)&(2011)  & 2010 & 2011 &2012--13& 2015 &2016& $\ge$2021 \\
Weeks in physics & - & - & 4 & 3.5& 3 & 2.5& 1, 2 &-\\
Fill no.  & & & 1541 & 2351 & 3544& 4720 &5562& -\\
%circumference $C$   & km            & \multicolumn{5}{c}{ 26.659 } \\
%%mass and charge numbers $A,Z$   & ...  & \multicolumn{3}{c}{--- 208, 82 ---} \\
Species                         &Pb--Pb&\textcolor{red}{p--Pb}& Pb--Pb&Pb--Pb & \textcolor{red}{p--Pb}& Pb--Pb &\textcolor{red}{p--Pb}&Pb--Pb \\
Beam energy \qty{E[Z}{TeV]}  & \multicolumn{2}{c|}{ 7 } & \multicolumn{2}{c}{3.5} & 4 & 6.37  & 4,6.5 & 7 \\
Pb beam energy \qty{E[A}{ TeV]}  & \multicolumn{2}{c|}{2.76} & \multicolumn{2}{c}{ 1.38 } & 1.58 &   2.51 & 1.58,2.56  & 2.76 \\
Collision energy \qty{\sqrt{\sNN}}{[TeV]}  & 5.52& \textcolor{red}{8.79}& \multicolumn{2}{c}{2.51}& \textcolor{red}{\textbf{5.02}} & \textbf{5.02}& \textcolor{red}{\textbf{5.02},8.16}& 5.52  \\
Bunch intensity \qty{N_b}{[10^8]}       & \multicolumn{2}{c|}{0.7} & 1.22 &1.07   & 1.2 & 2.0 &2.1 &1.8 \\
No,\ of bunches $k_b$             & 592& &137& 338  & 358  & 518 & 540 &1232 \\
%%Colliding bunches  $k_c$             & 592& & &    &   &  & 912\\
Pb norm. emittance \qty{\epsilon_N}{[\mu m]}  & \multicolumn{2}{c|}{ 1.5 }  & 2.& 2.0 & 2. & 2.1 & 1.6 &  1.65   \\
Pb bunch length \qty{\sigma_{z}}{m}   & \multicolumn{2}{c|}{0.08} & \multicolumn{5}{c|}{0.07--0.1} &  0.08 \\
\qty{\bstar}{[m]}&       \multicolumn{2}{c|}{0.5}& 3.5 & 1.0 & 0.8 & 0.8  & 10, 0.6  &0.5\\
%beam-beam parameter $\xi$/IP  $10^{-3}$   & 0.20 & 0.35 &   &  & 0.64 \\
Pb stored energy   MJ/beam    & 3.8 & 2.3 &0.65   & 1.9 &  2.77 & 8.6  & 9.7 &  21 \\
Peak  lumi. \qty{\LAA}{[10^{27}cm^{-2}s^{-1}]}   & 1& \textcolor{red}{150} &0.03  & 0.5  & \textcolor{red}{116}  & 3.6  &\textcolor{red}{850}& 6   \\
NN lumi. \qty{ \LNN}{[10^{30} {cm}^{-2} {s}^{-1}]}   &\fbox{43}   & \fbox{31}   & 1.3  & 22. & 24& \fbox{156} & \fbox{177}& \fbox{260}  \\
Integrated lumi./expt. [\qty{}{\mu b^{-1}}]& 1000 & \textcolor{red}{$10^5$}  &  9  & 160 & \textcolor{red}{32000} & 650 &\textcolor{red}{\enum{1.9}{5}} & $10^4$ \\
%Int. NN lumi./expt. [\qty{}{nb^{-1}}]& \enum{4.3}{4} & \enum{2.1}{4}  &  ~380  & ~6700 & 6650 & \enum{2.8}{4} &\enum{4.0}{4} & \enum{4.3}{5} \\
Int. NN lumi./expt. [\qty{}{nb^{-1}}]& 43000 & 21000  &  ~380  & ~6700 & 6650 & 28000& 40000 & \enum{4.3}{5} \\
 \hline
\end{tabular}
 
%%% $^{*}$ Pb and p beams reached 1577 and 4000~GeV/nucleon in 2012.
 
\end{table*} 


\subsection{Schedule for future LHC heavy-ion runs} 

Present nominal plan based on ALICE Letter of Intent 2012.

\subsection{Pb-Pb luminosity at HL-LHC} 

The LHC Design Report originally foresaw Pb-Pb collisions with a 
luminosity of  . 

A major step towards the future upgraded performance is the provision of sufficient total bunch intensity from the LHC injectors.  
A detailed specification of the requirements on the beams at LHC injection 
has been given 
\cite{HLLHCPbPbspec}
and is well on track to being fulfilled.  
The necessary single-bunch intensities have already been attained. 
Slip-stacking. 


The High-Luminosity (HL-LHC) upgrade of the LHC will implement a number of 
hardware upgrades of the present LHC that will enable a substantial increase in p-p luminosity after 
Long Shutdown~3 (LS3).   


\subsubsection{Upgrades during LHC Long Shutdown 2}
 
\subsection{p-Pb at HL-LHC} 
 
\subsection{Lighter nuclei at HL-LHC} \label{sec:lightions}

Assumptions.
Bunch-intensity scaling. 

The bunch intensity limits in the injectors depend largely on the ion charge which changes at the various stripping stages which must be optimised for space-charge limits, intra-beam scattering, efficiency of electron-cooling,   beam losses on residual gas and other effects in the ion source, Linac4,  LEIR, the PS and SPS.  
Given the uncertainties, the deliverable intensity for other species can only be determined after sufficient time spend commissioning and empirically optimising the many parameters and operating modes of the whole injector chain. 
To simplify present considerations, we postulate a simple form relating the number of ions per bunch, \Nb, to the well-established value for Pb beams  
% MathType!MTEF!2!1!+-
% feaagKart1ev2aaatCvAUfeBSjuyZL2yd9gzLbvyNv2CaerbwvMCKf
% MBHbqedmvETj2BSbqefm0B1jxALjhiov2DaerbuLwBLnhiov2DGi1B
% TfMBaebbnrfifHhDYfgasaacH8wrps0lbbf9q8WrFfeuY-Hhbbf9v8
% qqaqFr0xc9pk0xbba9q8WqFfea0-yr0RYxir-Jbba9q8aq0-yq-He9
% q8qqQ8frFve9Fve9Ff0dmeaabaqaciGacaGaaeqabaWaaqaafaaaka
% baaaaaaaaapeqaaiaad6eadaWgaaWcbaGaamOyaaqabaGccaGGOaGa
% amOwaiaacYcacaWGbbGaaiykaiaaxcW7cqGH9aqpcaWGobWaaSbaaS
% qaaiaadkgaaeqaaOGaaiikaiaaiIdacaaIYaGaaiilaiaaikdacaaI
% WaGaaGioaiaacMcadaqadaqaamaalaaabaGaamOwaaqaaiaaiIdaca
% aIYaaaaaGaayjkaiaawMcaamaaCaaaleqabaGaeyOeI0IaamiCaaaa
% aaa!4F11!
\begin{equation}
{\Nb}(Z,A)   = {\Nb}(82,208){\left( {\frac{Z}{{82}}} \right)^{ - p}} 
\label{eq:pscaling}
\end{equation}
Fitting such an expression to the limited information~\cite{Manglunki:2016qzl} from the few species used for SPS fixed-target in recent years (since the commissioning of the present ECR ion source and LEIR) yields a value of the fit parameter \(p=1.9\).    
Beam quality requirements for fixed-target beams are, of course, less stringent than for injection into the collider.
Fitting to the first commissioning of Xe beams for the LHC~\cite{Schaumann:2018qat}, on the other hand, gives a much less optimistic \(p=0.75\).   
Although this was the only occasion where any other species than Pb was delivered to the collider, only the simplest version of the injection scheme was used and it is clear that, given time, significantly higher intensities could be achieved.  
We consider that  \(1.5\le p\le 1.9\) corresponds to a representative range of possibilities that could be 
realised in fully-prepared future operation.  

In addition, we assume, conservatively, that the \emph{geometric} transverse beam emittances at the start of 
collisions will be equal to those of Pb beams~\cite{HLLHCPbPbspec} and  that a similar filling scheme 
(basic bunch-spacing of \qty{50}{ns}) can yield \(k_c=1120\)  bunch pairs colliding in ALICE.  


\begin{table}%% Table from HL-LHC species.nb for p=1.5 
	\begin{tabular}{|*{1}{*{7}{|l}|}|}
\speciesheader\\
\hline
$\gamma$                                                                       &  \(3760.\) & \(3390.\) & \(3760.\) & \(3470.\) & \(3150.\) & \(2960.\) \\
$\sqrt{s_{\text{NN}}}\text{/TeV}$                                              &  \(7.\) & \(6.3\) & \(7.\) & \(6.46\) & \(5.86\) & \(5.52\) \\
$\sigma _{\text{had}}\text{/b}$                                                &  \(1.41\) & \(2.6\) & \(2.6\) & \(4.06\) & \(5.67\) & \(7.8\) \\
$\sigma _{\text{BFPP}}\text{/b}$                                               &  \(2.36\times 10^{-5}\) & \(0.00688\) & \(0.0144\) & \(0.88\) & \(15.\) & \(280.\) \\
$\sigma _{\text{EMD}}\text{/b}$                                                &  \(0.0738\) & \(1.24\) & \(1.57\) & \(12.2\) & \(51.8\) & \(220.\) \\
$\sigma _{\text{tot}}\text{/b}$                                                &  \(1.48\) & \(3.85\) & \(4.18\) & \(17.1\) & \(72.5\) & \(508.\) \\
$N_b$                                                                          &  \(6.24\times 10^9\) & \(1.85\times 10^9\) & \(1.58\times 10^9\) & \(6.53\times 10^8\) & \(3.56\times 10^8\) & \(1.9\times 10^8\) \\
$\epsilon _{\text{xn}}\text{/$\mu $m}$                                         &  \(2.\) & \(1.8\) & \(2.\) & \(1.85\) & \(1.67\) & \(1.58\) \\
$f_{\text{IBS}}\text{/(m Hz)}$                                                 &  \(0.0662\) & \(0.0894\) & \(0.105\) & \(0.13\) & \(0.144\) & \(0.167\) \\
$W_b\text{/MJ}$                                                                &  \(68.9\) & \(45.9\) & \(43.6\) & \(32.5\) & \(26.5\) & \(21.5\) \\
$L_{\text{AA0}}/\text{cm}^{-2}s^{-1}$                                          &  \(1.46\times 10^{31}\) & \(1.29\times 10^{30}\) & \(9.38\times 10^{29}\) & \(1.61\times 10^{29}\) & \(4.76\times 10^{28}\) & \(1.36\times 10^{28}\) \\
$L_{\text{NN0}}/\text{cm}^{-2}s^{-1}$                                          &  \(3.75\times 10^{33}\) & \(2.06\times 10^{33}\) & \(1.5\times 10^{33}\) & \(9.79\times 10^{32}\) & \(7.93\times 10^{32}\) & \(5.88\times 10^{32}\) \\
$P_{\text{BFPP}}\text{/W}$                                                     &  \(0.0031\) & \(0.179\) & \(0.303\) & \(5.72\) & \(43.4\) & \(350.\) \\
$P_{\text{EMD1}}\text{/W}$                                                     &  \(4.98\) & \(16.5\) & \(16.9\) & \(40.5\) & \(76.7\) & \(141.\) \\
$\tau _{\text{L0}}\text{/h}$                                                   &  \(16.4\) & \(21.3\) & \(23.\) & \(13.5\) & \(5.87\) & \(1.57\) \\
$T_{\text{opt}}\text{/h}$                                                      &  \(9.04\) & \(10.3\) & \(10.7\) & \(8.23\) & \(5.42\) & \(2.8\) \\
\qty{\left\langle L_{\text{AA}}\right\rangle}{cm^{-2}s^{-1}}             &  \(8.99\times 10^{30}\) & \(8.34\times 10^{29}\) & \(6.17\times 10^{29}\) & \(9.46\times 10^{28}\) & \(2.23\times 10^{28}\) & \(3.8\times 10^{27}\) \\
$\left\langle L_{\text{NN}}\text{$\rangle $/}\text{cm}^{-2}s^{-1}\right.$      &  \(2.3\times 10^{33}\) & \(1.33\times 10^{33}\) & \(9.87\times 10^{32}\) & \(5.76\times 10^{32}\) & \(3.71\times 10^{32}\) & \(1.64\times 10^{32}\) \\
$\int _{\text{month}}L_{\text{AA}}\text{ dt/}\text{nb}^{-1}$                   &  \(1.17\times 10^4\) & \(1080.\) & \(799.\) & \(123.\) & \(28.9\) & \(4.92\) \\
$\int _{\text{month}}L_{\text{NN}}\text{ dt/}\text{pb}^{-1}$                   &  \(2980.\) & \(1730.\) & \(1280.\) & \(746.\) & \(481.\) & \(213.\) \\
$R_{\text{had}}\text{/kHz}$                                                    &  \(2.07\times 10^4\) & \(3340.\) & \(2440.\) & \(653.\) & \(270.\) & \(106.\) \\
$\mu$                                                                          &  \(1.64\) & \(0.266\) & \(0.194\) & \(0.0518\) & \(0.0215\) & \(0.00842\) \\
\end{tabular}

   \caption{Parameters and performance for a range of light nuclei with scaling parameter \(p=1.5\) in \ref{eq:pscaling} }
%\end{table}
\vspace{2em}
%\begin{table}%% Table from HL-LHC species.nb for p=1.5 
	\begin{tabular}{|*{1}{*{7}{|l}|}|}
\speciesheader\\
\hline$\gamma$                                                                       &    \(3760.\) & \(3390.\) & \(3760.\) & \(3470.\) & \(3150.\) & \(2960.\) \\
$\sqrt{s_{\text{NN}}}\text{/TeV}$                                              &    \(7.\) & \(6.3\) & \(7.\) & \(6.46\) & \(5.86\) & \(5.52\) \\
$\sigma _{\text{had}}\text{/b}$                                                &    \(1.41\) & \(2.6\) & \(2.6\) & \(4.06\) & \(5.67\) & \(7.8\) \\
$\sigma _{\text{BFPP}}\text{/b}$                                               &    \(2.36\times 10^{-5}\) & \(0.00688\) & \(0.0144\) & \(0.88\) & \(15.\)& \(280.\) \\
$\sigma _{\text{EMD}}\text{/b}$                                                &    \(0.0738\) & \(1.24\) & \(1.57\) & \(12.2\) & \(51.8\) & \(220.\) \\
$\sigma _{\text{tot}}\text{/b}$                                                &    \(1.48\) & \(3.85\) & \(4.18\) & \(17.1\) & \(72.5\) & \(508.\) \\
$N_b$                                                                          &    \(1.58\times 10^{10}\) & \(3.39\times 10^9\) & \(2.77\times 10^9\) & \(9.08\times 10^8\) & \(4.2\times 10^8\) & \(1.9\times 10^8\) \\
$\epsilon _{\text{xn}}\text{/$\mu $m}$                                         &    \(2.\) & \(1.8\) & \(2.\) & \(1.85\) & \(1.67\) & \(1.58\) \\
$f_{\text{IBS}}\text{/(m Hz)}$                                                 &    \(0.168\) & \(0.164\) & \(0.184\) & \(0.18\) & \(0.17\) & \(0.167\) \\
$W_b\text{/MJ}$                                                                &    \(175.\) & \(84.3\) & \(76.6\) & \(45.2\) & \(31.4\) & \(21.5\) \\
$L_{\text{AA0}}/\text{cm}^{-2}s^{-1}$                                          &    \(9.43\times 10^{31}\) & \(4.33\times 10^{30}\) & \(2.9\times 10^{30}\) & \(3.11\times 10^{29}\) & \(6.66\times 10^{28}\) & \(1.36\times 10^{28}\) \\
$L_{\text{NN0}}/\text{cm}^{-2}s^{-1}$                                          &    \(2.41\times 10^{34}\) & \(6.93\times 10^{33}\) & \(4.64\times 10^{33}\) & \(1.89\times 10^{33}\) & \(1.11\times 10^{33}\) & \(5.88\times 10^{32}\) \\
$P_{\text{BFPP}}\text{/W}$                                                     &    \(0.0199\) & \(0.601\) & \(0.935\) & \(11.\) & \(60.6\) & \(350.\) \\
$P_{\text{EMD1}}\text{/W}$                                                     &    \(32.\) & \(55.6\) & \(52.2\) & \(78.3\) & \(107.\) & \(141.\) \\
$\tau _{\text{L0}}\text{/h}$                                                   &    \(6.45\) & \(11.6\) & \(13.1\) & \(9.74\) & \(4.96\) & \(1.57\) \\
$T_{\text{opt}}\text{/h}$                                                      &    \(5.68\) & \(7.62\) & \(8.08\) & \(6.98\) & \(4.98\) & \(2.8\) \\
$\left\langle L_{\text{AA}}\text{$\rangle $/}\text{cm}^{-2}s^{-1}\right.$      &    \(4.54\times 10^{31}\) & \(2.45\times 10^{30}\) & \(1.69\times 10^{30}\) & \(1.68\times 10^{29}\) & \(2.95\times 10^{28}\) & \(3.8\times 10^{27}\) \\
$\left\langle L_{\text{NN}}\text{$\rangle $/}\text{cm}^{-2}s^{-1}\right.$      &    \(1.16\times 10^{34}\) & \(3.93\times 10^{33}\) & \(2.71\times 10^{33}\) & \(1.02\times 10^{33}\) & \(4.91\times 10^{32}\) & \(1.64\times 10^{32}\) \\
$\int _{\text{month}}L_{\text{AA}}\text{ dt/}\text{nb}^{-1}$                   &    \(5.89\times 10^4\) & \(3180.\) & \(2190.\) & \(218.\) & \(38.2\) & \(4.92\) \\
$\int _{\text{month}}L_{\text{NN}}\text{ dt/}\text{pb}^{-1}$                   &    \(1.51\times 10^4\) & \(5090.\) & \(3510.\) & \(1330.\) & \(636.\) &\(213.\) \\
$R_{\text{had}}\text{/kHz}$                                                    &    \(1.33\times 10^5\) & \(1.12\times 10^4\) & \(7540.\) & \(1260.\) & \(378.\) & \(106.\) \\
$\mu$                                                                          &    \(10.6\) & \(0.893\) & \(0.598\) & \(0.1\) & \(0.03\) & \(0.00842\) \\
\end{tabular}

   \caption{Parameters and performance for a range of light nuclei with scaling parameter \(p=1.9\) in \ref{eq:pscaling} }
\end{table}
 
 
 Collimation \cite{Hermes:2016axvg}


\subsection{Heavy-ion performance of HE-LHC} 

For the moment assume same injected beams as HL-LHC in the absence of work on 
further possible upgrades to the injectors. 

Arguments from FCC Week in Amsterdam. 
Modest factor in integrated luminosity.

BFPP power with Pb very high. 
Could be alleviated with lighters species.
Expect roughly similar species scaling to HL-LHC in Section~\ref{sec:lightions}.

\end{document}
