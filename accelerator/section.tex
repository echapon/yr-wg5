% don't remove the folling lines, and edit the defintion of \main if needed
\documentclass[../report.tex]{subfiles}
\providecommand{\main}{..}
\IfEq{\jobname}{\currfilebase}{\AtEndDocument{\biblio}}{}
\IfEq{\jobname}{\currfilebase}{%file for shortcuts

\newcommand{\nch}{\ensuremath{N_{\mathrm {ch}}\xspace}}
\newcommand{\Ncoll}{\ensuremath{N_{\mathrm {coll}}}}
\newcommand{\Npart}{\ensuremath{N_{\mathrm {part}}}}
\newcommand{\dNdeta}{\mathrm{d}N_\mathrm{ch}/\mathrm{d}\eta}
\newcommand{\snn}         {\ensuremath{\sqrt{s_{\mathrm {NN}}}}}
\newcommand{\kT}          {\ensuremath{k_{\mathrm {T}}}}

\newcommand{\pp}          {pp}
\newcommand{\pPb}         {pPb}
\newcommand{\pA}          {pA}
\newcommand{\PbPb}        {PbPb}
\newcommand{\AuAu}        {AuAu}
\newcommand{\CuCu}        {CuCu}
\newcommand{\pAu}         {pAu}
\newcommand{\dAu}         {dAu}
\newcommand{\lsim}        {\,{\buildrel < \over {_\sim}}\,}
\newcommand{\gsim}        {\,{\buildrel > \over {_\sim}}\,}
\newcommand{\co}[1]       {\relax}
\newcommand{\nl}          {\newline}
\newcommand{\el}          {\\\hline\\[-0.4cm]}}{}
% until here


%<Notes>
% Dear contacts,

% Several chapters have started defining custom commands, and this lead to some compilation issues.

% I have now prepared a file commands.tex which is in the main folder. 
% This file is included when you compile the full report (report.tex), but also when you compile the separate sections.

% Please add your custom commands to this file. When you do that, please
% - check that a command does not already exist and can be used,
% - add a comment above your commands, e.g. "% heavy flavour"

% NB. pT is \pT (already predefined in a CERN report style file). Please replace in case you are using \pt

% Best regards,
% Jan Fiete
%</Notes>

% 05/09/18 09:31:59
% Hello John,

% Very good. In addition, we should write explictly in "your" chapter that we decided for the projections to use p = 1.5 ... 1.9 which converts into an increase of LNN of 8 ... 25.

% Best regards,
% Jan Fiete


\begin{document}

\section{Heavy-ion performance of LHC, HL-LHC and HE-LHC} 

\subsection{Heavy-ion performance of LHC in Runs~1 and 2}


This year's 4th one-month Pb-Pb run of the LHC will bring its Run~2 to an end and launch the hardware upgrades to the collider, and to the ALICE experiment, that should allow the full  ``HL-LHC'' heavy-ion performance to be delivered from 2021 onward.   
Indeed, if we compare the specifications of the 2004 LHC Design Report  \cite{lhc1}, for Pb-Pb collisions only with a peak luminosity of \lumival{1}{27},  much  of the upgraded performance is already in hand.   
Not only has that peak Pb-Pb luminosity goal been exceeded by a factor of 3.6, but the p-Pb collision mode---an upgrade beyond the initial design whose very feasibility was widely doubted---has yielded similarly high luminosity 
in multiple operating conditions  (see \cite{Jowett:2018yqk} and references therein). 
Table~\ref{tab:LHCparams} summarises the main parameters of the runs to date. 
Most recently, in 2017, the LHC has collided beams of Xe nuclei \cite{Schaumann:2018qat}, 
providing many of the new results presented in this conference.  
The goal for 2018 will be to complete the accumulation 
of an integrated \PbPb\ 
luminosity of \qty{1}{nb^{-1}} to 
each of the ALICE, ATLAS and CMS experiments.  

	
\begin{table*}[t]
\caption{Representative simplified beam parameters 
at the start of the highest luminosity physics   fills, 
in conditions that lasted for $>\qty{5}{days}$,   
in each annual Pb-Pb and p--Pb run~\cite{ipac2011:Jowett:2011zz,Jowett:1492972,ipac2013:Jowett:1572994,ipac2016:PbPb2015,ipac2017:Jowett:2289686}.  
The original design values for Pb--Pb~\cite{lhc1} 
and p-Pb~\cite{Salgado:2011wc}  
and future upgrade Pb--Pb goals are also shown 
(in these columns the integrated luminosity goal is to be attained over the 4~P--Pb 
runs in the
{10-year periods} before and after 2020).
Peak and integrated luminosities are averages for ATLAS and CMS 
(ALICE being levelled).
The smaller luminosities delivered to LHCb from 2013--2016 and 
in the minimum-bias part of the run in 2016 are not shown.
Emittance and bunch length are RMS values. 
Single bunch parameters for \pPb\ or \Pbp\ runs 
are generally those of the Pb beam.
The series of runs with $\sqrt{\sNN}=\qty{5.02}{TeV}$ also included \pp\ reference runs, not shown here.
Design and record achieved nucleon-pair luminosities are \protect\fbox{boxed}, and some key \pPb\ parameters are 
set in \textcolor{red}{red type}, for easy comparison. 
The upgrade peak luminosity is reduced by a factor $\simeq3$ from its potential value by levelling.
\label{tab:LHCparams}}	
\centering
    {\small
        {\renewcommand{\arraystretch}{1.2}
\begin{tabular}{m{3.7cm}|cc|ccccc|c@{}}
\hline                   
  Quantity                        & \multicolumn{2}{c|}{``design''} & \multicolumn{5}{c|}{achieved} & upgrade \\
  \hline
Year                                & (2004)&(2011)  & 2010 & 2011 &2012--13& 2015 &2016& $\ge$2021 \\
Weeks in physics & - & - & 4 & 3.5& 3 & 2.5& 1, 2 &-\\
Fill no. (best)  & & & 1541 & 2351 & 3544& 4720 &5562& -\\
%circumference $C$   & km            & \multicolumn{5}{c}{ 26.659 } \\
%%mass and charge numbers $A,Z$   & ...  & \multicolumn{3}{c}{--- 208, 82 ---} \\
Species                         &\PbPb &\textcolor{red}{\pPb}& \PbPb&\PbPb & \textcolor{red}{\pPb}& \PbPb &\textcolor{red}{\pPb}& \PbPb \\
Beam energy \qty{E[Z}{TeV]}  & \multicolumn{2}{c|}{ 7 } & \multicolumn{2}{c}{3.5} & 4 & 6.37  & 4,6.5 & 7 \\
Pb beam energy \qty{E[A}{ TeV]}  & \multicolumn{2}{c|}{2.76} & \multicolumn{2}{c}{ 1.38 } & 1.58 &   2.51 & 1.58,2.56  & 2.76 \\
\raggedright Collision energy \qty{\sqrt{\sNN}}{[TeV]}  & 5.52& \textcolor{red}{8.79}& \multicolumn{2}{c}{2.51}& \textcolor{red}{\textbf{5.02}} & \textbf{5.02}& \textcolor{red}{\textbf{5.02},8.16}& 5.52  \\
\hline
Bunch intensity \qty{N_b}{[10^8]}       & \multicolumn{2}{c|}{0.7} & 1.22 &1.07   & 1.2 & 2.0 &2.1 &1.8 \\
No,\ of bunches $k_b$             & 592& &137& 338  & 358  & 518 & 540 &1232 \\
%%Colliding bunches  $k_c$             & 592& & &    &   &  & 912\\
Pb norm. emittance \qty{\epsilon_N}{[\mu m]}  & \multicolumn{2}{c|}{ 1.5 }  & 2.& 2.0 & 2. & 2.1 & 1.6 &  1.65   \\
Pb bunch length \qty{\sigma_{z}}{m}   & \multicolumn{2}{c|}{0.08} & \multicolumn{5}{c|}{0.07--0.1} &  0.08 \\
\qty{\bstar}{[m]}&       \multicolumn{2}{c|}{0.5}& 3.5 & 1.0 & 0.8 & 0.8  & 10, 0.6  &0.5\\
%beam-beam parameter $\xi$/IP  $10^{-3}$   & 0.20 & 0.35 &   &  & 0.64 \\
Pb stored energy   MJ/beam    & 3.8 & 2.3 &0.65   & 1.9 &  2.77 & 8.6  & 9.7 &  21 \\
\hline
Luminosity \qty{\LAA}{[10^{27}cm^{-2}s^{-1}]}   & 1& \textcolor{red}{150} &0.03  & 0.5  & \textcolor{red}{116}  & 3.6  &\textcolor{red}{850}& 6   \\
NN luminosity \qty{ \LNN}{[10^{30} {cm}^{-2} {s}^{-1}]}   &\fbox{43}   & \fbox{31}   & 1.3  & 22. & 24& \fbox{156} & \fbox{177}& \fbox{260}  \\
Integrated luminosity/experiment [\qty{}{\mu b^{-1}}]& 1000 & \textcolor{red}{$10^5$}  &  9  & 160 & \textcolor{red}{32000} & 650 &\textcolor{red}{\enum{1.9}{5}} & $10^4$ \\
%Int. NN lumi./expt. [\qty{}{nb^{-1}}]& \enum{4.3}{4} & \enum{2.1}{4}  &  ~380  & ~6700 & 6650 & \enum{2.8}{4} &\enum{4.0}{4} & \enum{4.3}{5} \\
Int. NN lumi./expt. [\qty{}{nb^{-1}}]& 43000 & 21000  &  ~380  & ~6700 & 6650 & 28000& 40000 & \enum{4.3}{5} \\
 \hline
\end{tabular}
}
    }
%%% $^{*}$ Pb and p beams reached 1577 and 4000~GeV/nucleon in 2012.
 
\end{table*} 


\subsection{Request and schedule for future LHC heavy-ion runs} 

Present nominal plan based on ALICE Letter of Intent 2012.

\subsection{\PbPb\ luminosity in Run~3 and HL-LHC}  \label{sec:HLPbPb} 
% John, Michaela
 
The High Luminosity LHC (HL-LHC) is an upgrade of the LHC  to achieve instantaneous \pp\ luminosities a factor of five larger than the LHC nominal value.  
Its operational phase is scheduled to start in LHC Run~4,  in the second half of the 2020s.   
The project also includes 
hardware upgrades  of the present LHC that will  
allow the LHC to 
operate with instantaneous \PbPb\ luminosities an order of magnitude larger than the nominal~\cite{lhc1}. 


A major step towards the future upgraded performance is the provision of sufficient total bunch intensity from the LHC injectors.  
A detailed specification of the requirements on the beams at LHC injection 
has been given 
\cite{HLLHCPbPbspec}
and is well on track to being fulfilled.  
The necessary single-bunch intensities have already been attained. 
Slip-stacking in SPS  
will be implemented in 2021. 



\subsubsection{Secondary beams from the IPs}
% John, Michaela

Ultra-peripheral electromagnetic interactions of Pb nuclei lead to lepton-pair production. Most of this is innocuous except for the (single) bound-free pair production (BFPP1):
\begin{equation}
^{208}\mathrm{Pb}^{82+} + ^{208}\mathrm{Pb}^{82+} \longrightarrow ^{208}\mathrm{Pb}^{82+} + ^{208}\mathrm{Pb}^{81+} +  e^+,
\end{equation}
in which the electron is bound to one nucleus.
As extensively discussed in e.g.~\cite{Klein:2000ba,BFPP2003,ref:HI2004,PhysRevLett.99.144801,BFPP2009,MSchaumannThesis}, the modified nuclei emerge from the collision point, as a narrow secondary beam with modified magnetic rigidity, following a dispersive trajectory that impacts on the beam screen in a superconducting magnet in the dispersion suppressor (DS) downstream.  
These secondary beams emerge in both directions from every interaction point (IP) where ions collide. 
Each carries a power of  
 \begin{equation}
 \mathrm{P_{BFPP}} = L \sigmaBFPP\Eb,
 \end{equation}
 where $L$ is the luminosity and  $\sigmaBFPP \simeq \qty{276}{b}$  is  the cross-section at the 2015/18 run energy of $\Eb=\qty{6.37Z}{TeV}$~\cite{IPAC16_PbPbRun2015,HMeier}.
These losses carry much greater power than the luminosity debris
(generated by the nuclear collision cross-section of \qty{8}{b})
and can quench magnets and directly limit luminosity. With a peak luminosity of \lumival{3.5}{27} each secondary beam carries $\mathrm{P_{BFPP}} \lesssim \qty{80}{W}$, which is enough to quench as demonstrated in 2015~\cite{IPAC16_BFPP_Quench_Test}.

To reduce the risk of quenching these magnets, orbit bumps were implemented around the impact locations in IP1 and IP5 in order to move the losses out of the dipole and into the adjacent connection cryostat (``missing dipole'' in the DS) that does not contain a superconducting magnet coil and therefor is less likely to quench. This technique was first used in 2015, but will become more important in Run~3 when the luminosity again increases but the dispersion suppressor collimators \cite{XXXX} are not yet installed around IP1 and IP5 (to be upgraded in LS3). 


\subsubsection{Collimation and intensity limit}
% Contribution from Roderik

\subsubsection{Upgrades during LHC Long Shutdown 2}
- ALICE and LHCb detector upgrate
- collimators in cryostat around IP2 \cite{Christinas IPAC paper about fluka studies}



\subsection{\pPb\ in Run~3 and HL-LHC}



Within colliding nuclei,
with charges $Z_1$, $Z_2$
and nucleon numbers ${A_1}$, ${A_2}$,
in  rings with magnetic field set for protons of momentum $p_p$\footnote{Conditions imposed by the two-in-one magnet design of the LHC.},
the  colliding nucleon pairs will have an average centre-of-mass energy
\begin{equation}\label{eq:sNN}
\sqrt{\sNN}  \approx 2c\,{p_p}\sqrt{\frac{{{Z_1}{Z_2}}}{{{A_1}{A_2}}}}
\approx 2c\,{p_p}
\begin{cases}
1 & \text{p-p}\\
0.628 & \text{p-Pb}\\
0.394 & \text{Pb-Pb}
\end{cases}
\end{equation}
and a central rapidity shift in the direction of the $(Z_1,A_1)$ beam   
\begin{equation} 
\yNN \approx \frac{1}{2}\log \left( \frac{{{Z_1}{A_2}}}{{{A_1}{Z_2}}}\right)
\approx
\begin{cases}
0 & \text{\pp, \PbPb}\\
0.465 & \text{p-Pb}\\
-0.465 & \text{Pb-p}
\end{cases}. 
\end{equation} 
We present parameters for operation at the nominal LHC momentum
\(p_p c=\qty{7}{TeV} \). 
%Figure~\ref{fig:FutureRuns} shows $\sqrt{\sNN} $ according to (\ref{eq:sNN}) for past and expected future runs of the LHC.


% Contribution from Marc
The injection and ramp of protons and lead ions with equal magnetic rigidity leads to moving long-range beam-beam encounters in the four interaction regions of the LHC. These beam-beam encounters were one of the reasons why the feasibility of p-Pb operation in the LHC was initially questioned. This effect has been proven small in the LHC and calculations have shown this effect also being negligible for the HL-LHC era despite larger bunch numbers and higher proton bunch intensities.

The dynamic range of the interlock strip-line BPMs, common for the lead and proton beam, limited the proton intensity to $N_b=5\times10^{10}$ protons per bunch during run 1. Gating the strip-line BPM read-out removed this constraint in 2016. The higher proton intensity of $N_b=2.8\times10^{10}$ resulted in increased luminosities at the IPs but also lead to the substantial deposition of collision debris from the Pb beam in the dispersion suppressors at ATLAS and CMS risking a beam dump. The respective collision debris collimators (TCLs), which could have intercepted emerging fragments from the IPs, were not commissioned for the 2016 p-Pb run. Adequate TCL settings are expected to neutralise these fragments and should allow for higher peak luminosities in the future.

A potential p-Pb run during Run 3 and beyond will greatly benefit from the longitudinal slip stacking in the SPS and the small $\beta^*$ in all experiments. The proton intensity cannot be pushed to values much larger than the maximum achieved in 2016 as bunches colliding in multiple IPs and especially in ATLAS and CMS will approach the interlock BPM threshold of $2\times 10^9$ charges per bunch quickly if not luminosity levelled. This would lead to an undesirable beam dump while ALICE is still luminosity levelled.  [\textit{EDIT: talk about optimal fill time vs interlock bpm time and alice levelling after discussed with John}]

In order to predict the potential performance of a future p-Pb run, the expected Pb-Pb filling pattern [\textit{EDIT:bibliography?}] is used providing 1136 collisions in ATLAS/CMS, 1120 collisions in ALICE and 81 collisions in LHCb. This approximation is made since the proton injection should be flexible enough to reproduce most of the Pb filling pattern. A simulation of the beam evolution based on ordinary differential equations including IBS and radiation damping leads to a luminosity evolution in the different IPs as displayed in Fig.~XXX. [\textit{EDIT: Discuss with John which levelling scenraio is reasonable. Discussion of the results goes here.}]

 
\subsection{Colliding lighter nuclei at HL-LHC} \label{sec:lightions}
 

The bunch intensity limits in the injectors depend largely on the ion charge which changes at the various stripping stages which must be optimised for space-charge limits, intra-beam scattering, efficiency of electron-cooling,   beam losses on residual gas and other effects in the ion source, Linac4,  LEIR, the PS and SPS.  
Given the uncertainties, the deliverable intensity for other species can only be determined after sufficient time spend commissioning and empirically optimising the many parameters and operating modes of the whole injector chain. 
To simplify present considerations, we postulate a simple form relating the number of ions per bunch, \Nb, to the well-established value
(\Nb(82,208)=\enum{1.9}{8}) for Pb beams  
\begin{equation}
{\Nb}(Z,A)   = {\Nb}(82,208){\left( {\frac{Z}{{82}}} \right)^{ - p}} 
\label{eq:pscaling}
\end{equation}
Fitting such an expression to the limited information~\cite{Manglunki:2016qzl} from the few species used for SPS fixed-target in recent years (since the commissioning of the present ECR ion source and LEIR) yields a value of the fit parameter \(p=1.9\).    
Beam quality requirements for fixed-target beams are, of course, less stringent than for injection into the collider.
Fitting to the first commissioning of Xe beams for the LHC~\cite{Schaumann:2018qat}, on the other hand, gives a much less optimistic \(p=0.75\).   
Although this was the only occasion where any other species than Pb was delivered to the collider, only the simplest version of the injection scheme was used and it is clear that, 
given time, significantly higher intensities could be achieved.  
We consider that  \(1.5\le p\le 1.9\) corresponds to a representative range of possibilities that could be 
realised in fully-prepared future operation.  


In addition, we make a number of simplifying assumptions to allow a simplified, yet meaningful, comparison between species 
\begin{itemize}
    \item  The     \emph{geometric} transverse beam emittances at the start of collisions will be equal to those of Pb beams~\cite{HLLHCPbPbspec}.
    This is justified, at least at the level of the LHC, since the scaling of intra-beam scattering with \(\Nb\), \(Z\) and \(A\), 
    given by  the parameter 
    $f_{\text{IBS}}\text{/(m Hz)}$  
    is generally smaller than for \Pb\ as long as 
    $p\lsim1.9$.  A similar scaling should hold in the injectors such as the SPS where intra-beam scattering may also blow up the emittances.  
    This ignores possible space-charge limits in the injectors which
    should also be considered once the appropriate stripping schemes and charge states have been defined. 
    
    \item Same filling scheme with \(k_c=\)
    \item No luminosity-levelling in any experiment.
     \item Fill length optimised for intensity evolution dominated by luminosity burn-off. 
     \item Equal operational efficiency of 50\%.     Following conventional practice for HL-LHC, the integrated luminosity for a 1-month run is estimated assuming back-to-back ideal fills of optimal length and a turn-around time of \qty{2.5}{h} between the end  of one fill and the resumption of ``Stable Beams''  for collisions in the next.
\end{itemize} 
The parameters are estimated using analytical approximations unlike the more elaborate simulations used in Section~\ref{sec:HLPbPb}.  
Together with the assumption that there is no luminosity levelling, these lead to a higher estimate of integrated luminosity in a one-month run.  
Nevertheless they can be used as a guide to the relative gain factors in integrated nucleon-nucleon luminosity by changing from Pb to a lighter nucleus. 

\begin{table}%% Table from HL-LHC species.nb for p=1.5 
\centering
{\small
	\begin{tabular}{|*{1}{*{7}{|l}|}|}
\speciesheader\\
\hline
$\gamma$                                                                       &  \(3760.\) & \(3390.\) & \(3760.\) & \(3470.\) & \(3150.\) & \(2960.\) \\
$\sqrt{s_{\text{NN}}}\text{/TeV}$                                              &  \(7.\) & \(6.3\) & \(7.\) & \(6.46\) & \(5.86\) & \(5.52\) \\
$\sigma _{\text{had}}\text{/b}$                                                &  \(1.41\) & \(2.6\) & \(2.6\) & \(4.06\) & \(5.67\) & \(7.8\) \\
$\sigma _{\text{BFPP}}\text{/b}$                                               &  \(2.36\times 10^{-5}\) & \(0.00688\) & \(0.0144\) & \(0.88\) & \(15.\) & \(280.\) \\
$\sigma _{\text{EMD}}\text{/b}$                                                &  \(0.0738\) & \(1.24\) & \(1.57\) & \(12.2\) & \(51.8\) & \(220.\) \\
$\sigma _{\text{tot}}\text{/b}$                                                &  \(1.48\) & \(3.85\) & \(4.18\) & \(17.1\) & \(72.5\) & \(508.\) \\
$N_b$                                                                          &  \(6.24\times 10^9\) & \(1.85\times 10^9\) & \(1.58\times 10^9\) & \(6.53\times 10^8\) & \(3.56\times 10^8\) & \(1.9\times 10^8\) \\
$\epsilon _{\text{xn}}\text{/$\mu $m}$                                         &  \(2.\) & \(1.8\) & \(2.\) & \(1.85\) & \(1.67\) & \(1.58\) \\
$f_{\text{IBS}}\text{/(m Hz)}$                                                 &  \(0.0662\) & \(0.0894\) & \(0.105\) & \(0.13\) & \(0.144\) & \(0.167\) \\
$W_b\text{/MJ}$                                                                &  \(68.9\) & \(45.9\) & \(43.6\) & \(32.5\) & \(26.5\) & \(21.5\) \\
$L_{\text{AA0}}/\text{cm}^{-2}s^{-1}$                                          &  \(1.46\times 10^{31}\) & \(1.29\times 10^{30}\) & \(9.38\times 10^{29}\) & \(1.61\times 10^{29}\) & \(4.76\times 10^{28}\) & \(1.36\times 10^{28}\) \\
$L_{\text{NN0}}/\text{cm}^{-2}s^{-1}$                                          &  \(3.75\times 10^{33}\) & \(2.06\times 10^{33}\) & \(1.5\times 10^{33}\) & \(9.79\times 10^{32}\) & \(7.93\times 10^{32}\) & \(5.88\times 10^{32}\) \\
$P_{\text{BFPP}}\text{/W}$                                                     &  \(0.0031\) & \(0.179\) & \(0.303\) & \(5.72\) & \(43.4\) & \(350.\) \\
$P_{\text{EMD1}}\text{/W}$                                                     &  \(4.98\) & \(16.5\) & \(16.9\) & \(40.5\) & \(76.7\) & \(141.\) \\
$\tau _{\text{L0}}\text{/h}$                                                   &  \(16.4\) & \(21.3\) & \(23.\) & \(13.5\) & \(5.87\) & \(1.57\) \\
$T_{\text{opt}}\text{/h}$                                                      &  \(9.04\) & \(10.3\) & \(10.7\) & \(8.23\) & \(5.42\) & \(2.8\) \\
\qty{\left\langle L_{\text{AA}}\right\rangle}{cm^{-2}s^{-1}}             &  \(8.99\times 10^{30}\) & \(8.34\times 10^{29}\) & \(6.17\times 10^{29}\) & \(9.46\times 10^{28}\) & \(2.23\times 10^{28}\) & \(3.8\times 10^{27}\) \\
$\left\langle L_{\text{NN}}\text{$\rangle $/}\text{cm}^{-2}s^{-1}\right.$      &  \(2.3\times 10^{33}\) & \(1.33\times 10^{33}\) & \(9.87\times 10^{32}\) & \(5.76\times 10^{32}\) & \(3.71\times 10^{32}\) & \(1.64\times 10^{32}\) \\
$\int _{\text{month}}L_{\text{AA}}\text{ dt/}\text{nb}^{-1}$                   &  \(1.17\times 10^4\) & \(1080.\) & \(799.\) & \(123.\) & \(28.9\) & \(4.92\) \\
$\int _{\text{month}}L_{\text{NN}}\text{ dt/}\text{pb}^{-1}$                   &  \(2980.\) & \(1730.\) & \(1280.\) & \(746.\) & \(481.\) & \(213.\) \\
$R_{\text{had}}\text{/kHz}$                                                    &  \(2.07\times 10^4\) & \(3340.\) & \(2440.\) & \(653.\) & \(270.\) & \(106.\) \\
$\mu$                                                                          &  \(1.64\) & \(0.266\) & \(0.194\) & \(0.0518\) & \(0.0215\) & \(0.00842\) \\
\end{tabular}

	}
   \caption{Parameters and performance for a range of light nuclei with a moderately optimistic value of the scaling parameter \(p=1.5\) in \eqref{eq:pscaling}. }
\end{table}
%\vspace{1em}


\begin{table}%% Table from HL-LHC species.nb for p=1.5 
\centering
{\small
	\begin{tabular}{|*{1}{*{7}{|l}|}|}
\speciesheader\\
\hline$\gamma$                                                                       &    \(3760.\) & \(3390.\) & \(3760.\) & \(3470.\) & \(3150.\) & \(2960.\) \\
$\sqrt{s_{\text{NN}}}\text{/TeV}$                                              &    \(7.\) & \(6.3\) & \(7.\) & \(6.46\) & \(5.86\) & \(5.52\) \\
$\sigma _{\text{had}}\text{/b}$                                                &    \(1.41\) & \(2.6\) & \(2.6\) & \(4.06\) & \(5.67\) & \(7.8\) \\
$\sigma _{\text{BFPP}}\text{/b}$                                               &    \(2.36\times 10^{-5}\) & \(0.00688\) & \(0.0144\) & \(0.88\) & \(15.\)& \(280.\) \\
$\sigma _{\text{EMD}}\text{/b}$                                                &    \(0.0738\) & \(1.24\) & \(1.57\) & \(12.2\) & \(51.8\) & \(220.\) \\
$\sigma _{\text{tot}}\text{/b}$                                                &    \(1.48\) & \(3.85\) & \(4.18\) & \(17.1\) & \(72.5\) & \(508.\) \\
$N_b$                                                                          &    \(1.58\times 10^{10}\) & \(3.39\times 10^9\) & \(2.77\times 10^9\) & \(9.08\times 10^8\) & \(4.2\times 10^8\) & \(1.9\times 10^8\) \\
$\epsilon _{\text{xn}}\text{/$\mu $m}$                                         &    \(2.\) & \(1.8\) & \(2.\) & \(1.85\) & \(1.67\) & \(1.58\) \\
$f_{\text{IBS}}\text{/(m Hz)}$                                                 &    \(0.168\) & \(0.164\) & \(0.184\) & \(0.18\) & \(0.17\) & \(0.167\) \\
$W_b\text{/MJ}$                                                                &    \(175.\) & \(84.3\) & \(76.6\) & \(45.2\) & \(31.4\) & \(21.5\) \\
$L_{\text{AA0}}/\text{cm}^{-2}s^{-1}$                                          &    \(9.43\times 10^{31}\) & \(4.33\times 10^{30}\) & \(2.9\times 10^{30}\) & \(3.11\times 10^{29}\) & \(6.66\times 10^{28}\) & \(1.36\times 10^{28}\) \\
$L_{\text{NN0}}/\text{cm}^{-2}s^{-1}$                                          &    \(2.41\times 10^{34}\) & \(6.93\times 10^{33}\) & \(4.64\times 10^{33}\) & \(1.89\times 10^{33}\) & \(1.11\times 10^{33}\) & \(5.88\times 10^{32}\) \\
$P_{\text{BFPP}}\text{/W}$                                                     &    \(0.0199\) & \(0.601\) & \(0.935\) & \(11.\) & \(60.6\) & \(350.\) \\
$P_{\text{EMD1}}\text{/W}$                                                     &    \(32.\) & \(55.6\) & \(52.2\) & \(78.3\) & \(107.\) & \(141.\) \\
$\tau _{\text{L0}}\text{/h}$                                                   &    \(6.45\) & \(11.6\) & \(13.1\) & \(9.74\) & \(4.96\) & \(1.57\) \\
$T_{\text{opt}}\text{/h}$                                                      &    \(5.68\) & \(7.62\) & \(8.08\) & \(6.98\) & \(4.98\) & \(2.8\) \\
$\left\langle L_{\text{AA}}\text{$\rangle $/}\text{cm}^{-2}s^{-1}\right.$      &    \(4.54\times 10^{31}\) & \(2.45\times 10^{30}\) & \(1.69\times 10^{30}\) & \(1.68\times 10^{29}\) & \(2.95\times 10^{28}\) & \(3.8\times 10^{27}\) \\
$\left\langle L_{\text{NN}}\text{$\rangle $/}\text{cm}^{-2}s^{-1}\right.$      &    \(1.16\times 10^{34}\) & \(3.93\times 10^{33}\) & \(2.71\times 10^{33}\) & \(1.02\times 10^{33}\) & \(4.91\times 10^{32}\) & \(1.64\times 10^{32}\) \\
$\int _{\text{month}}L_{\text{AA}}\text{ dt/}\text{nb}^{-1}$                   &    \(5.89\times 10^4\) & \(3180.\) & \(2190.\) & \(218.\) & \(38.2\) & \(4.92\) \\
$\int _{\text{month}}L_{\text{NN}}\text{ dt/}\text{pb}^{-1}$                   &    \(1.51\times 10^4\) & \(5090.\) & \(3510.\) & \(1330.\) & \(636.\) &\(213.\) \\
$R_{\text{had}}\text{/kHz}$                                                    &    \(1.33\times 10^5\) & \(1.12\times 10^4\) & \(7540.\) & \(1260.\) & \(378.\) & \(106.\) \\
$\mu$                                                                          &    \(10.6\) & \(0.893\) & \(0.598\) & \(0.1\) & \(0.03\) & \(0.00842\) \\
\end{tabular}

	}
   \caption{Parameters and performance for a range of light nuclei with an optimistic value of the scaling parameter \(p=1.9\) in \eqref{eq:pscaling}. }
\end{table} 
 
 Collimation \cite{Hermes:2016axvg}

\subsection{Short run for \pO}
\label{sec:pOrun} %% Hans Dembinski can refer to this section  

As discussed in Section~
\ref{sec:pOcosmic}, a  short \pO\ collision run is of interest for cosmic-ray physics.  
This could be scheduled with rapid set-up in LHC on the successful 
model of the 2012 \pPb\ run\cite{} 
which was later re-used in the 
2017 \XeXe\ run\cite{Schaumann:2018qat}.  
Each of those runs took  about 16~h of LHC operation time, 
including set-up and physics data-taking. 

Because oxygen is used as the carrier gas in the CERN heavy-ion source, 
the idea has been mooted that 
it may be possible to switch from \Pb\ to
\isotope{16}{O}{8+}  beams for the LHC, and back,  
somewhat more rapidly than other species.  
However it appears that the process is still not quick enough to allow, say, 
a short p--O run at the end of one of the annual Pb-Pb runs.  
A more practical possibility would be to schedule the run 
earlier in the year in order to provide time for 
the source to be switched back to Pb operation
afterwards for the one-month run later in the year.   
Commissioning of the O beam in the injectors for single-bunch injection into the LHC
would need to be scheduled in parallel with p-p operation in the weeks before the \pO\ run. 

To keep the LHC commissioning time on the order of 8~h, as in the 2017 Xe--Xe run,  
the total charge per beam should be limited to similar values.  
This would lead to a filling scheme with 
\(\simeq 10\)~bunches. 
A  probable disadvantage of this scheme is that the \pO\  run 
would have to be done with the currently operational \pp\ optics with    \bstarval{10}, at IP2, rather than  \bstarval{0.5} as would be typical of an optics set up for heavy-ion operaion.  
This would mean that the ALICE experiment would receive less integrated luminosity than ATLAS and CMS, 
as  occurred in the 2017 \XeXe\ run~\cite{Schaumann:2018qat}.

\subsection{Heavy-ion performance of HE-LHC} 
% John and Michaela

For the moment assume same injected beams as HL-LHC 
in the absence of work on 
further possible upgrades to the injectors. 

Arguments from FCC Week in Amsterdam. 
Modest factor in integrated luminosity.

BFPP power with Pb very high. 
Could be alleviated with lighter species.
Expect roughly similar species scaling to HL-LHC in Section~\ref{sec:lightions}.


\end{document}
