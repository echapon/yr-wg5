% don't remove the following lines, and edit the definition of \main if needed
\documentclass[../report.tex]{subfiles}
\providecommand{\main}{..} 
\IfEq{\jobname}{\currfilebase}{\AtEndDocument{\biblio}}{}
\IfEq{\jobname}{\currfilebase}{%file for shortcuts

\newcommand{\nch}{\ensuremath{N_{\mathrm {ch}}\xspace}}
\newcommand{\dNdeta}{\mathrm{d}N_\mathrm{ch}/\mathrm{d}\eta}
}{}



\begin{document}

\section{Summary of luminosity requirements and proposed run schedule}
\label{sec:schedule}

The physics programme presented in this report requires data-taking campaigns with various colliding systems with centre-of-mass energies and integrated luminosities $L_{\rm int}$ as outlined in the following. In some cases the requirements are updated or new with respect to the present baseline LHC programme (see Section~\ref{sec:LHCpresentbaseline} and Ref.~\cite{Abelevetal:2014cna}). The main variations are: a much larger $L_{\rm int}$ target for p--Pb collisions, motivated by high-precision studies of both initial and final state effects, following the surprising discoveries of collective-like effects in small collision systems; a large sample of pp collisions at top LHC energy to reach the highest possible multiplicities with the smallest hadronic colliding system; moderate samples of O--O and p--O collisions, to study the onset of hot-medium effects and to tune cosmic-ray particle production models, respectively. Finally, as discussed in the previous chapter, extended LHC running with colliding intermediate-mass nuclei (as for example, Ar--Ar or Kr--Kr), offers the unique opportunity of a large increase in nucleon--nucleon luminosity to access novel probes of the QGP and to open a precision era for probes which are still rare with the Pb--Pb system. The working group considers the high-luminosity Pb--Pb and p--Pb programmes to be the priorities that should be achieved in Run-3 and Run-4. Intermediate-mass nuclei collisions are regarded as a compelling motivation to extend the heavy-ion programme at the LHC after LS4.   

\begin{itemize}

\item {\bf Pb--Pb at $\sqrtsNN=5.5$~TeV}, $L_{\rm int}=13~{\rm nb}^{-1}$ (ALICE, ATLAS, CMS), $2~{\rm nb}^{-1}$ (LHCb)

\item {\bf pp at $\sqrt s=5.5$~TeV}, $L_{\rm int}=600~{\rm pb}^{-1}$ (ATLAS, CMS), $6~{\rm pb}^{-1}$ (ALICE),  $50~{\rm pb}^{-1}$ (LHCb) 

\item {\bf pp at $\sqrt s=14$~TeV}, $L_{\rm int}=200~{\rm pb}^{-1}$ with low pileup (ALICE, ATLAS, CMS)

\item {\bf p--Pb at $\sqrtsNN=8.8$~TeV}, $L_{\rm int}=2~{\rm pb}^{-1}$ (ATLAS, CMS), $0.6~{\rm pb}^{-1}$ (ALICE, LHCb) 

\item {\bf pp at $\sqrt s=8.8$~TeV}, $L_{\rm int}=200~{\rm pb}^{-1}$ (ATLAS, CMS, LHCb), $3~{\rm pb}^{-1}$ (ALICE)

\item {\bf O--O at $\sqrtsNN=7$~TeV}, $L_{\rm int}=500~{\rm \mu b}^{-1}$ (ALICE, ATLAS, CMS, LHCb)

\item {\bf p--O at $\sqrtsNN=9.9$~TeV}, $L_{\rm int}=200~{\rm \mu b}^{-1}$ (ALICE, ATLAS, CMS, LHCb)

\item {\bf Ar--Ar or Kr--Kr at $\sqrtsNN\approx 6.4$~TeV}, e.g.\,$L_{\rm int}^{\rm Ar-Ar}=3~{\rm pb}^{-1}$ (about 3 months) gives NN luminosity equivalent to Pb--Pb with $L_{\rm int} =75$--$250~\invnb$

\end{itemize}

The proposed update running schedule is reported in the following table. The extended physics programme for Run-3 and Run-4 could be accommodated by increasing the heavy-ion running time from 4 to 6 weeks per year in 2022 and 2027. 



\begin{table}[!h]
\begin{center}
\begin{tabular}{llll}
Year & Systems, $\sqrtsNN$ & time & $L_{\rm int}$ \\
\hline
2021 &  Pb--Pb 5.5 TeV & 3 weeks & $2.3~\invnb$ \\
     &  pp 5.5 TeV & 1 week & $3~\invpb$ (ALICE), $300~\invpb$ (ATLAS, CMS), $25~\invpb$ (LHCb)  \\
\hline
2022 &  Pb--Pb 5.5 TeV & 5 weeks & $3.9~\invnb$ \\
     &  O--O, p--O & 1 week & $500~{\rm \mu b}^{-1}$ and $200~{\rm \mu b}^{-1}$ \\
\hline
2023 &  p--Pb 8.8 TeV & 3 weeks & $0.6~\invpb$ (ATLAS, CMS), $0.3~\invpb$ (ALICE, LHCb) \\
     &  pp 8.8 TeV & few days & $1.5~\invpb$ (ALICE), $100~\invpb$ (ATLAS, CMS, LHCb)  \\
\hline
2027 &  Pb--Pb 5.5 TeV & 5 weeks & $3.8~\invnb$ \\
     &  pp 5.5 TeV & 1 week & $3~\invpb$ (ALICE), $300~\invpb$ (ATLAS, CMS), $25~\invpb$ (LHCb)  \\
\hline
2028 &  p--Pb 8.8 TeV & 3 weeks & $0.6~\invpb$ (ATLAS, CMS), $0.3~\invpb$ (ALICE, LHCb) \\
     &  pp 8.8 TeV & few days & $1.5~\invpb$ (ALICE), $100~\invpb$ (ATLAS, CMS, LHCb)  \\
\hline
2029 &  Pb--Pb 5.5 TeV & 4 weeks & $3~\invnb$ \\
\hline
Run-5 &  Ar--Ar or Kr--Kr & 11 weeks & e.g.\,Ar--Ar $3~\invpb$ \\
     &  pp reference & 1 week & \\
\hline
\end{tabular}
\end{center}
\end{table}

\end{document}
