% don't remove the following lines, and edit the definition of \main if needed
\documentclass[../report.tex]{subfiles}
\providecommand{\main}{..} 
\IfEq{\jobname}{\currfilebase}{\AtEndDocument{\biblio}}{}
\IfEq{\jobname}{\currfilebase}{%file for shortcuts

\newcommand{\nch}{\ensuremath{N_{\mathrm {ch}}\xspace}}
\newcommand{\Ncoll}{\ensuremath{N_{\mathrm {coll}}}}
\newcommand{\Npart}{\ensuremath{N_{\mathrm {part}}}}
\newcommand{\dNdeta}{\mathrm{d}N_\mathrm{ch}/\mathrm{d}\eta}
\newcommand{\snn}         {\ensuremath{\sqrt{s_{\mathrm {NN}}}}}
\newcommand{\kT}          {\ensuremath{k_{\mathrm {T}}}}

\newcommand{\pp}          {pp}
\newcommand{\pPb}         {pPb}
\newcommand{\pA}          {pA}
\newcommand{\PbPb}        {PbPb}
\newcommand{\AuAu}        {AuAu}
\newcommand{\CuCu}        {CuCu}
\newcommand{\pAu}         {pAu}
\newcommand{\dAu}         {dAu}
\newcommand{\lsim}        {\,{\buildrel < \over {_\sim}}\,}
\newcommand{\gsim}        {\,{\buildrel > \over {_\sim}}\,}
\newcommand{\co}[1]       {\relax}
\newcommand{\nl}          {\newline}
\newcommand{\el}          {\\\hline\\[-0.4cm]}}{}



\begin{document}

\section{Summary of luminosity requirements and proposed run schedule}
\label{sec:schedule}

The physics programme presented in this report requires data-taking campaigns with various colliding systems with centre-of-mass energies and integrated luminosities \Lint as outlined in the following. In some cases the requirements are updated or new with respect to the present baseline LHC programme (see Sec.~\ref{sec:LHCpresentbaseline} and Ref.~\cite{Abelevetal:2014cna}). The main variations are: a much larger \Lint target for \pPb collisions, motivated by high-precision studies of both initial and final-state effects, following the surprising discoveries of collective-like effects in small collision systems; a large sample of pp collisions at top LHC energy to reach the highest possible multiplicities with the smallest hadronic colliding system; moderate-statistics samples of \OO and \pO collisions, to study the onset of hot-medium effects and to tune cosmic-ray particle production models, respectively. Finally, as discussed in Sec.~\ref{sec:smallAsum}, extended LHC running with colliding intermediate-mass nuclei (as, for example, \ArAr or \KrKr), offers the unique opportunity of a large increase in nucleon--nucleon luminosity to access novel probes of the QGP and to open a precision era for probes which are still rare with the \PbPb system. The working group considers the high-luminosity \PbPb and \pPb programmes to be the priorities that should be pursued in Run 3 and Run 4. High-luminosity runs with intermediate-mass nuclei are regarded as an appealing case for extending the heavy-ion programme at the LHC after LS4.
This case should be studied further from the theoretical and operational points of view, both of which could be informed with one or two pilot runs with different nuclear species. These studies would also indicate the optimal species for extended running.

\begin{itemize}

\item {\bf Pb--Pb at $\sqrtsNN=5.5$~TeV}, $\Lint=13~{\rm nb}^{-1}$ (ALICE, ATLAS, CMS), $2~{\rm nb}^{-1}$ (LHCb)

\item {\bf pp at $\sqrt s=5.5$~TeV}, $\Lint=600~{\rm pb}^{-1}$ (ATLAS, CMS), $6~{\rm pb}^{-1}$ (ALICE),  $50~{\rm pb}^{-1}$ (LHCb) 

\item {\bf pp at $\sqrt s=14$~TeV}, $\Lint=200~{\rm pb}^{-1}$ with low pileup (ALICE, ATLAS, CMS)

\item {\bf p--Pb at $\sqrtsNN=8.8$~TeV}, $\Lint=1.2~{\rm pb}^{-1}$ (ATLAS, CMS), $0.6~{\rm pb}^{-1}$ (ALICE, LHCb) 

\item {\bf pp at $\sqrt s=8.8$~TeV}, $\Lint=200~{\rm pb}^{-1}$ (ATLAS, CMS, LHCb), $3~{\rm pb}^{-1}$ (ALICE)

\item {\bf O--O at $\sqrtsNN=7$~TeV}, $\Lint=500~{\rm \mu b}^{-1}$ (ALICE, ATLAS, CMS, LHCb)

\item {\bf p--O at $\sqrtsNN=9.9$~TeV}, $\Lint=200~{\rm \mu b}^{-1}$ (ALICE, ATLAS, CMS, LHCb)

\item {\bf Intermediate AA}, e.g.\,$\Lint^{\rm Ar-Ar}=3$--$9~{\rm pb}^{-1}$ (about 3 months) gives NN luminosity equivalent to Pb--Pb with $L_{\rm int} =75$--$250~\invnb$

\end{itemize}

Based on this requirements, the proposed updated running schedule is reported in the following table. It can be seen that the physics programme for Run 3 and Run 4 discussed in this report is achievable by a modest increase of the ``heavy-ion running'' time from 12 to 14 weeks per run.

\begin{center}
\begin{tabular}{llll}
Year & Systems, $\sqrtsNN$ & Time & $\Lint$ \\
\hline
2021 &  Pb--Pb 5.5 TeV & 3 weeks & $2.3~\invnb$ \\
     &  pp 5.5 TeV & 1 week & $3~\invpb$ (ALICE), $300~\invpb$ (ATLAS, CMS), $25~\invpb$ (LHCb)  \\
\hline
2022 &  Pb--Pb 5.5 TeV & 5 weeks & $3.9~\invnb$ \\
     &  O--O, p--O & 1 week & $500~{\rm \mu b}^{-1}$ and $200~{\rm \mu b}^{-1}$ \\
\hline
2023 &  p--Pb 8.8 TeV & 3 weeks & $0.6~\invpb$ (ATLAS, CMS), $0.3~\invpb$ (ALICE, LHCb) \\
     &  pp 8.8 TeV & few days & $1.5~\invpb$ (ALICE), $100~\invpb$ (ATLAS, CMS, LHCb)  \\
\hline
2027 &  Pb--Pb 5.5 TeV & 5 weeks & $3.8~\invnb$ \\
     &  pp 5.5 TeV & 1 week & $3~\invpb$ (ALICE), $300~\invpb$ (ATLAS, CMS), $25~\invpb$ (LHCb)  \\
\hline
2028 &  p--Pb 8.8 TeV & 3 weeks & $0.6~\invpb$ (ATLAS, CMS), $0.3~\invpb$ (ALICE, LHCb) \\
     &  pp 8.8 TeV & few days & $1.5~\invpb$ (ALICE), $100~\invpb$ (ATLAS, CMS, LHCb)  \\
\hline
2029 &  Pb--Pb 5.5 TeV & 4 weeks & $3~\invnb$ \\
\hline
Run-5 & Intermediate AA & 11 weeks & e.g.\,Ar--Ar 3--9~$\invpb$ (optimal species to be defined) \\
     &  pp reference & 1 week & \\
\hline
\end{tabular}
\end{center}

\end{document}
