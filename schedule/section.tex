% don't remove the following lines, and edit the definition of \main if needed
\documentclass[../report.tex]{subfiles}
\providecommand{\main}{..} 
\IfEq{\jobname}{\currfilebase}{\AtEndDocument{\biblio}}{}
\IfEq{\jobname}{\currfilebase}{%file for shortcuts

\newcommand{\nch}{\ensuremath{N_{\mathrm {ch}}\xspace}}
\newcommand{\dNdeta}{\mathrm{d}N_\mathrm{ch}/\mathrm{d}\eta}
}{}



\begin{document}

\section{Summary of luminosity requirements and proposed run schedule}
\label{sec:schedule}

The physics programme presented in this report requires data-taking campaigns with various colliding systems with centre-of-mass energies and integrated luminosities $L_{\rm int}$ are outlined in the following. In some cases the requirements are {\it updated} or {\it new} with respect to the present baseline LHC programme (see Section~\ref{sec:LHCpresentbaseline} and Ref.~\cite{Abelevetal:2014cna}). The main variations are a much larger $L_{\rm int}$ target for p--Pb collisions, motivated by high-precision studies of both initial and final state effects, a large sample of pp collisions at top LHC energy to reach the highest possible multiplicities with the smallest hadronic colliding system, and moderate samples of O--O and p--O collisions, to study the onset of hot-medium effects and to tune cosmic-ray particle production models, respectively. Finally, as discussed in the previous section, extended LHC running with intermediate-mass colliding nuclei (as for example, Ar--Ar or Kr--Kr), offers the unique opportunity of a large increase in nucleon--nucleon luminosity to access novel probes of the QGP and to open a precision era for probes which are still rare with the Pb--Pb system. The working group considers the high-luminosity Pb--Pb and p--Pb programmes to be the priorities that should be achieved in Run-3 and Run-4. Intermediate-mass nuclei collisions are regarded as a compelling motivation to extended the heavy-ion programme at the LHC after LS4.   

\begin{itemize}

\item {\bf Pb--Pb at $\sqrtsNN=5.5$~TeV}, $L_{\rm int}=13~{\rm nb}^{-1}$ (ALICE, ATLAS, CMS), $3~{\rm nb}^{-1}$ (LHCb)

\item {\bf pp at $\sqrt s=5.5$~TeV}, $L_{\rm int}=600~{\rm pb}^{-1}$ (ATLAS, CMS), $6~{\rm pb}^{-1}$ (ALICE),  $200~{\rm pb}^{-1}$ (LHCb) 

\item {\bf pp at $\sqrt s=14$~TeV} ({\it updated}), $L_{\rm int}=200~{\rm pb}^{-1}$ with low pileup (ALICE, ATLAS, CMS)

\item {\bf p--Pb at $\sqrtsNN=8.8$~TeV}  ({\it updated}), $L_{\rm int}=2~{\rm pb}^{-1}$ (ATLAS, CMS), $0.6~{\rm pb}^{-1}$ (ALICE, LHCb) 

\item {\bf pp at $\sqrt s=8.8$~TeV}  ({\it new}), $L_{\rm int}=200~{\rm pb}^{-1}$ (ATLAS, CMS, LHCb), $3~{\rm pb}^{-1}$ (ALICE)

\item {\bf O--O at $\sqrtsNN=7$~TeV} ({\it new}), $L_{\rm int}=500~{\rm \mu b}^{-1}$ (ALICE, ATLAS, CMS, LHCb)

\item {\bf p--O at $\sqrtsNN=9.9$~TeV} ({\it new}), $L_{\rm int}=200~{\rm \mu b}^{-1}$ (ALICE, ATLAS, CMS, LHCb)

\item {\bf Ar--Ar or Kr--Kr} ({\it new}), e.g.\, $L_{\rm int}^{\rm Ar-Ar}=13~{\rm nb}^{-1}$ gives NN luminosity equivalent to Pb--Pb with $\Lin75-250

\end{itemize}



\end{document}
