% don't remove the following lines, and edit the definition of \main if needed
\documentclass[../report.tex]{subfiles}
\providecommand{\main}{..} 
\IfEq{\jobname}{\currfilebase}{\AtEndDocument{\biblio}}{}
\IfEq{\jobname}{\currfilebase}{%file for shortcuts

\newcommand{\nch}{\ensuremath{N_{\mathrm {ch}}\xspace}}
\newcommand{\Ncoll}{\ensuremath{N_{\mathrm {coll}}}}
\newcommand{\Npart}{\ensuremath{N_{\mathrm {part}}}}
\newcommand{\dNdeta}{\mathrm{d}N_\mathrm{ch}/\mathrm{d}\eta}
\newcommand{\snn}         {\ensuremath{\sqrt{s_{\mathrm {NN}}}}}
\newcommand{\kT}          {\ensuremath{k_{\mathrm {T}}}}

\newcommand{\pp}          {pp}
\newcommand{\pPb}         {pPb}
\newcommand{\pA}          {pA}
\newcommand{\PbPb}        {PbPb}
\newcommand{\AuAu}        {AuAu}
\newcommand{\CuCu}        {CuCu}
\newcommand{\pAu}         {pAu}
\newcommand{\dAu}         {dAu}
\newcommand{\lsim}        {\,{\buildrel < \over {_\sim}}\,}
\newcommand{\gsim}        {\,{\buildrel > \over {_\sim}}\,}
\newcommand{\co}[1]       {\relax}
\newcommand{\nl}          {\newline}
\newcommand{\el}          {\\\hline\\[-0.4cm]}}{}



\begin{document}

\section{Summary of luminosity requirements and proposed run schedule}
\label{sec:schedule}

The physics programme presented in this report requires data-taking campaigns with various colliding systems with centre-of-mass energies and integrated luminosities are outlined in the following. In some cases the requirements are {\it updated} or {\it new} with respect to the present baseline LHC programme (see Section~\ref{sec:LHCpresentbaseline} and Ref.~\cite{Abelevetal:2014cna}). The main variations are a much larger $L_{\rm int}$ target for p--Pb collisions, motivated by high-precision studies of both initial and final state effects, a large sample of pp collisions at top LHC energy to reach the highest possible multiplicities with the smallest hadronic colliding system, and moderate samples of O--O and p--O collisions, to study the onset of hot-medium effects and to tune cosmic-ray particle production models, respectively. Finally, as discussed in the previous section, extended LHC running with intermediate-mass colliding nuclei (as for example, Ar--Ar), offers the unique opportunity of large increase in nucleon--nucleon luminosity to access novel probes of the QGP and to open a precision era for probes which are still rare with the Pb--Pb system.   

\begin{itemize}
\item[Pb--Pb at 5.5 TeV] hjhj 

\item 

\end{itemize}



\end{document}
